\documentclass[a4paper]{report}
\usepackage{../template}
\begin{document}
\section{Strukturtheorie zu Gruppen (``Einige Aussagen'')}
Sei im Weiteren $M$ ein Monoid, $G$ eine Gruppe und $X$ eine Menge.
\addcontentsline{toc}{section}{Wirkungen}
\begin{defi}[\textbf{Wirkung}]
\label{def:Wirkung}
  Eine Abbildung
  $$\lambda : M \times X \to X, (m,x) \mapsto m \cdot x := \lambda(m,x)$$
  heißt Linkswirkung (left action, Linksoperation) von $M$ auf $X$, wenn es gelten $\forall x \in X, m, m' \in M:$
  \begin{enumerate}[(i)]
    \item Neutrales Element: $e\cdot x = x$
    \item Assoziativität: $m\cdot (m'\cdot x) = (m\cdot m') \cdot x$
  \end{enumerate}
\end{defi}
\begin{bez*}
  Ist $M$ eine Gruppe, so heißt $\lambda$ auch Gruppenwirkung und $X$ heißt Links-$M$-Menge.
\end{bez*}
\begin{bem*} Analog kann man auch Rechtswirkungen
  $$\rho : X \times M \to X, (x,m) \mapsto x \cdot m$$
  definieren. (Axiome: $x\cdot e = c$ und $(x\cdot m)\cdot m' = x \cdot (m \cdot m')$)
\end{bem*}
\begin{bem*}[Übung]
  Jede Links-$G$-Wirkung kann man in eine Rechts-$G$-Wirkung überführen:
  zu $\lambda: G \times X \to X$ definiere $\rho : X \times G \to X$ durch
  $$\rho(x,g) := \lambda(g^{-1}, x) \iff x \cdot g := g^{-1} \cdot x$$
\end{bem*}
\begin{prop}[Alternative Beschreibung von Wirkungen] \item
\begin{enumerate}[(a)]
  \item Sei $\lambda: G \times X \to X$ eine Linkswirkung, dann ist
        $$\ph : G \to \Bij(X), g \mapsto (\ph_{g}: X \to X, x \mapsto gx)$$
        ein wohl-definierter Gruppenhomomorphismus.
  \item Sei $\ph : G \to \Bij(X)$ ein Gruppenhomomorphismus, dann ist
        $$\lambda: G \times X \to X, (g,x) \mapsto \ph(g)(x)$$
        eine Linkswirkung von $G$ auf $X$.
\end{enumerate}
\end{prop}
\begin{proof}[Beweis]
  \begin{enumerate}[(a)]
          \item Für $g \in G$ sei $\ph_{g} : X \to X, x \mapsto gx$, dann gelten:
          $\ph_{e}: X \to X, x \mapsto ex = x$ ist $\id_{X}$ (Axiom (i)), und
          $$(*) \quad \ph_{g} \circ \ph_{g'} = \ph_{gg'}$$
          denn $\forall x \in X:$
          $$(\ph_{g} \circ \ph_{g'})(x) = \ph_{g}(\ph_{g'}(x)) = g(g'x) \overset{(ii)} = (gg')x = \ph_{gg'}(x)$$
          Damit folgen:
          \begin{enumerate}[1.]
            \item $\ph_{g} \circ \ph_{g^{-1}} = \underbrace{\ph_{e}}_{\id_{X}} = \ph_{g^{-1}} \circ \ph_{g}$
            $\implies \ph_{g}$ ist eine bijektive Abbildung mit Inverse $\varphi_{g^{-1}}$, d.h. $$\ph : G \to \Bij(X), g \mapsto \ph_{g}$$ist wohl-definiert.
            \item $\ph$ ist ein Gruppenhomomorphismus: folgt aus $(*)$ (Verknüpfung in $\Bij(X)$ ist die Verkettung von Abbildungen.)
          \end{enumerate}
        \item Übung.
  \end{enumerate}
\end{proof}
\begin{bem*}
\begin{enumerate}[(a)]
  \item Das Analogon von Proposition 2 gilt auch für Monoide. Die Linkewirkungen eines Monoids $M$ auf $X$ entsprechen Monoidhomomorphismen $M \to (\Abb(X,X), \id_{X}, \circ)$
\item Eine Gruppe kann auch auf ``Objekten'' mit mehr Struktur als eine Menge wirken, z.B. auf eine Gruppe!
\end{enumerate}
\end{bem*}
\begin{bsp*}
$G$ wirkt auf eine Gruppe $N$ heißt, man hat einen Gruppenhomomorphismus $G \to \Aut(N)$ (vgl. Lemma 1.56)
\end{bsp*}
\addcontentsline{toc}{subsection}{Eigenschaften von Wirkungen}
\begin{defi}[Eigenschaften von Wirkungen]
  Sei $\lambda: G \times X \to X$ eine Linkswirkung von $G$ auf $X$.
  \begin{enumerate}[(a)]
\addcontentsline{toc}{subsection}{Bahn}
    \item Die \textbf{Bahn} zu $x \in X$ ist $Gx = \{gx \mid g \in G\}$. Die Länge der Bahn zu $x$ ist $\#Gx$
    \item $\lambda$ ist transitiv $\iff \forall y, z \in X \exists g \in G : gy=z \overset{\text{Übung}}\iff \forall y \in X : Gy = X \overset{\text{Übung}}\iff \exists x \in X : Gx = X$
    \item $\lambda$ ist $n$-fach transitiv ($n \in \N$), wenn für alle Paare von $n$-Tupeln $(x_{1}, ..., x_{n}), (y_{1}, ..., y_{n}) \in X^{n}$ mit $\#\{x_{1}, ..., x_{n}\} = \#\{y_{1}, ..., y_{n}\}$ gilt $\exists g \in G : gx_{i} = y_{i}, \forall i$.
    \item Die Wirkung heißt \textbf{treu}, wenn der induzierte Gruppenhomomorphismus $\ph: G \to \Bij(X)$ (aus Proposition 2) injektiv ist $$\overset{\text{Übung}} \iff \forall g \in G \setminus \{e\} : \exists x \in X : \underbrace{gX \ne X}_{\ph_{g}(x) \ne \id_{X}(x)}$$
  \end{enumerate}
\end{defi}
\begin{bsp}\item
\begin{enumerate}
  \item Ist $V$ ein $K$-Vektoraum, so wirkt das Monoid $(K, 1, \cdot)$ auf $V$ durch Skalarmultiplikation $(\lambda, v) \mapsto \lambda v$
  \item Die folgenden 3 Beispiele sind Linkswirkungen von $\GLn(K)$:
    \begin{enumerate}[(i)]
      \item $\GLn(K) \times K^{n} \to K^{n}, (g,v) \mapsto gv$.
            (Übung: Es gibt die Bahnen $\{0\}, K^{n} \setminus \{0\}$)
      \item Sei $\mathcal B = \{\text{geordnete Basen von } K^{n}\}$ und
            $$\GLn(K) \times  \mathcal B\to \mathcal B, (g, (b_{1}, ..., b_{n})) \mapsto (gb_{1}, ..., gb_{n})$$
            die Wirkung ist treu und transitiv.
      \item $\GLn(K) \times \End_{K}(K^{n}) \to \End_{K}(K^{n}), (A,B) \mapsto ABA^{-1}$ die Wirkung ist nicht treu $Z(\GLn(K))$ wirkt trivial. (Übung: Bahnen stehen in Bijektion zu den Frobeniusnormalformen von Matrizen.)
    \end{enumerate}
  \item $S_{n} \times \{1, ..., n\} \to \{1, ..., n\}, (\sigma, i) \mapsto \sigma(i)$ Wirkung ist treu und $n$-fach transitiv.
  \item Abstrakte Beispiele: Sei $H \le G$ eine Untergruppe.
        \begin{enumerate}[(i)]
          \item $\lambda: H \times G \to G, (h,g) \mapsto hg$. Die Bahnen sind die Mengen $Hg$, also die Rechtsnebenklassen zu $H$ (treu?) Menge der Rechtsnebenklassen $$\mfaktor HG := \{Hg \mid g \in G\}$$
          \item $\rho : G \times H \to G, (g,h) \mapsto gh$ Bahnen = Linksnebenklassen zu $H$ und $$\faktor GH = \{gH \mid g \in G\}$$
          \item $c_{\cdot}: G \times G \to G, (g, g') \mapsto gg'g^{-1}$ ist eine Linkswirkung, denn der nach Proposition 2 zugehörige Gruppenhomomorphismus ist $c: G \to \Aut(G), g \mapsto c_{g}$.
          \item $G \times \faktor GH \to \faktor GH, (g, g'H) \mapsto gg'H$ Die Klassen $gH$ heißen Linksnebenklassen wegen der Links-$G$-Wirkung auf ihnen.
        \end{enumerate}
\end{enumerate}
\end{bsp}

\begin{prop}
  Sei $X$ eine Links-$G$-Menge (zu der Wirkung $\lambda: G \times X \to X, (g,x), \mapsto gx$) definiere Relation $\sim$ auf $X$ durch
  $$x \sim y \iff \exists g \in G : gx = y$$
  dann gelten:
  \begin{enumerate}[(a)]
    \item $\sim$ ist eine Äquivalenzrelation.
    \item Die Äquivalenzklasse zu $x \in X$ bezüglich $\sim$ ist die Bahn $Gx$. Insbesondere ist $X$ die disjunkte Vereinigung seiner Bahnen. (Ist $(x_{i})_{i \in I}$ ein Repräsentantensystem der $G$-Bahnen, so gilt also $\#X = \sum_{i \in I}\#Gx$)
  \end{enumerate}
\end{prop}
\begin{proof}[Beweis]
  \begin{enumerate}[(a)]
    \item $\sim$ ist eine Äquivalenzrelation: Prüfe
          \begin{itemize}
          \item $\sim$ reflexiv: $ex = x \implies x \sim x.$
          \item $\sim$ symmetrisch: Gelte $x \sim y$, d.h. $\exists g \in G : gx = y$, dann gilt $x = ex = g^{-1}(gx) = g^{-1}y \implies y \sim x$.
            \item $\sim$ transitiv: Gelte $x \sim y$ und $y \sim z$, d.h. $\exists g, h' \in G : gx = y, g'y = z$
                  $$\implies (g'g)x = g'(gx) = g'y = z \implies x \sim z$$
          \end{itemize}
    \item Sei $x \in X$, dann ist $$\{y \in X \mid x \sim y\} = \{y \in X \mid \exists g \in G : y = gx\} = \{gx \mid g \in G\} = Gx.$$
  \end{enumerate}
\end{proof}
\addcontentsline{toc}{subsection}{Satz von Cayley}
\begin{satz}[\textbf{Satz von Cayley}]
\label{satz:Satz von Cayley}
Jede Gruppe $G$ (jedes Monoid $M$) ist isomorph zu einer Untergruppe (einem Untermonoid) von $(\Bij(G), \id_{G}, \circ)$ (bzw. $(\Abb(G,G), \id_{G}, \circ)$).
\end{satz}
\begin{proof}[Beweis](Für Gruppen, Rest ist eine Übung) Definiere die Wirkung $\lambda G \times G \to G, (g,h) \mapsto gh$, dann erhalten wir den induzierten Gruppenhomomorphismus $\varphi: G \to \Bij(G)$, wir zeigen $\varphi$ ist injektiv: Sei $g \in G \setminus \{e\}$, dann gilt $ge = g \ne e \implies$ Wirkung treu, also $\varphi$ ist ein Gruppenmonomorphismus. D.h. $G$ ``ist'' Untergruppe von $\Bij(G)$.
\end{proof}

\addcontentsline{toc}{subsection}{Stabilisator}
\begin{defi}[\textbf{Stabilisator}]
  Sei $X$ eine Links-$G$-Menge und $x \in X$, dann heißt
  \[G_{x} := \stb_{G}(x) := \{g \in G \mid gx = x\}\]
  \textbf{Stabilisator} von $x$ (unter $G$).
  Warnung: $G_{x} \ne G\cdot x$.
\end{defi}
\begin{bsp*}
  $\stb_{\Sn}(\{n\}) = \{\sigma \in \Sn \mid \sigma(n) = n\} \cong S_{n-1}$ mit der üblichen $\Sn$-Wirkung auf $\{1, ..., n\}$.
\end{bsp*}
\begin{ubng*}
  $G$-Wirkung auf einer Menge $X$ ist treu
  \[\iff \bigcap_{x \in X} \stb_{G}(x) = \{e\}\]
\end{ubng*}
\begin{prop}
  Sei $X$ eine links-$G$-Menge, $x \in X, g \in G$, dann gilt
  \begin{enumerate}[(a)]
    \item $\stb_{G}(x) \le G$ ist eine Untergruppe.
    \item $\stb_{G}(gx) = g\stb_{G}(x)g^{-1}$
  \end{enumerate}
\end{prop}
\begin{proof}[Beweis] \item
  \begin{enumerate}[(a)]
    \item $e \in \stb_{G}(x)$, denn $ex = x$. Seien $\ubr{g_{1}, g_{2} \in \stb_{G}(x)}_{\substack{\text{bedeutet } g_{1}x = x, g_{2} x= x}}$, zu zeigen ist $g_{1}^{-1}g_{2} \in \stb_{G}(x)$
          \[\overset{g_{1}^{-1}\cdot \underline\ }\imp x = ex = g_{1}^{-1}g_{1}x = g^{-1}x\]
          Damit gilt $(g_{1}^{-1}\cdot g_{2}^{-1})x = g_{1}^{-1}(g_{2}x) = g_{1}^{-1}x = x$
    \item Sei $h \in G$, dann:
          \[h \in \stb_{G}(gx) \iff hgx = gx \overset{g^{-1}\cdot \underline\ } \iff g^{-1}hgx = x \]
          \[ \iff g^{-1}hg \in \stb_{G}(x) \underset{\text{Konj. mit } g} \iff h \in g \stb_{G}(x)g^{-1}. \qedhere\]
  \end{enumerate}
\end{proof}

\addcontentsline{toc}{subsection}{Bahngleichung}
\begin{prop}[Bahngleichung]
  Sei $X$ eine links-$G$-Menge, $x \in X$, dann gilt:
  \begin{itemize}
    \item $\psi : \fak{G}{G_{x}} \to Gx, hG_{x} \mapsto hx$ ist wohl-definiert und eine Bijektion.
    \item Ist $G$ endlich, so folgt $\# G\cdot x = [G : G_{x}]$.
  \end{itemize}
\end{prop}
\begin{proof}[Beweis] \item
  \begin{itemize}
    \item $\psi$ injektiv und wohl definiert: Seien $g, h \in G$, dann
          \[hx = gx \iff g^{-1}hx = x \iff g^{-1}h \in G_{x} \le G\]
          \[\iff g^{-1}h G_{x} = G_{x} \iff hG_{x} = gG_x\]
    \item $\psi$ surjektiv nach Definition von $G\cdot x$.
    \item Aussage über Mächtigkeiten: $\psi$ bijektiv $\imp \# \fak G{G_{x}} = \#G \cdot x$.
  \end{itemize}
\end{proof}

\begin{bem*} Die Abbildung $\psi$ ist ein Homomorphismus von links-$G$-Mengen (ein Isomorphismus!), $\fak G{G_{x}}$ und $G \times x \subseteq X$ sind links-$G$-Mengen und $\psi$ erfüllt:
  \[\psi (g \cdot h G_{x}) = g\cdot  \psi (hG_{x})\]
  (beides ist $=gx \cdot x$)
\end{bem*}


\begin{defi}% definition 10
  Sei $X$ eine links-$G$-Menge,
  \begin{enumerate}[(a)]
\addcontentsline{toc}{subsection}{Freie Operation}
    \item Man sagt $G$ operiert \textbf{frei} auf $X \iff \forall x \in X : G_{x} = \{e\}$
\addcontentsline{toc}{subsection}{Fixpunkte}
    \item Die Menge der \textbf{Fixpunkte} der $G$-Wirkung ist \[X^{G} := \{x \in X \mid G_{x} = G\}\]
  \end{enumerate}
\end{defi}

\begin{bsp*}
$\GLn(K)$ operiert frei auf der Menge der geordneten Basen von $K^{n}$.
\end{bsp*}

\begin{kor}
  Sei $X$ eine links-$G$-Menge. Sei $x_{1}, ..., x_{n}$ ein Repräsentantensystem der Bahnen der Länge $\ge 2$. Dann:
  \begin{enumerate}[(a)]
  \item \(X = X^{G} \sqcup \bigsqcup_{i \in \{1, ..., n\}} G \cdot x_{i}\)
  \item \(\# X = \# X^{G} + \sum_{i \in \{1, ..., n\}}\ubr{[G:G_{x_{i}}]}_{=\#G \cdot x}\)
  \end{enumerate}
\end{kor}
\begin{proof}[Beweis]
Aus Proposition 5 folgt (a), Lemma 9 gibt (b).
\end{proof}
\begin{anw*}
Sei $X:= G$. Sei die $G$-Wirkung durch Konjugation gegeben, d.h. \[g \ubr {\circ}_{\text{Wirk.}} h = ghg^{-1}\]
\addcontentsline{toc}{subsection}{Konjugationsklasse}
Die Bahnen unter dieser $G$-Wirkung heißen \textbf{Konjugationsklassen}. Die Konjugationsklasse zu $h \in G = X$ ist \[G_{h} := \{ghg^{-1} \mid g \in G\}\]
Bahnen der Länge 1 sind Fixpunkte unter Konjugation mit allen $g \in G$
\[= \{h \in G \mid \forall g \in G : \ubr{ghg^{-1} = h}_{gh = hg} \} =: Z(G) \text{ das Zentrum von }G\]
Stabilisator zu $h \in G$ (unter Konjugationswirkung)
\[ = \{g \in G \mid ghg^{-1} = h\} = C_{G}(h) \text{ Zentralisator von } h\]
\end{anw*}
Aus Korollar 11 ergibt sich nun:

\addcontentsline{toc}{subsection}{Klassengleichung}
\begin{satz}[Klassengleichung]
  Sei $G$ endlich. Ist $g_{1}, ..., g_{n}$ ein Repräsentantensystem der Konjugationsklassen der Länge $\ge 2$, so gilt:
  \[\# \ubr{G}_{X} = \#\ubr{Z(G)}_{X^{G}} + \sum_{i = 1}^{n}[G : \ubr{C_{G}(g_{i})}_{C_{g}}]\]
\end{satz}

\addcontentsline{toc}{subsection}{p-Gruppe}
\begin{defi}[\textbf{$p$-Gruppe}]
Sei $p$ eine Primzahl, eine Gruppe $G$ heißt $p$-Gruppe $\iff \# = p^{m}$ füe ein $m \in \N$
\end{defi}
\begin{bsp*}
  \[\fak \Z{(p^{m})} \text{ oder } U_{3}(\Bbb F_{p}) =
    \l\{\begin{pmatrix}
      1 & a & b \\
      0 & 1 & c \\
      0 & 0 & 1
  \end{pmatrix} \ \middle | \ a, b, c \in \Bbb F_{p} \r\}\]
\end{bsp*}

\begin{kor}
  Ist $G$ eine $p$-Gruppe, so gilt $p | \#Z(G)$, (d.h. $Z(G)$ ist nicht-trivial und also eine $p$-Gruppe)
\end{kor}

\begin{proof}[Beweis]
  Seien $g_{1}, ..., g_{n}$ wie im Satz 12. Dann gilt: $C_{G}(g_{i}) < G$ ist eine echte Untergruppe. (sonst $g_{i} = Z(G)$, ist ausgeschlossen)
  \[\underset{\text{Lagrange}}\imp [G: C_{G}(g_{i})] \text{ teilt } \# G = p^{m}\]
  ist ungleich 1!
  \[ \imp p | [G: C_{G}(g_{i})], \forall i \in \{1, ..., n\}\]
  Klassengleichung modulo $p:$
  \[\ubr 0_{\# G} \cong \# Z(G) + \sum_{i=1}^n \ubr 0_{[G : C_{G}(g_{i})]} \mod p \imp p | \# Z(G). \qedhere\]
\end{proof}

\addcontentsline{toc}{subsection}{Satz von Cauchy}
\begin{ubng}[Satz von Cauchy](?) Sei $p$ eine Primzahl und $G$ endlich, dann gilt:
  \[p | \#G \imp \ex g \in G : \ord(g) = p.\]
  ($\imp \#G$ und $\#\exp(G)$ haben dieselben Primteiler)

  Idee: Verwende Induktion über $\#G$ und die Klassengleichung. In Induktionsschritt 2 Fälle:
  \begin{enumerate}
\item $\ex H < G$ echte Untergruppe mit $p | \#H$
\item $\neg \ex H < G$ echte Untergruppe mit $p | \#H$
  \end{enumerate}
  Im 2. Fall wende Klassengleichung mod $p$ an!
\end{ubng}

\section{Permutationsgruppen}
Sei $n \in \N$, $\Sn = \Bij(\{1, ..., n\})$, Notation für $\sigma \in \Sn$, d.h. $\sigma : \{1, ..., n\} \to\{1, ..., n\}$ bijektiv ist
\[\begin{pmatrix}
1 & 2 & \cdots & n \\
\sigma(1) & \sigma(2) & \cdots & \sigma(n) \\
  \end{pmatrix}\]
Dabei gilt: $(\sigma(1), ..., \sigma(n))$ ist eine Permutation von $\{1, ..., n\}$, d.h. \[\#\{\sigma(1), ..., \sigma(n)\} = n\]

\begin{kor}
  $\#S_{n} = n!$
\end{kor}
\begin{proof}[Beweis](Übung) Betrachte die möglichen ``Wertetabellen'' für Permutationen.
\end{proof}

\begin{defi}
  Für $\sigma, \tau \in \Sn$ definiere
  \begin{enumerate}[(a)]
\addcontentsline{toc}{subsection}{Träger}
    \item $\supp(\sigma) =$ \textbf{Träger} von $\sigma, \supp(\sigma) := \{i \in \{1, ..., n\} \mid \sigma(i) \ne i\}$
\addcontentsline{toc}{subsection}{disjunkte Permutationen}
    \item $\sigma$ und $\tau$ sind \textbf{disjunkt} $\iff \supp(\sigma) \cap \supp(\tau) = \emptyset$
  \end{enumerate}
\end{defi}

\begin{bem*}
$\supp(\sigma) = \emptyset \iff 0 = \id$
\end{bem*}

\begin{lemm}[Andere Interpretation des Trägers]
  Sei $\sigma \in \Sn$, dann gilt für die Wirkung von $\< \sigma \> : \supp(\sigma) =$ Vereinigung der Bahnen von $\< \sigma \>$ auf $\{1, ..., n\}$ der Länge $\ge 2$.
\end{lemm}
\begin{proof}[Beweis] \item
\begin{itemize}
\item ``$\subseteq$'': Sei $i \in \supp(\s) \imp \s(i) \ne i \imp \{i, \sigma(i), \sigma^{2}(i), ..., \sigma^{m}(i), ...\}$ ist Bahn von $\<\s\> = \{\s^{j} \mid j \in \Nn\} = \{\id, \s, ..., \s^{r-1}\}$ der Länge $\ge 2$. für $r = \ord(\s)$.
\item ``$\supseteq$'': Sei $i \notin \supp(\s) \imp \s(i) = i \imp \s^{j}(i) = i, \forall j \in \N \imp$ Bahn von $i$ unter $\<\s\>$ ist 1-elementig.
\end{itemize}
\end{proof}

\begin{kor}
  Für $\s \in \Sn$ gelten:
  \begin{enumerate}[(a)]
    \item $i \in \supp (\s) \iff \s(i) \in \supp(\s)$
    \item  Auf jeder $\<\s\>$-Bahn (durch $i \in \{1, ..., n\}$) wirkt $\s$ als ``zyklische Permutation'', d.h.
          \[\begin{tikzcd}
i_n:=i \arrow[r, maps to] & i_2=\sigma(i) \arrow[r, maps to] & i_3 = \sigma^2(i) \arrow[r, maps to] & \cdots \arrow[r, maps to] & i_r = \sigma^{r-1}(i) \arrow[llll, "\sigma"', "(\text{mit } \# \{1\cdots  n\} = r)", maps to, bend left]
\end{tikzcd}\]
  \end{enumerate}
\end{kor}
\begin{proof}[Beweis]
\begin{enumerate}[(a)]
  \item \[i \in \supp(\s) \imp \s(i) \ne i \underset{\s \text{ anwenden}} \imp \s(\s(i)) \ne \s(i) \imp \s(i) \in \supp(\s)\]
        Falls $\s(i) \in \supp(\s)$, so gilt $\s(\s(i)) \ne \s(i)$
        \(\underset{\s^{-1} \text{ anwenden}} \imp \s(i) \ne i\)
  \item Sei $r$ die Länge der Bahn durch $i$ unter $\<\s\>$. Dann sind $i_{j+1}:= \s^{j}(i), j = 0, ..., r-1$ \emph{paarweise verschieden}. Sonst $\ex 0 \le j_1 < j_{2} \le r-1$ mit $\s^{j_{1}}(i) = \s^{j_{2}}(i)$
        \[\underset{\s^{-1} \text{ anwenden}} \imp i = \s^{j_{2} - j_{1}}(i) \quad (*)\]
        \(\imp\) Bahn durch $i$ hat höchstens $j_{2} - j_{1} < r$ Elemente, die Bahn ist wegen $(*)$ \[= \{i, \s(i), ..., \s^{j_{2}-j_{1}}(i)\}\]
        Und nun: Wiederholtes Anwenden von $\s$ gibt den Zykel
        \[\begin{tikzcd}
i_1 \arrow[r, maps to] & i_2 \arrow[r, maps to] & \cdots \arrow[r, maps to] & i_r \arrow[lll, bend left]
\end{tikzcd} \qedhere \]
\end{enumerate}
\end{proof}

\begin{lemm}
  Sind $\s, \tau \in \Sn$ disjunkt, so gilt $\s \tau = \tau \s$.
\end{lemm}

\begin{proof}[Beweis]
  Zeige $\sigma \circ \tau = \tau \circ \s$ als Abbildungen $\{1, ..., n\} \to \{1, ..., n\}$, sei $i \in \{1, ..., n\}$
  \begin{itemize}
    \item Fall 1: $i \in \supp(\s) \imp \s(i) \in \supp(\s) \imp i, \s(i) \notin \supp(\tau)$.
          Also $\tau(i) = i, \tau(\sigma(i)) = \sigma(i)$
    \item Fall 2: $i \in \supp(\tau)$ analog zu Fall 1.
    \item Fall 3: $i \notin \supp(\s) \cup \supp(\tau) \imp \s(i) = i = \tau(i)$.
  \end{itemize}
Also $\s(\tau(i)) = \s(i) = i = \tau(i) = \tau(\sigma(i)).$
\end{proof}
(Folge: $\s, \tau$ disjunkt $\imp \ord(\s \tau) = \kgv(\ord(\s), \ord(\tau))$)

\begin{defi}
\addcontentsline{toc}{subsection}{Zykel/Transposition}
  Seien $i_{1}, ..., i_{r} \in \{1, ..., n\}$ paarweise verschieden. Der $r$-\textbf{Zykel} \[(i_{1} \ i_{2}\ \cdots \ i_{r})(j) =
  \begin{cases}
    j & j \notin \{i_{1}, ..., i_{r}\} \\
    i_{s+1} & j = i_{s} \ (s \in \{1, ..., n\}) \\
    i_{1} & j = i_{r}
  \end{cases}\]
$2$-Zykel heißen \textbf{Transposition}.
Konvention: $( \cdot ):= \id_{\{1, ..., n\}}$ (leerer Zykel).
Beachte:
\begin{enumerate}[(i)]
  \item $(i) = (\cdot)$ für $i \in \{1, ..., n\}$
  \item $\supp(i_{1} \ i_{2} \ \cdots \ i_{r}) =
        \begin{cases}
          \{i_{1}, ..., i_{r}\} & r \ge 2 \\
          \emptyset & r =1
        \end{cases}$
  \item $(i_{1} \ i_{2}\ \cdots \ i_{r}) = (i_{r} \ i_{1} \ i_{2} \ \cdots i_{r-1})$ (Notation ist nicht eindeutig, können Einträge zyklisch weiterschieben.)
        z.B.
        \[(1 \ 4\ 7) = (7\ 1\ 4) = (4\ 7\ 1)
= \begin{tikzcd}
             & 1 \arrow[rd] &              \\
7 \arrow[ru] &              & 4 \arrow[ll]
\end{tikzcd}
        \]
  \item $\ord(i_{1}\ \cdots \ i_{r}) = r$, z.B. $\ord(1 \ 2) = 2$
\end{enumerate}
\end{defi}

% Vorlesung 7
\addcontentsline{toc}{subsection}{Zykeldarstellung von Permutationen}
\begin{satz}[\textbf{Zykeldarstellung von Permutationen}]
  Sei $\s \in \Sn$, seien $I_{1}, ..., I_{t} \subseteq \{1, ..., n\}$ die paarweise verschiedenen Bahnen von $\< \s \>$ auf $\{1, ..., n\}$ der Länge $\ge 2$, dann:
  \begin{enumerate}[(a)]
    \item Für $j \in \{1, ..., t\}\ \ex !$ Zykel $\s_{j} \in \Sn$ mit $\supp(\s_{j}) =I_{j}$, und $\s_{j}|_{I_{j}} = \s|_{I_{j}}$
    \item $\s = \s_{1} \cdot ... \cdot \s_{t}$ und die $\s_{i}$ kommutieren paarweise.
    \item Die Darstellung in (b) ist eindeutig bis auf Permutation der Faktoren.
    \item Für $\s$ gilt: \(\ord(\s) = \kgv(\# I_{j} \mid j \in \{1, ..., t\})\)
  \end{enumerate}
\end{satz}
\begin{proof}[Beweis]
\begin{enumerate}[(a)]
  \item Sei $r_{j}$ die Länge von $I_{j}$. Sei $i_{j} \in I_{j}$, dann ist (vgl. Beweis von Korollar 19)
        \[\s_{j}:= (i_{j}, \s(i_{j}), \s^{2}(i_{j}), ..., \s^{r_{j}-1}(i_{j}) \in \Sn\]
        ein $r_{j}$-Zykel und $\sigma|_{I_{j}} = \s_{j}$
  \item Die $(\s_{j})$ kommutieren paarweise, denn deren Träger, die Mengen $I_j$, sind paarweise disjunkt.

        Um $\s = \s_{1} \cdot ... \cdot \s_{t}$ zu prüfen, wende beide Abbildungen an auf $i \in \{1, ..., n\}$.
        \begin{itemize}
          \item Fall $j \in \{1, ..., t\}: i \in J$

               $(*)$ Es gilt $\s_{j'}(i) = i$ für $j' \ne j$ (da $I_{j'} \cap I_{j} = \emptyset$)
                \[\imp \s(i) = \s_{j}(i) \overset{(*)} = \l(\s_{j} \cdot \textstyle \prod_{j' \ne j}\s_{j'}\r)(i)\]
                \[\overset{\s_{j} \text{ kommutieren}} = (\s_{1} \cdot ... \cdot \s_{j} \cdot ... \cdot \s_{t})(i)\]
          \item Fall $0: i \in \{1, ..., n\} \setminus \bigcup_{j \in \{1, ..., t\}} I_{j}$. Dann:
                \(\s(i) = i\) (1-elementige Bahn).

                Da $i \notin I_{j}: \s_{j}(i) = i, \forall j \in \eb t$. also $(\s_{1} \cdot \ldots \cdot \s_{t})(i) = i = \s(i)$
        \end{itemize}
  \item Es gelte $\s = \s_{1}' \cdot \ldots \s_{t'}'$ mit paarweise disjunkten Zykeln $\s = \s_{1}' \cdot \ldots \s_{t'}'$  der Länge $\ge 2$.
        Sei $I'_{j'} := \supp(\s'_{j'})$ für $j' \in \eb {t'}$. Dann:
        \[\s|_{I'_{j'}} = \s'_{j'}|_{I'_{j'}}\]
        \(\imp {I'_{j'}}\) ist Bahn von $\<\s\>$ der Länge $\ge 2$. $\imp t' = t$ und nach Umindizieren der $I'_{j'}$ gelte \[I'_{j} = I_{j} \text{ für } j \in \eb t\]
        und $\s_{j}|_{I_{j}} = \s|_{I_{j}} = \s'_{j}|_{I_{j}} \overunderset{\s_{j}, \s'_{j} \text{ sind}}{r_{j} = \#I_{j} \text{-Zykel}}\imp \s_{j} = \s'j$
  \item (Übung). \qedhere
\end{enumerate}
\end{proof}
\begin{bsp}
\[\s = \begin{pmatrix}
        1 & 2 & 3 & 4 & 5 & 6 & 7 & 8 \\
        2 & 5 & 8 & 4 & 1 & 6 & 3 & 7
      \end{pmatrix} \in S_{8}\]
    \(\imp \<\s\>\)-Bahnen: $\{1, 2, 5\}, \{3, 8, 7\}, \{4\}, \{6\}$ und $\sigma= (1\ 2\ 5)(3\ 8\ 7)$
\end{bsp}

\addcontentsline{toc}{subsection}{Young-Diagramm/Partition}
\begin{defi}[\textbf{Young-Diagramm/Partition}] Sei $\s \in \Sn$, seien $I_{1}, ..., I_{t}$ die Bahnen von $\<\s\>$ (auch Bahnen der Länge $1$), und gelte o.E. $\#I_{1} \ge \#I_{2} \ge \cdots \ge \#I_{t}$.
  \begin{enumerate}[(a)]
\item Das Young-Diagramm zu $\s$ ist das Diagramm der Form:
  \[\begin{ytableau}
     \ & \ & \ & \ & \ & \ & \ & \ & \none &  \none[\#I_{1}] \\
     \ & \ & \ & \ & \ & \none & \none & \none & \none &  \none[\#I_{2}] \\
     \none[\cdots] & \none & \none & \none & \none & \none & \none & \none & \none &  \none[\vdots]\\
     \ & \none & \none & \none & \none & \none & \none & \none & \none &  \none[\#I_{t}]
   \end{ytableau}\]
 im obigen Beispiel 23 \[\ydiagram{3, 3, 1, 1}\]
    \item Eine Partition von $n$ ist ein Tupel $(n_{1}, ..., n_{t})$ aus  $\N$ mit $n_{1} \ge \cdots \ge n_{t}$ unt $n = n_{1}, + \cdots + n_{t}$.
          (Young-Diagramm: Möglichkeit eine Partition zu veranschaulichen z.B. ist $(\#I_{1}, ..., \#I_{t})$ eine Partition von $n$)
  \end{enumerate}
\end{defi}

\begin{satz}[Übung] \item
\begin{enumerate}[(a)]
  \item Seien $i_{1}, ..., i_{r}$ aus $\eb n$ paarweise verschiedene Elemente. Dann gilt $\forall \s \in \Sn:$
        \[\s \circ (i_{1} \ i_{2} \cdots \ i_{r}) \circ \s^{-1} = (\s(i_{1}) \ \s(i_{2}) \cdots \ \s(i_{r}))\]
  \item $\s_{1}$ und $\s_{2}$ aus $\Sn$ liegen in dieselben Konjugationsklasse $\iff$ sie haben dasselbe Young-Diagramm.
\end{enumerate}
\end{satz}

\begin{bsp*}
  $S_{5}$ hat 7 Youngdiagramme
  \[
    \underline{\ydiagram{1, 1, 1, 1, 1}} \ \
    \ydiagram{2, 1, 1, 1} \ \
    \underline{\ydiagram{2, 2, 1}}\ \
    \underline{\ydiagram{3, 1, 1}}\ \
    \ydiagram{3, 2}\ \
    \ydiagram{4, 1}\ \
    \underline{\ydiagram{5}}
  \]
  also auch 7 Konjugationsklassen.
\end{bsp*}
\addcontentsline{toc}{subsection}{Signum-Funktion/Alternierende Gruppe}
\begin{defi*}[\textbf{Signum-Funktion/Alternierende Gruppe}]
  Sei $\sgn : \Sn \to \{\pm 1\}$ die Signum-Funktion aus der linearen Algebra. $\sgn$ ist eindeutig bestimmt durch:
  \begin{enumerate}[(i)]
    \item $\sgn$ ist ein Gruppenhomomorphismus.
    \item $\sgn(\tau) = -1$, für $\tau$ eine Transposition.
  \end{enumerate}
  (jedes $\s \in \Sn$ lässt sich schreiben als Produkt von Transpositionen)
  $A_{n} = \ker(\sgn) = $ die alternierende Gruppe auf $n$ Elementen
  \[A_{n} = \{\tau_{1} \cdot ... \cdot \tau_{2m} \mid \tau_{i} \in \Sn, \sgn(\tau) = -1, m \in \N\}\]
\end{defi*}

\begin{prop}[Formeln für $\sgn$] (Übung) \item
\begin{enumerate}[(a)]
  \item Jeder $r$-Zykel $\s$ ist ein Produkt von $r-1$ Transpositionen, und also gilt $\sgn(\s) = (-1)^{r-1}$
  \item Hat $\s$ die Zykeldarstellung $\s = \s_{1} \cdot ... \cdot \s_{t}$ mit Zykellängen $r_{i}$ (von $\s_{i}$), $i \in \eb t$, so gilt $\sgn(\s) = (-1)^{r_{1} + \cdots + r_{t} - t}$
\end{enumerate}
\end{prop}

\begin{bem*}
Man kann $\sgn$ durch (b) bestimmen und kann dann nachprüfen: $\s$ ist ein Gruppenhomomorphismus.
\end{bem*}

\begin{lemm}
  Sei $C_{3} = \{\s \in A_{n} \mid \s \text{ ist 3-Zykel}\}$ und sei $C_{2,2} = \{\s \in A_{n} \mid \s = \tau_{1} \cdot \tau_{2} \text{ mit } \tau_{1}, \tau_{2} \text{ disjunkt.}\}$, dann
 \begin{enumerate}[(a)]
\item Für $n \ge 3$ gilt $A_{n} = \<C_{3}\> =: H_{3}$
\item Für $n \ge 5$ gilt $A_{n} = \<C_{2, 2}\> =: H_{2, 2}$
\item Für $n \ge 5$ sind $C_{3}$ und $C_{2, 2}$ $A_{n}$-Konjugationsklassen.
 \end{enumerate}
\end{lemm}
% BEWEIS VON LEMMA 27
\begin{proof}[Beweis]
\[A_{n} = \{\ubr{\tau_{1} \cdot ... \cdot \tau_{2m}}_{\text{gerade Anzahl}} \mid \tau_{i} \in \Sn \text{ Transpositionen.}\}\]

\begin{enumerate}[(a)]
  \item Zeige: $\tau, \tau' \in H_{3}$ für $\tau, \tau'$ beliebige Transpositionen in $\Sn$
        \begin{enumerate}[(i)]
          \item $\tau = \tau'$:

                \(\tau \cdot \tau' = \id = \sigma^{3}\) für jeden 3-Zykel $\s \in H_{3}$
          \item $\tau \ne \tau'$ und $\tau, \tau'$ nicht disjunkt:

                also $\tau = (a\ b)$, $\tau' = (b \ c)$ mit $\#\{a, b, c\} = 3, a, b, c \in \eb n$.
                \[\tau \tau' = (a\ b\ c) = \underset{\substack{a \leftarrow b \leftarrow c \\ c \leftarrow c \leftarrow b \\ b \leftarrow a \leftarrow a}}{(a \ b)(b\ c)}\]
          \item $\tau \ne \tau'$ und $\tau, \tau'$ disjunkt
                also $\tau = (a\ b), \tau' = (c\ d), \#\{a, b, c, d\}=4, \{a, b, c, d\} \subseteq \eb n$.
                \[(a\ c\ b)(a\ c\ d) \overset{\text{(Übung)}}= (a\ b)(c\ d)\]
        \end{enumerate}
  \item Zeige $\tau \cdot \tau \in H_{2, 2}$ für $\tau, \tau' \in \Sn$ Transpositionen.
        \begin{itemize}
          \item Fall (iii) trivial.
          \item Fall (i) trivial \[(\tau_{1}\cdot \tau_{2})(\tau_{1}\cdot \tau_{2}) \in \<C_{2, 2}\>= H_{2, 2}\]
          \item Fall (ii) $\tau = (a\ b), \tau' = (b \ c)$ (wie oben). Wegen $n \ge 5$, finde $d\ne e \in \eb n \setminus \{a, b, c\}$. Dann
                \[\tau \cdot \tau' = ((a\ b)(d\ e))((b\ c)(d\ e))\]
        \end{itemize}
  \item $C_{3}$ ist $A_{n}$-Konjugationsklasse.

        Zu zeigen $(a\ b\ c)$ ($\{a, b, c\} \in \eb n$ 3 elementig) ist konjugiert zu  $(1\ 2\ 3)$.

        Wahle $\s \in \Sn$ mit $\s(1)=a, \s(2)=b, \s(3)=c$.
        \[\overset{\text{Satz 25}}\imp \s(1\ 2\ 3)\s^{-1} \overset{(*)} (\underbrace a_{{\s(1)}}\ \underbrace b_{{\s(2)}}\ \underbrace c_{{\s(3)}})\]
        Aber $\sgn(\s)$ ist unklar $+1, -1?$

        Beachte: $(*)$ gilt auch für $\s(4\ 5)$ und: entweder gilt $\sgn(\s) = 1$ oder $\sgn(\s(4\ 5)) = 1$
        \(\imp (1\ 2\ 3) \in A_{n}\) konjugiert zu $(a\ b\ c)$

        Für $C_{2, 2}$: zu zeigen $(a\ b)(c\ d)$ $A_{n}$-konjugiert zu $(1\ 2)(3\ 4)$ für $\{a, b, c, d\} \subseteq \eb n$ 4-elementig.

        Wähle $\s \in \Sn$ mit $\s(1)=a, \s(2)=b, \s(3)=c, \s(4)=d$
        \[\imp \s (1\ 2)(3\ 4)\s^{-1} \overset{(**)}= (a\ b)(c\ d)\]
        und $(*)$ gilt auch für $\s(1\ 2)$ ... etc. (Schließe wie für $C_{3}$.)
\end{enumerate}
\end{proof}

\addcontentsline{toc}{subsection}{Einfache Gruppe}
\begin{defi}[\textbf{Einfache Gruppe}]
  Eine Gruppe $G$ heißt \textbf{einfach} $\iff \{e\}$ und $G$ sind die einzigen Normalteiler von $G$. (d.h. $G$ hat keine nicht-trivialen Normalteiler)
\end{defi}

\begin{satz}
Für $n \ge 5$ ist $A_{n}$ einfach.
\end{satz}
\begin{proof}[Beweis]
  Sei $N \nt A_{n}$ ein Normalteiler und $\{e\} \subsetneq N$ und sei $\s \in N \setminus \{e\}$.
\begin{itemize}
\item $n = 5$:
  \[\underbrace{\ydiagram{2, 2, 1}}_{(*)}\ \ \underbrace{\ydiagram{3, 1, 1}}_{(**)} \ \ \ydiagram{5}\]
  $(*)$ Doppeltranspositionen bilden $A_{5}$-Konjugationsklasse und erzeugen $A_{5}$ (Lemma 27). Falls Doppeltranspositionen in $N$, so folgt $N = A_{5}$.
  \newline
  $(**)$ 3-Zykel bilden $A_{5}$-Konjugationsklasse und erzeugen $A_{5}$ (Lemma 27). Falls $\s$ ein $3-Zykel \imp N = A_{5}$.
  \newline
  Gelte $\s = 5$-Zykel $= (a\ b\ c\ d\ e)$. Nun:
  $N \ni \underbrace{(a\ b\ c)\s(a\ b\ c)^{-1}}_{\in N} \underbrace{\s}_{\in N}$
        \(\overset{\text{Übung}} = (a\ b\ d) \text{ 3-Zykel}\)
  \item $n=6$: möglichen Youngdiagramme: (zu $\s \in A_{6} \setminus \{e\}$)
        \[
        \overbrace{\ydiagram{2, 2, 1, 1}}^{(*)} \ \
        \overbrace{\ydiagram{3, 1, 1, 1}}^{(*)} \ \
        \ydiagram{3, 3} \ \
        \overbrace{\ydiagram{5, 1}}^{(*)}
        \]
        $(*)$ wurden schon im $A_{5}$-Fall erklärt.

        Sei also $\s^{2} = (a\ b\ c)(d\ e\ f) \in N$, mit $\{a, \ldots f\} = \eb 6$. Sei $\tau = (a\ b\ c)$, berechne $\tau(\s)(\tau^{-1})$ (Satz 25)
        \[\underbrace{\underbrace{\tau \s \tau^{-1}}_{\in N} \underbrace{\s}_{\in N}}_{\in N} = \underset{
        f\  \longleftarrow\
        f\  \longleftarrow\ \
        e\  \longleftarrow\
        e\  \longleftarrow\  f}{(b\ d\ c)(a\ e\ f)(a\ c\ b)(e\ d\ f)}
        \overset{\text{Übung}}=(a\ b\ e\ c\ d) \in 5-\text{Zykel}\]
        wurde schon bei $n=5$ geklärt.
  \item $n \ge 6:$ o.E. (Permutation von $1, ..., n$) $\s(1) \ne 1$
        Wähle $\{j, k\} \in \eb n \setminus \{1, \s(1)\}$. Sei $\tau:= (\s(1)\ j \ k) \imp \s^{-1}\tau \s \tau^{-1} \in N$
        Dann:
        \begin{enumerate}[(i)]
          \item $\ph := \tau \s \tau^{-1} \s^{-1} \in N$
          \item $\ph(\s(n)) = \tau \s \tau^{-1}(1) \overunderset{1 \notin \supp(\tau)}{1 \notin \supp(\tau^{-1})}=\tau \s(1) = j \ne \s(1)$, also $\ph \ne \id$.
          \item $\# \supp(\ph) \le 6$, denn:
                \[\ph = \underbrace{\tau}_{\text{3-Zykel}} \cdot \underbrace{\s\underbrace{\tau^{-1}}_{\text{3-Zykel}}\s^{-1}}_{\text{3-Zykel}}\]
                o.E: $\supp(\ph) \subseteq \eb 6 \imp \ph \in A_{6} \setminus \{e\}$
        \end{enumerate}
  \item Fälle $n \le 6$: Nurmalteiler, der von $\ph$ erzeugt wird enthält 3-Zykel oder Doppeltransposition.
        Dann fertig wegen Lemma 27.\qedhere
\end{itemize}
\end{proof}

\begin{bem*}
  Es gibt eine Klassifikation aller endlich einfachen Gruppen:
  Liste:
  \begin{itemize}
    \item $\fak \Z{(p)}, p$ prim
    \item $A_{n}, n \ge 5$
    \item endliche Gruppen vom Lie typ:
          \begin{enumerate}[(i)]
            \item $\fak {\SLn(K)}{Z(\SLn(K))}$ bis auf einige kleine $\#K$ sind einfach (endlich falls $K$ endlich).
            \item Weitere Untergruppen von $\SLn$, welche zu ``linearen algebraischen Gruppen'' korrespondieren.
          \end{enumerate}
          \item 26 weitere.
  \end{itemize}
\end{bem*}

\section{Sylow Theoreme}

\addcontentsline{toc}{subsection}{Sylow I}
\begin{satz}[Sylow I, nach Wieland]
  Sie $G$ eine endliche Gruppe, $p$ ein Primteiler von $\#G$, $k \in \N$ sodass $p^{k} | \#G$, setze
  \[n_{k}:= \#\{H \le G \mid \#H = p^{k}\}\]
  Dann gilt:
  \[n_{k} \equiv 1 \mod p\]
  Insbesondere ist $n_{k} \ne 0$, d.h. $\exists H \le G$ mit $\#H = p^{k}$.
\end{satz}
\begin{ubng*}[Vorbereitung]
  Sei $p$ eine Primzahl, $k \in \Nn, m \in \N$, dann:
  \[\binom {mp^{k}}{p^{k}} = m \cdot u\]
  wobei $\N \ni u \equiv 1 \mod p.$
\end{ubng*}
\begin{proof}[Beweis](zu 30)
  Durch Analyse der Wirkung von $G$ auf $X:= \{S \subseteq G \mid \#S=p^{k}\}$
  gegeben durch
  \[\lm: G \times X \to X, (g, S) \mps g \cdot S = \{g \cdot s \mid s \in S\}\]
  (beachte: $\ell_{g} : h \mps g \cdot h$ ist bijektiv $\imp \#gS = \#S = p^{k}$ d.h. $g \cd S  \in X$)
  Setze $m:= \# \fak G{p^{k}}$, für $S \in X$ definiere
  \[G_{S} := \mathrm{Stab}_{G}(S) = \{g \in G \mid gS = S\}\]
\begin{enumerate}
  \item $\forall S \in X : \#G_{S} | p^{k}$:

        Beachte: $G_{S}$ wirkt auf $S$ (da $gS = S \forall g \in G_{S}$) durch Linkstranslation:\[G_{S} \tm S \to S, (g, s) \mps g \cd s\]
        Scchreibe $S$ als disjunkte Vereinigung seiner $G_{S}$-Bahnen.
        \[S = \bigsqcup_{i \in \eb \ell} G_{S}h_{i}\]
        wobei $h_{1}, ..., h_{\ell}$ ein Repräsentantensystem der Bahnen ist.

        Beachte: $r_{h_{i}}: g \mps gh_{i}$ ist bijektiv.
        Also folgt $\#G_{S}h_{i} = \#G_{S}$
        \[\imp p^{k} = \#S = \sum_{i=1}^{\ell} \#G_{S}h_{i} = \sum_{i=1}^{\ell} \#G_{S} = \ell \#G_{S}\]
        d.h. $\#G_{S} | p^{k}$.
  \item Sei $X_{0}:= \{S \in X \mid \#G_{S} = p^{k}\}$ und $X_{1}:= X \setminus X_{0}$

        Behauptung: $\#X_{0} = m \cdot n_{k}$
        \begin{enumerate}[(a)]
          \item Sei $H \le G$ eine Untergruppe mit $\#H = p^{k}$, dann:
                \[\{S \in X_{0} \mid G_{S} = H\} = \{Hg \mid g \in G\}\]
                Denn:
                \begin{itemize}
                  \item ``$\subseteq$'': Gelte $G_{S} = H$, d.h. $H \cd S = S \imp H\cd s \subseteq S, \forall s \in S$.

                        Aber: $\#H\cd s \underset{r_{s}\text{ist bij.}}= \#H = p^{k} = \#S \imp H\cd s = S \imp s$ (ist das gesuchte $g$)
                  \item ``$\supseteq$'': Zu zeigen: $\stb_{G}(H \cd s) = H$. Sei $g \in G$.
                        \[g \in \stb_{G}(Hs) \iff g Hs = Hs \underset{r_{s}\text{ist bij.}} \iff gH = H \underset{H\le G} \iff g \in H\]
                \end{itemize}
          \item \[X_{0} = \bigsqcup_{H \le G, \#H = p^{k}}\{S \in X \mid G_{S} = H\} \overset{(a)} = \bigsqcup_{H \le G, \#H = p^{k}}\{Hg \mid g \in G\}\]
                \[\#X_{0} = \sum_{H \le G, \#H = p^{k}}\#\ubr{\{Hg \mid g \in G\}}_{= \mfaktor HG} \overset{\text{Lagrange}} = \frac{\#G}{\#H} = \frac{\#G}{p^{k}} =m\]
                \[ = m\l(\sum_{H \le G, \#H = p^{k}} 1\r) = m \cdot n_{k}\]
        \end{enumerate}
  \item $pm | \#X_{1}$
        \begin{enumerate}[(a)]
          \item $G$ wirkt auf $X_{1}$ (durch $(g, S) \mps gS$)

                d.h. gilt $S \in X_{1}$ und $g \in G$, so auch $gS \in X_{1}$. Es genügt also zu zeigen: $\#G_{gS} = \#G_{S}$

                Dazu: \[G_{gS} = \stb_{G}(gS) = g \stb_{G}(S)g^{-1} = gG_{S}g^{-1} \overunderset{\text{Konj. mit }g}{\text{ist Gruppenisom.}} \cong G_{S}. \]
          \item Betrachte nun $G$-Bahn durch $S \in X_{1}$, Behauptung: $\#G\cdot S$ ist Vielfaches von $p\cdot m$

                Dazu: Bahngleichung:
                \[\#G\cdot S = \fak{\#G}{\#G_{S}} = \fak{mp^{k}}{\#G_{S}}\]
                da $\#G_{S}$ echter Teiler von $p^{k}$, also Teiler von $p^{k-1} \imp \#GS$ ist Vielfaches von $\fak{mp^{k}}{p^{k-1}} = mp$
                \[(m \cd \fak{2^{5}}{2^{4}} = m \cd 2, \quad m \cd \fak{2^{5}}{2^{2}} = m \cd 2^{3},)\]
          \item Schreibe nun $X_{1}$ als disjunkte Vereinigung seiner Bahnen
                \[X_{1} = \bigsqcup_{j \in I}G \cdot \ubr{S_{j}}_{\text{Bahnrepr.}}\]
                und $\#G \cdot S_{j} = m \cd p \cd a_{j}, a_{j} \in \N$
                \[\imp \#X_{1} = \sum_{j \in J}\#G \cd S_{j} = m \cd p \cd \ubr{\sum_{j \in J}a_{j}}_{=: N \in \N}\]
        \end{enumerate}
  \item $\#X = \#X_{0} + \#X_{1} = m \cd n_{k} + m \cd p \cd N = m(n_{k}+pN)$

        gleichzeitig: \[\#X = \#\{S \subseteq G \mid \#S = p^{k}\} \underset{\#G = m \cd p^{k}} = \binom{m\cd p^{k}}{p^{k}} \underset{\text{Übung}} = m \cd u\]
        für ein $u \in \N : u \equiv 1 \mod p$.
        \[\imp m(n_{k} + pN) = n \cd u \imp n_{k}+pN = u \underset{\mod p} n_{k} \equiv u \equiv 1 \mod p. \qedhere\]
\end{enumerate}
\end{proof}

\addcontentsline{toc}{subsection}{Satz von Cauchy}
\begin{kor}[\textbf{Satz von Cauchy}]
  Sei $G$ eine endliche Gruppe mit $p | \#G$ für $p$ eine Primzahl, dann $\ex g \in G : \ord(g) = p$
  \begin{proof}[Beweis]
    Nach Sylow I $\ex H \le G : \#H = p$, sei $g \in H \setminus \{e\}$. Dann gilt $\ord(g) = p$.

    ($\ord(g) \ne 1$ und $\ord(g) | \#G=p$).
  \end{proof}
\end{kor}

\addcontentsline{toc}{subsection}{$p$-Sylow Gruppe}
\begin{defi}[\textbf{$p$-Sylow Gruppe}]
  Sei $G$ endlich, gelte $\#G = p^{f} \cd m$ für $m, f \in \N$ sodass $p \not | m$. Eine Untergruppe $H \le G$ mit $\#H = p^{f}$ heißt $p$-Sylow (Unter-)Gruppe von $G$, schreiben \[\mathrm{Syl}_{p}(G) = \{H \le G \mid H \text{ ist }p-\text{Sylow}\}\]
  \[\mathrm{syl}_p(G) = \#\mathrm{Syl}_{p}(G)\]
\end{defi}

\addcontentsline{toc}{subsection}{Normalisator}
\begin{defi}[\textbf{Normalisator}]
  Der Normalisator einer Untergruppe $H \le G$ ist \[N_{G}(H):=\{g \in G \mid gHg^{-1} = H\}\]
  ($c_{g}$ ist Automorphismus $\imp \#gHg^{-1} = \#H, \forall g \in G$)
\end{defi}


\begin{int*}
Sei $X:= \{H \mid H \le G\}$, $X$ ist eine $G$-Menge durch Konjugation $c: G \times X \to X, (g, H) \mps gHg^{-1}$
\end{int*}


\begin{prop}[Übung]
\begin{enumerate}[(a)]
  \item $N_{G}(H) \underset{\text{für }H \le G} = \stb_{G}(H)$

        (Insbesondere ist $N_{G}(H) \le G$ eine Untergruppe.)
  \item Es gelten: $H \nt N_{G}(H)$ und $N_{G}(H)$ ist die größte Untergruppe von $G$, sodass $H$ ein Normalteiler in dieser ist.
\end{enumerate}
\end{prop}


\begin{lemm}
  Sei $H \le G$ eine $p$-Gruppe, $P \in \mathrm{Syl}_{p}(G)$ ($p$ eine Primzahl), dann:
  \begin{enumerate}[(a)]
    \item Gilt $P \le H$, so folgt $P = H$.
    \item Ist $H \le N_{G}(P)$, so gilt $H \le P$.
    \item Gilt $H \nsubseteq P$ , so folgt $\stb_{H}(P) < H$ (ist echte Untergruppe)
  \end{enumerate}
\begin{proof}[Beweis]
  \begin{enumerate}[(a)]
          \item Schreibe $\#G = p^{f}\cd m$, so dass $p \not | m$ ($m, f \in \N$), $P$ $p$-Sylow Untergruppe $\imp \#P = p^{f}$.

          $H$ eine $p$-Gruppe in $G \underset{\text{Lagrange}}\imp \#H | p^{f} \cd m$. also $\#H | p^{f}$

          Nun: $P \subseteq H$ und $p^{f} = \#P \ge \#H \imp P = H$ (und $\#H = p^{f}$)
    \item Sei $G' = N_{G}(P)$. Aus Proposition
          \[\imp P \nt G' \overunderset{\text{Nach}}{\text{Voraussetzung}}\imp H \le G' \overunderset{\text{Erster}}{\text{Isomorphiesatz}} \imp P \nt P \cd H\] und
          \[\fak {(P \cd H)}{P} \cong \fak H{P \cap H}\]
          Ordnung ist $p$-Potenz, evtl $p^{f}$
          \[\underset{\text{Lagrange}}\imp \#P \cd H = \ubr{\#P}_{p\text{-Potenz}} \cd \ubr{\#\fak{P \cd H}{P}}_{p\text{-Potenz}}\]
          Also ist $P \cd H$ eine $p$-Gruppe mit $P \subseteq PH$
          \[\underset{(a)} \imp PH = P \underset{eH \subseteq P}\imp H \subseteq P\]
    \item Gelte $H \nsubseteq P$. zu zeigen: $\stb_{H}(P) < H$

          Angenommen: $H = \stb_{H}(P) = \ubr{\{h \in H \mid hPh^{-1} = P\}}_{=H \cap \stb_{G}(P)} = H \cap N_{G}(P)$
          Dann folgt
          \[H \subseteq N_{G}(P) \underset{(b)} \imp H \subseteq G. \qedhere\]
  \end{enumerate}
\end{proof}
\end{lemm}

\addcontentsline{toc}{subsection}{Sylow II}
\begin{satz}[\textbf{Sylow II}]
  Sei $G$ endlich, $p$ ein Primteiler von $\#G$. Dnan:
  \begin{enumerate}[(a)]
    \item Je 2 $p$-Sylow Gruppen von $G$ sind kunjugiert.
    \item Jede $p$-Gruppe $H$ mit $H \le G$ liegt in einer $p$-Sylow Gruppe von $G$.
    \item $\forall P \in \Syl_{p}(G): \syl_{p}(G) = [G : N_{G}(P)]$ und insbesondere $(P \le N_{G}(P))$ gilt $\syl_{p}(G) | [G : P]$
  \end{enumerate}
  \begin{proof}[Beweis]
    \begin{enumerate}[(a)]
      \item $X:= \Syl_{p}(G)$ ist $G$-Menge via Konjugation ($P \in \Syl_{p}(G)$ und $g \in G \imp \#gPg^{-1} = \#P \imp gPg^{-1} \in \Syl_{p}(G)$)

      Zu zeigen: $G$ wirkt transitiv auf $X$.

      Annahme: $X$ besteht aus $t \ge 2$ Bahnen, also \[X = \bigsqcup_{i \in \eb t} G \circ P_{i}\]
      für geeignete Repräsentantensystem $P_{1}, ..., P_{t} \in \Syl_{p}(G)$ ($g \circ P = gPg^{-1}$)

      Behauptung: $p | \#G \circ P_{i}, \forall i \in \eb t$.

      Dazu: Wähle $j \ne i$ betrachte die $P_{j}$-Wirkung auf $G \circ P_{i}$. Schreibe wieder $G \circ P_{i}$ als disjunkte Vereinigung von $P_{j}$-Bahnen:
      \[G \circ P_{i} = P_{j} \circ Q_{1} \sqcup \cdots \sqcup P_{j}\circ Q_{s} \quad (*)\]
      ($s \in \N$ geeignet, $Q_{\ell} \in \Syl_{p}(G)$ geeignet)

      Bahngleichung:
      \[\#P_{j}\circ Q_{\ell} = \fak{\#P_{j}}{\#\stb_{P_{j}}(Q_{\ell})}\]
      beachte: $P_{j} \notin G \circ P_{i}$, d.h. $P_{j} \ne Q_{\ell}$
      \[\underset{35(c)} \imp \stb_{P_{j}}(Q_{\ell}) < P_{j} \imp \#P_{j} \circ Q_{\ell} \ne 1 \text{ und teilt } \#P_{j} \imp p | \#P_{j}\circ Q_{\ell}\]
      $\imp p$ alle Bahnlängen in $(*)$ von $G \circ P$ als $P_{j}$-Menge $\imp p | \#G \circ P_{i}, \forall i \imp p | \#\Syl_{p}(G)$
      \[\imp \Syl_{p}(G) = \bigsqcup_{i \in \eb t}G \circ P_{i}\]
      Widerspruch zu (0): $\syl_{p}(G) \equiv 1 \mod p$.
      \item Annahme: $H \le G$ eine $p$-Gruppe liegt in keiner $p$-Sylow. Betrachte Konjugationswirkung von $H$ auf $X = \Syl_{p}(G)$. Schreibe
            \[X= H \circ R_{1} \sqcup \cdots \sqcup H \circ R_{w}\]
            ($w \in \N$) die $R_{i}$ sind Repräsentanten der Bahnen. Beachte $H \nsubseteq R_{i}$ ($i \in \eb w$). Wie in $(a)$ gilt $\stb_{H}(R_{i}) < H$ also, dass $p | \#H \circ R_{i}, \forall i \imp p | \#X$ Widerspruch zu (0).
      \item Bahngleichung für $P \in \Syl_{p}(G)$
            (Verwenden (a), d.h. $G \circ P = \Syl_{p}(G)$)
            \[\syl_{p}(G)=\#\Syl_{p}(G) = \fak {\#G}{\#\stb_{G}(P)} : \fak {\#G}{\#N_{G}(P)} = [G : N_{G}(P)]\]
            ($\syl_{p}(G)$ teilt $[G: P]$ schon oben eingesehen, da $P \le N_{G}(P)$)
    \end{enumerate}

\end{proof}
\end{satz}

\begin{kor}
  Sei $G$ endlich und $p$ ein Primteiler von $\#G$, dann $\syl_{p}(G) = 1 \iff$ jede $p$-Sylow ist ein Nullteiler in $G$.
  \begin{proof}[Beweis]
    Für $P \in \Syl_{p}(G)$ gilt:
    \[P \nt G \iff N_{G}(P) = G \underset{36(c)}\iff \syl_{p}(G) = [G : N_{p}(G)] = 1.\qedhere\]
  \end{proof}
\end{kor}

% \begin{satz} % TODO Supposed to be 36
%   Sei $G$ eine endliche Gruppe, $p$ Primteiler von $\#$, dann:
%   \begin{enumerate}[(a)]
%     \item $Syl_p(G) = 1 \mod p$
%     \item $G$ wirkt transitiv auf $Syl_{p}(G)$
%     \item $\forall H \le G$ mit $H$ eine $p$-Gruppe $\exists P \in Syl_{p}(G) : H \le G$
%     \item $Syl_{p}(G) = [G : N_{G}(\cdot)] \#P \in Syl_{p}(G)$
%           Insbesondere $Syl_{p}(G)$ teilt $[G: P] = \fak{\#G}{\#P}$
%   \end{enumerate}
% \end{satz}

\begin{kor}% KOROLLAR 38
  Sei $G$ endlich, seien $p_{1}, ..., p_{t}$ die paarweise verschiedenen Primteiler von $\#G$. Sei $P_{i} \in \Syl_{p_{i}}(G)$. Dann gilt:
  sind $P_{1}, ..., P_{t}$ Normalteiler von $G$, so folgt: die Abbildung $P_{1} \times \cdots \times P_{t} \to G, (g_{1}, ..., g_{t}) \mapsto g_{1}\cdot ... \cdot g_{t}$ ist ein Gruppenisomorphismus.
\end{kor}
\begin{proof}[Beweis]
  $P_{i} \nt G$ für $i \in \eb t$

  und $\ggt(\#P_{i}, \#P_{j}) \underset{i \ne j} = 1$ ($p_{i}, p_{j}$ versch. Primzahlen)
  und $\prod_{i=1}^{t}\#P_{i} = \#G$
$\underset{\text{Kor. 1.55}}\imp $ die angegebene Abbildung ist ein Gruppenisomorphismus.
\end{proof}

\begin{bsp*}
  Ist $G$ abelsch, so sind alle Untergruppen Normalteiler.
\end{bsp*}

\begin{kor} %39
  $G$ endlich abelsch und $p_{i}$ und $P_{i}$ wie in Korollar 38. So gilt: $\bigtimes_{i=1}^{t}P_{i} \xrightarrow{\text{wie in Kor. 38}} G$ ist Gruppenisomorphismus. ($P_{i}$ sind abelsche $p_{i}$-Gruppen).
\end{kor}

\begin{satz} %40
Sei $G$ eine endliche abelsche $p$-Gruppe, dann $\ex ! t \in \N, \ex! e_{1} \ge e_{2} \ge \cdots \ge e_{t} \in \N$, sodass
\[G \cong \bigtimes_{i=1}^{t}\fak{\Z}{p^{e_{i}}}\]
\end{satz}

\begin{bsp*}
$G$ abelsch mit $\ord(G) = 105 \imp G \cong \fak \Z {3\Z} \times \fak \Z {5\Z} \times \fak \Z {7\Z}$
\end{bsp*}

\begin{whg*}
$G$ heißt einfach $\iff $ einzige Nullteiler von $G$ sind $\{e\}$ und $G$.
\end{whg*}


\begin{lemm}[Übung] %41
  sei $G$ endlich, $\# G = p^{f}\cd  m$ mit $f, m \in \N, p$ Primzahl und $p \not | m$. Dann:
  $p^{f} \not |(m-1)! \imp G$ ist nicht einfach.
\begin{proof}[Beweis]
  Idee: Sei $P \in Syl_{p}(G)$, betrachte $G$-Wirkung auf $\fak GP$ durch Linkstranslation, d.h.
  \[\rho: G \to \Bij(\fak GP), g \mapsto \ell g\]
  Trick: $\ker(\rho)$ ist der gesuchte Normalteiler.
\end{proof}
\end{lemm}

\begin{satz}
Ist $G$ einfache Gruppe mit $\#G < 60$, so gilt $G \cong \fak \Z p$ für $p$ eine Primzahl.
\end{satz}
\begin{proof}[Beweis]
  Sei $G$ einfach mit $\#G < 60$. o.E. $\#G$ keine Primzahl, sonst fertig.
  o.E. $G$ ist keine $p$-Gruppe für Primzahl $p$. (sonst: $Z(G) \supsetneq \{e\} \overunderset{G \text{ einfach}}{Z(G) \nt G} \imp G = Z(G)$, d.h. $G$ abelsch. $\underset{G \text{ einfach}} \imp G \cong \fak \Z p$)

  Fall $\# = p^{f}m$ mit $p^{f} \not | (n-1)! \imp G$ nicht einfach (Lemma 41)

  (Übung)Es bleiben $\#G \in \{\ubr{30}_{2 \cdot 3 \cdot 5}, \ubr{40}_{2^{3} \cdot 5}, \underbrace{56}_{2^{3}\cdot 7}\}$

  Fall 1: $\#G = 2^{3} \cdot 5$, dann:
  $Syl_{5}(G) \cong 1(5)$ (Sylow I)

  $Syl_{5}(G)$ teilt $\fak {\#G}5 = 8$ (Sylow II)

  Teiler von $8: 1, 2, 4, 8$
  Kongruenz erzwingt $Syl_{5}(G) = 1 \underset{37} \imp$ die einzige $5$-Sylow Untergruppe von $G$ ist ein Normalteiler (Widerspruch zu $G$ einfach)

  Fall 2: $\# G = 2^{3} \cdot 7$, dann (Shritte wie im Fall 1 für $p=7$)
  \[Syl_{7}(G) \in \{1, 8\}\]
  (teilt $8$, $\cong 1 \mod 7$)

  Fall: Es gibt $8$ $7$-Sylow Untergruppen, isomorph zu $\fak \Z 7$

  Beachte: 2 $7$-Sylow's schneiden sich nur in $\{e\}$ (sonst sinid sie gleich, Elemente $\ne e$ sind Erzeuer)

  $\imp$ es gibt $8\cdot 6$ Elemente in $G$ der Ordnung $7$

  $\imp$ Es gibt $56-48 = 8$ Elemente in $G$ der Ordnung $\ne 7$

  Aber: Es gibt (mindestens) eine 2-Sylow Untergruppe von $G$ und die hat Ordnung $8 = 2^{3}$.

  Es folgt: Die 8 obigen Elemente bilden die einzig mögliche 2-Sylow Untergruppe von $G$.

  $\imp Syl_{2}(G) = 1 \imp$ die 2-Sylow ist ein nicht triviale Normalteiler von $G$.


  Fall 3 (Übung)
\end{proof}
\begin{bem*} Die Zahl 60 ist optimal, denn $A_{5}$ ist einfach, nicht zyklisch (von Primzahlordnung) und hat 60 Elemente.
\end{bem*}

\section{Auflösbare Gruppen} %TODO FROM HERE
\begin{defi}
  \begin{enumerate}[(a)]
    \item Eine aufsteigende Folge von Untergruppen $G_{0} < G_{1} < G_{2} < \cdots < G_{t} = G$ von $G$ heißt Normalreihe, wenn
          $\forall i \in \eb t : G_{i-1} \nt G_{i}$ ist Normalteiler.

          Schreibe auch $(G_{i})_{i=0}^{t}$ oder $\&$ für die Folge.
    \item die Faktorgruppe $(\fak{G_{i}}{G_{i-1}})_{i=1}^{t}$ heißen Faktoren der Normalreihe.
    \item Eine Normalreihe $\G$ heißt Zerlegungsreihe $\iff$ alle Faktoren sind einfach.
    \item $X$ heißt abelsch $\iff$ alle Faktoren sind abelsch.
    \item $G$ heißt auflösbar $\iff G$ besitzt eine abelsche Normalreihe.
    \item Ist $\G': G_{0}' < G_{1}' < \cdots < G'_{t'} = G$ eine weitere Normalreihe, so heißt $\G'$ \textbf{echt} feiner als $\G \iff$
          \[\{G_{i} \mid i \in \{0, ..., t\}\} \subsetneq \{G_{j}' \mid j \in \{0, ..., t'\}\}\]
  \end{enumerate}
\end{defi}
\begin{bsp*} %TODO Did not write this weew

\end{bsp*}


\begin{prop} %44
  Sei $G " \{e\} = G_{0} \nt G_{1} \nt \cdots G_{t} = G$ eine Normalereihe, dann:
  \begin{enumerate}[(a)]
    \item $G$ ist eine Zerlegungsreihe $iff G$ besitzt keine echte Verfeinerung.
    \item Es gilt $2^{t} \le \#G$
    \item Ist $G$ endlich, so besitzt $G$ eine Verfeinerung, die eine Zerlegungsreihe ist.
    \item Ist $G$ abelsch, so ist auch Verfeinerung abelsch.
  \end{enumerate}
\begin{proof}[Beweis]
  \begin{enumerate}[(a)]
    \item $G$ ist keine Zerlegungsreihe $\iff \ex i \in \eb t : \fak {G_{i}}{G_{i-1}}$ nicht einfach $\iff \ex i \in \eb t : \bar H \nt \fak {G_{i}}{G_{i-1}}$ ein Normalteiler mit $\bar H \ne \{e\}, \bar H \subsetneq \fak {G_{i}}{G_{i-1}}$
          $\overset{\text{2. Isometriesatz}} \iff \{e\} i \in \eb t : \ex H \triangleleft G_{i} $ ein Normalteiler mit $G_{i-1} \triangleleft H$

          $\iff \ex i \in \eb t : G$ kann zwischen $G_{i-1}$ und $G_{i}$ echt verfeinert werden $\iff G$ besitzt eine echte Verfeinerung.
    \item Lagrange: (Für $H \le G : \#G = \#H \cdot \# \fak GH$)

          \[G = G_{t} = \#G_{t-1} \cdot \#\fak{G_{t}}{G_{t-1}} = \#G_{t-2} \cdot \# \fak {G_{t-1}}{G_{t-2}}\cdot \#\fak{G_{t}}{G_{t-1}}\]
          \[\vdots\]
          \[\prod_{i=1}^{t}\#\fak{G_{i}}{G_{i-1}} \ge 2^{t}\]
          $\imp t \le \log_{2} \#G$
    \item Sei $G'$ eine Verfeinerung von $G$, maximaler Länge $t'$. Das gibt es, da $t' \le \log_{2} \#G$ dieses $G'$ kann nicht echt verweinert werden ($t'$ maximal!)

          $\imp G'$ ist Zerlegungsreihe, die $G$ verfeinert.
    \item Sei $G$ abelsch und $G'$ eine Verfeinerung von $G$, z.z. $G'$ ist abelsch.
          ($G' :\quad G_{0}' = \{e\} \triangleleft G_{1}' \triangleleft \cdots \triangleleft G_{t'}' = G$)
          Sei $j \in \eb {t'}$, z.z. $\fak{G_{j}'}{G_{j-1}'}$ abelsch. Finde zu $j, j-1$ ein $i \in \eb t$, sodass
          \[\begin{matrix}
              G & \cdots & G_{i-1} & \triangleleft & G_{i} & \cdots \\
                & & = \\
                & & G_{\ell}' & \le G_{j-1}' \triangleleft G_{j'} \\
%TODO its a mess
            \end{matrix}
          \]
  \end{enumerate}
\end{proof}
\end{prop}

\addcontentsline{toc}{subsection}{Jordan-Hölder Satz}
\begin{satz}[\textbf{Jordan-Hölder}]
Ist $G$ endlich, so ist die Folge der Faktoren einer Zerlegungsreihe $G(srqa)$ von $G$ bis auf Reihenfolge der Faktoren unabhängig von der Wahl der Zerlegungsreihe von $G$.
\end{satz}
\begin{proof}[Beweis] %TODO See it
  \item Jantzen Schwermer Satz II. 2.4
  \item Jacobson §4.6
\end{proof}
\begin{kor} %46
$G$ endlich, dann $G$ auflösbar $\iff$ die Faktoren jeder Zerlegungsreihe sind (abelsch und ) von Primzahlordnung.
\end{kor}
\begin{proof}[Beweis]
  \item ``$\imp$'': Sei $\G$ eine abelsche Normalreihe
  \(\underset{44} \imp \ex \) Zerlegungsreihe $G'$ die $G$ verfeinert, diese ist dann (stets) wiederr abelsch.

  Ihre Faktoren sind einfach und ablesch (und endliche Gruppen), also zyklisch von Primzahlordnung.

  Wende nun Jordan-Hölder an.

\item ``$\impliedby$'': Hat man $G$ wie angegeben (zu $G$), dann ist $\G$ abelsch, also $G$ auflösbar.
\end{proof}
\end{document}
