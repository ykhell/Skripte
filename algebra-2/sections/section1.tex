\documentclass[a4paper]{report}
\usepackage{../template}
\begin{document}

% Lecture 1
\section{Modules}
Let $(R, 0, 1, +, \cd)$ or simply $R$ be a ring.
\begin{defi}
\begin{enumerate}[(a)]
        \item A left $R$-module $(M, 0, +, \cd)$ or simply $M$ is an abelian group $(M, 0, +)$, together with an operation $\cd: R \tm M \to M, (r,m) \mps r \cd m = rm$, such that for all $a, b \in R, m, n \in M$
        \begin{enumerate}[(M1)]
          \item $a(m+n) = am+an$ and $(a+b)m = am + bm$
          \item $a(b \cd m) = (ab) \cd m$
                \item $1 \cd m = m$
        \end{enumerate}
  \item Let $M, N$ be left $R$-modules. A map $\ph : M \to N$ is called $R$-linear or a left $R$-module homomorphism $:\iff \ph: (M, 0, +) \to (N,0,+)$ is a group homomorphism, and $\forall a \in R, m \in M : \ph(am) = a\ph(m)$. Define $\Hom_{R}(M,N) = \{\ph: M \to N \mid \ph$ is $R$-linear$\}$.
\end{enumerate}
\end{defi}

\begin{facts}[Excers.]$\forall x \in M, a \in R: 0_{R} \cd x = 0_{M}, a \cd 0_{M} = 0_{M}, (-1)\cd x = -x$
\end{facts}
\begin{rem}[Excers.]
\begin{enumerate}[(a)]
  \item $\Hom_{R}(M,N)$ is an abelian group with $0 =$ the map $M \to \{0_{N}\}$ and $\ph + \psi: M \to N, m \mps \ph(m)+\psi(m)$.
  \item If $R$ is commutative, then $\Hom_{R}(M,N)$ is an $R$-module via \[r \cd \ph : M \to N, m \mps r \cd \ph(m)\]
  \item If an abelian group $(M,0,+)$ carries an operation $\cd: M \tm R \to M, (m,r) \mps m \cd r$ such that:
        \begin{enumerate}[(M1')]
          \item $(m+n) \cd a = m \cd a + n \cd a, m \cd (a+b) = ma + mb$
          \item $(m \cd a)\cd b = m \cd (ab)$
                \item $m \cd 1 = m$
        \end{enumerate}
then $(M,0,+,\cd)$ is called a right $R$-module. Analogously we can define right $R$-module homomorphisms.
\end{enumerate}
\end{rem}
\begin{conv} We shall use the term $R$-module for left $R$-module, since we will mainly work with these. In fact right $R$-modules are left $R^{\op}$-modules.
\end{conv}
\begin{defi}
The opposite ring (Gegenring) of $(R,0,1,+,\cd)$ is $R\op = (R,0,1,+,\cd\op)$ with $a \cd\op b = b \cd a$
\end{defi}
\begin{facts}[Excersize]
  \begin{enumerate}[(a)]
    \item $R\op$ is a ring
    \item $\id_{R}: R \to R$ is a ring homomorphism $\iff R$ is commutative.
    \item $\id_{R}: R \to (R\op)\op$ is an isomorphism.
    %\item Let $(M, 0, +)$ be an abelian group, with an operation $\cd: R \tm M \to M, (r,m) \mps m \cd\op r := r \cd m$, then $(M, 0, +, \cd)$ is a left $R$-module $\iff (M, 0, +, \cd\op)$ is a right $R\op$-module.
  \end{enumerate}
In particular: If $R$ is commutative, then left $R$-modules are right $R$-modules.
\end{facts}
\begin{rem}[Excersize]
  Let $(M, 0, +)$ be an abelian group.
  \begin{enumerate}[(a)]
    \item The abelian group $\End_{\Z}(M) = \Hom_{\Z}(M,M)$ is a ring with composition as multiplication.
    \item There is a bijection $\{$operations $*: R \tm M \to M \mid (M,0,+,*)$ is an $R$-module$\} \leftrightarrow \{$ring homomorphisms $\ph: R \to \End_{\Z}(M)\}$ via \[* \mps \ph_{*}: R \to \End_{\Z}(M), r \mps (\ph_{*}(r): m \mps r \cd m)\]figure out an inverse.
    \item If $M$ is an $R$-module, then $\End_{R}(M) \subseteq \End_{\Z}(M)$ is a subring
    \item The map $R\op \to \End_{R}(R), r \mps \rho_{r}: a \mps a \cd r$ is a ring isomorphism. The inverse is $\End_{R}(R) \to R\op, \ph \mps \ph(1)$
  \end{enumerate}
\end{rem}

\begin{exmp}
\begin{enumerate}[(a)]
        \item Let $K$ be a field, $K$-modules are $K$-vector spaces and vice versa.
  \item If $(M, 0, +)$ is an abelian group, it is in a unique way a $\Z$-module.
  \item Let $K$ be a field, $R = M_{n \tm n}(K), n > 1, V_{n}(K) =$ column $Z_{n}(K)$ row vectors of length $n$ over $K$, then:
        \begin{itemize}
          \item $V_{n}(K)$ is a left $R$-module.
                \item $Z_{n}(K)$ is a right $R$-module.
        \end{itemize}
  \item $R$ is a left $R$-module and right $R$ module with multiplication.
  \item If $M_{1}$ and $M_{2}$ are $R$-modules, we can define on $M_{1} \tm M_{2}$ a $R$-module structure via
        \[r \cd (m_{1}, m_{2}) := (rm_{1}, rm_{2})\] (group structure from Algebra 1)
        \item $\Hom_{R}(R, M) \to M, \ph: \ph(1)$ is an isomorphism of abelian groups, and if $R$ is commutative, then also an isomorphism of $R$-modules.
\end{enumerate}
\end{exmp}

\begin{defi}
An $R$-linear map $\ph: M \to M'$ is called a monomorphism/epimorphism/isomorphism $\iff \ph$ is injective/surjective/bijective respectively. We say $R$-modules $M, M'$ are isomorphic if there exists an isomorphism $M \to M'$.
\end{defi}
\begin{rem*}
$\ph$ is an $R$-linear isomorphism $\iff \ph^{-1}$ is an $R$-linear isomorphism.
\end{rem*}

\begin{defi}
\begin{enumerate}[(a)]
  \item Let $M$ be an $R$-module. A subset $N \subseteq M$ is an $R$-submodule if it is a subgroup and $\forall a \in R, n \in N : a \cd n \in N$ (i.e. $R \cd N \subseteq N$)
  \item An $R$-submodule $I \subseteq R$ is called a left ideal.
        \item $I \subseteq R$ is called a two sided ideal iff it is a left ideal and $I \cd R \subseteq I$
\end{enumerate}
\end{defi}

\begin{exmp}
\begin{enumerate}[(a)]
  \item If $N' \subseteq N$ and $M' \subseteq M$ are $R$-submodules of $R$-modules $M$ and $N$ and if $\ph: M \to N$ is an $R$-linear map, then: \[\ph(M') \subseteq N \text{ and } \ph^{-1}(N') \subseteq M\]
        are $R$-submodules. In particular $\ker(\ph) \le M$ and $im(\ph) \le N$ are submodules.
  \item If $(M_{i})_{i \in I}$ is a family of submodules of $M$, then $\bigcap_{i \in I}M_{i} \subseteq M$ is the largest submodule of $M$ contained in all $M_{i}$, and \[\sum_{i \in I} M_{i} = \{\sum_{i \in I}m_{i} \mid m_{i \in M)i}, \#\{i \mid m_{i \ne 0}\} < \infty\}\]
        is the smallest submodule of $M$ containing all $M_{i}$.
        \item 2-sided ideals of $M_{n \tm n}(R)$ are of the form $M_{n \tm n}(I)$ for $I \subseteq R$ a 2-sided ideal.
\end{enumerate}
\end{exmp}

\subsection*{Quotient Modules}
\begin{defi}
  Let $N \subseteq N$ be a submodule. From linear algebra $(\fak MN, \bar 0, \bar +)$ is an abelian group. ($\bar m = m + N$ are the equivalence classes and $\bar m + \bar m' = \bar{m+m'}$). This is an $R$-module (exercise) via \[\bar \cd : R \tm \fak MN \to \fak MN: (r, m+N) \mps rm + N\]
  We call $\fak MN$ (with $\bar 0, \bar +, \bar \cd$) the quotient module of $M$ by $N$, and we write \[\pi_{N \subseteq M}: M \twoheadrightarrow \fak MN, m \mps m+N\]
\end{defi}
\begin{defi}
  If $I \subseteq R$ is a 2-sided ideal of $R$, then
  \begin{enumerate}[(a)]
    \item $I \cd M := \{\sum_{i \in I} a_{i} \cd m_{i} \mid I$ finite, $a_{i \in I, m_{i} \in M}\}$ is an $R$-submodule of $M$ ($M$ an $R$-module)
          \item $(\fak RI, \bar 0, \bar 1, \bar +, \bar \cd)$ is a ring, and $\fak M{I \cd M}$ is an $\fak RI$-module.
  \end{enumerate}
\end{defi}
The following 3 results are proved as for groups:
\begin{thm}[Homomorphism theorem]
  Let $\ph: M \to M'$ be an $R$-linear map, then
  \begin{enumerate}[(a)]
    \item $\forall$ submodules $N \subseteq \ker(\ph): \ex ! R$-linear map $\bar \ph : \fak MN \to M', m+N \mps \ph(m)$ such that $\ph = \bar \ph \circ \pi_{N \subseteq M}$
          \item For $N = \ker(\ph)$, the map $\bar \ph : \fak M{\ker(\ph)} \to im(\ph)$ is an $R$-module isomorphism.
  \end{enumerate}
\end{thm}
\begin{thm}(First isomorphism theorem)
  Let $M$ be an $R$-module and $N_{1}, N_{2} \le M$ be $R$-submodules. Then the map \[\fak {N_{1}}{N_{1} \cap N_{2}} \to \fak {N_{1} + N_{2}}{N_{2}}, n_{1} + N_{1} \cap N_{2} \mps n_{1} + N_{2}\]
  is a well-defined $R$-linear isomorphism.
  \[\begin{tikzcd}
                        & N_1+N_2 \arrow[ld, no head]      &                         \\
N_1 \arrow[rd, no head] &                                  & N_2 \arrow[lu, no head] \\
                        & N_1 \cap N_2 \arrow[ru, no head] &
\end{tikzcd}\]
\end{thm}
\begin{thm}[Second isomorphism theorem]
  Let $M$ be an $R$-module and $N \le M$ an $R$-submodule. Then
  \begin{enumerate}[(a)]
    \item The following maps are bijective and mutually inverse to each other:
          \begin{align*}
            \{N' \subseteq M \text{ submodule} \mid N \subseteq N'\} &{\overunderset \ph \psi \rightleftarrows} \{\bar N \subseteq \fak MN \text{ submodule}\} \\
            \ph: N {\mps} \fak {N'}N \quad & \quad\ \pi_{N \subseteq M}^{-1}(\bar N) {\mpsfrom} \bar N : \psi
          \end{align*}
    \item For $N' \subseteq M$ a submodule with $N \subseteq N'$ we have the $R$-linear isomorphism:
          \[\fak{(M/N)}{(N'/N)} \to \fak M{N'}, \bar m + \fak {N'}M \mps m+N'\]
  \end{enumerate}
\end{thm}

\subsection*{Direct sums and products}
Let $(M_{i})_{i \in I}$ be a family of $R$-modules.
\begin{defi}
\begin{enumerate}[(a)]
  \item $\prod_{i \in I}M_{i} = \{(m_{i})_{i \in I} \mid m_{i} \in M_{i}, \forall i \in I\}$ is an $R$-module with component-wise operations:
        \[(m_{i})_{i \in I} + (n_{i})_{i \in I} =(m_{i}+n_{i})_{i \in I}\]
        \[r \cd (m_{i})_{i \in I} = (r \cd m_{i})_{i \in I}, \quad r \in R\]
        is called the (direct) product of $(M_{i})_{i \in I}$. One has the projection maps ($R$-module epimorphisms):\[\pi_{i_{0}}: \prod_{i \in I} M_{i} \to M_{i_{0}}, (m_{i}) \mps m_{i_{0}}\]
        \item $\bigoplus_{i \in I}M_{i} = \{(m_{i})_{i \in I} \in \prod_{i \in I} M_{i} \mid \{i \mid m_{i} \ne 0\} < \infty \}$ is an $R$-submodule of $\prod_{i \in I} M_{i}$. It is called the direct sum of $(M_{i})_{i \in I}$. One has $R$-module monomorphisms \[\iota_{i_{0}} : M_{i_{0}} \to \bigoplus_{i \in I} M_{i}, m_{i_{0}} \mps (\iota_{i_{0}}(m_{i_{0}}))\]
where the $i$-th component of $\iota_{i_{0}}(m_{i_{0}})$ is given by $\begin{cases} m_{i_{0}}, & i = i_{0}, \\ 0, & \text{otherwise}
\end{cases}$
\end{enumerate}
\end{defi}
\begin{thm}[Universal property of the direct product/sum]
\begin{enumerate}[(a)]
        \item $\forall R$-modules $M$, the map
        \[\Hom_{R}(M, \prod_{i \in I}M_{i}) \xrightarrow{\cong} \prod_{i \in I} \Hom_{R}(M, M_{i}), \ph \mps (\pi_{i} \circ \ph)_{i \in I}\]is well defined, bijective and a group isomorphism.
        \item $\forall R$-modules $M$, the map
        \[\Hom_{R}(\bigoplus_{i \in I}M_{i}, M) \xrightarrow{\cong} \prod_{i \in I} \Hom_{R}(M_{i}, M), \psi \mps (\psi \cd \iota_{i})_{i \in I}\]is well defined, bijective and a group isomorphism.
\end{enumerate}
\begin{proof}
  \begin{enumerate}[(a)]
    \item The inverse map is given by sending \[\underline \ph:= (\ph_{i}: M \to M_{i})_{i \in I} \in \prod_{i \in I}\Hom_{R}(M, M_{i})\]
          to \[\pi_{\underline \ph} : M \to \prod_{i \in I}M_{i}, m \mps (\ph_{i}(m))_{i \in I}\]
          now check: $\underline \ph \mps \pi_{\underline \ph}$ is inverse to the map in (a).
    \item The map is given by sending $\bar \ph = (\ph_{i} : M _{i} \to M)_{i \in I}$ to
          \[\coprod_{\bar \ph} : \bigoplus_{i \in I}: M_{i} \to M, (m_{i})_{i \in I} \mps \sum_{i \in I} \ph_{i}(m_{i})\]
          %LATER remark about finiteness
  \end{enumerate}
\end{proof}
\end{thm}

\begin{cor}[Important special case]
  Let $I$ be finite, then:
  \begin{enumerate}[(a)]
    \item $M := \prod_{i \in I} M_{i} \overset != \bigoplus_{i \in I}M_{i}$
    \item The maps $M_{i} \overunderset {\iota_{i}}{\pi_{i}}\rightleftarrows M$ satisfy \[\pi_{i} \circ \iota_{j} = \begin{cases} \id_{M_{i}},& i = j, \\ 0, &\text{otherwise}\end{cases}\quad \text{and}\quad \sum_{i \in I}\iota_{i} \circ \pi_{i} = \id_{M}\]
          \item If $M'$ is a module with maps $M_{i} \overunderset {\iota_{i}'}{\pi_{i}'}\rightleftarrows M'$ such that the formulas above hold, then $M \cong M'$
  \end{enumerate}
\end{cor}



% Lecture 2
\subsection{Generators and bases}
From now onwards let $R$ be a unitary ring and $M, M', N$ be $R$-modules.
\begin{nota*}
  \begin{itemize}
    \item For $I$ a set we write $M^{I}:= \prod_{i \in I}M$ and $M^{(I)}:= \bigoplus_{i \in I}M$ (where $M_{i} = M, \forall i \in I$).
          \item For $r \in \N$ we will write $M^{r}:= M^{\eb r}$, so if $I$ is finite then $M^{I} = M^{\#I} = M^{(I)}$
  \end{itemize}
\end{nota*}

\begin{defi} %20
  For $\underline m = (m_{i})_{i \in I} \in M^{(I)}$ we define a map $\ph_{\underline m}: R^{(I)} \to M, (r_{i}) \mps \sum_{i \in I}r_{i} \cd m_{i}$ where $r_{i}$ is non-zero only for finitely many $i$. %TODO organize
  We can also define $\ph_{\underline m}$ via the universal property of $R^{(I)}$ using maps $R \to M, r \mps r \cd m_{i}$ at component $i \in I$. %TODO make sure statement is right, watch the video.
  \begin{enumerate}[(a)]
    \item $\underline m$ is a generating set of $M \iff \ph_{\underline m}$ is surjective.
    \item $\underline m$ is a basis of $M \iff \ph_{\underline m}$ is an isomorphism.
    \item $M$ is a free $R$-module $\iff M$ has a basis.
    \item $\underline m$ is finitely generated $\iff$ it has a finite generating set.
    \item $\underline m$ is linearly independent $\iff \ph_{\underline m}$ is injective.
  \end{enumerate}
\end{defi}

\begin{rem*}
  Let $\iota_{j}: R \to R^{(I)}$ be the inclusion of the component $j \in I$ (1.18) and set $e_{j}:= \iota_{j}(1)$. Then we call $(e_{j})_{j \in I}$ the standard basis of $R^{(I)}$.
\end{rem*}
  \begin{exmp*}
\begin{enumerate}[(a)]
  \item If $K$ is a field, then any $K$-vector space has a basis.
        \item If $R = \Z$, then $M = \fak \Z{(3)}$ is finitely generated but not free (exercise).
\end{enumerate}

  \end{exmp*}
  \begin{rem}%21
    Every $R$-module is a quotient of a free $R$-module.
\end{rem}
\begin{proof}
Let $R^{(M)}$ be the free $R$-module over the index set $M$, then \[\ph_{\underline m}: R^{(M)} \to M, (r_{m})_{m \in M} \mps \sum_{m \in M}r_{m} \cd m\] is surjective for $\underline m = (m)_{m \in M}$.
\end{proof}
\begin{thm}%22
Let $R$ be commutative, then for $n_{1}, n_{2} \in \Nn$, then we have $R^{n_{1}} \cong R^{n_{2}} \iff n_{1} = n_{2}$.
\end{thm}
\begin{proof}
\begin{itemize}%TODO look at the whole thing
  \item ``$\impliedby$'': (By induction to linear algebra.) Let $\m \subseteq R$ be a maximal ideal. (Axiom of choice) Consider for $n \in \N$ the map $\ph_{n}: R^{n} \to (\fak R\m)^{n}, (r_{1}, \ldots r_{n}) \mps (r_{i} \mod n)_{i \in \eb n}$. Then $\ph_{n}$ is surjective with kernel $\m^{n} \in R^{n} \imp \fak{R^{n}}{\m^{n}} \cong (\fak R\m)^{n}$ by the homomorphism theorem. Now suppose $\psi: R^{n_{1}} \to R^{n_{2}}$ is an isomorphism. We show $n_{1} \ge n_{2}$ (by symmetry of argument we get $n_{1} = n_{2}$).
        Consider the map
        \[\begin{tikzcd}
R^{n_1} \arrow[r, "\cong"] \arrow[rr, "\rho"', bend right] & R^{n_2} \arrow[r, two heads] & \displaystyle\fak{R^{n_2}}{\m^{n_2}}
\end{tikzcd}\]
        this map is surjective and contains $\m^{n_{1}}$ in its kernel (check this). By the homomorphism theorem we get a surjective homomorphism \[\l(\fak R\m\r)^{n_{1}} = \fak{R^{n_{1}}}{\m^{n_{1}}} \to \fak{R^{n_{2}}}{\m^{n_{2}}} = \l(\fak R\m\r)^{n_{2}}\]
        by linear algebra we conclude that $n_{1} \ge n_{2}$.
\end{itemize}
\end{proof}

\begin{defi}%23
If $M$ is free and finitely generated, then define $\mathrm{rank}(M)$ (the rank of $M$) as the unique $n \in \Nn$ such that $M \cong R^{n}$.
\end{defi}

\begin{rem*}
If $R$ is non-commutative, then the rank of the finitely generated $R$-modules is not well-defined. (Jantzen Schwermer Bsp VII.4.2: $R \cong M \cong R^{2}$ for $R = \End_{K}(K[X])$)
\end{rem*}

\subsection{Exact sequences}
\begin{defi}
\begin{enumerate}[(a)]
  \item A diagram of $R$-modules \[M' \xrightarrow{f} M \xrightarrow{g}M''\]
        is called exact (at $M$) $:\iff \ker(g) = \mathrm{im}(f)$
  \item An exact sequence of $R$-modules is a family $(f_{j})_{j \in J}$ of $R$-module homomorphisms $f_{j}: M_{j} \to M_{j+1}$ index of an interval $J \subseteq \Z$, such that $\forall j \in J : j+1 \in J$, the sequence
        \[M_{j} \xrightarrow{f_{j}} M_{j+1} \xrightarrow{f_{j+1}}M_{j+2}\]
        is exact (at $M_{j+1}$). Other notation: %Read what he says
        \[M_{j_{0}} \xrightarrow{f_{j_{0}}} M_{j_{0}+1} \xrightarrow{f_{j_{0}+1}} \cdots \to M_{j+2}\]
        \item An exact sequence $0 \to M' \to M \to M'' \to 0$ is called a short exact sequence (s.e.s.)
\end{enumerate}
\end{defi}

\begin{rem*}
  \begin{itemize}
  \item $0 \to M' \xrightarrow{f} M$ is exact $\overset{\text{Exercise}}\iff f$ is injective.
  \item $M \xrightarrow{g} M'' \to 0$ is exact $\overset{\text{Exercise}}\iff g$ is surjective.
  \end{itemize}
  ($0$ stands for the $0$-module $\{0\}$)
\end{rem*}
\begin{exmp}
  Let $f: M \to N$ be an $R$-module homomorphism. Then one defines
  \[\mathrm{coker}(f):= \fak N{\mathrm{im}(f)}\]
 as the cokernel of $f$ , it comes toether with an $R$-module epimorphism $\pi: N \to \mathrm{coker}(f)$. As an exercise: The sequence
  \[0 \to \ker(f) \xrightarrow{i} M \xrightarrow f N \xrightarrow \pi \mathrm{coker}(f) \to 0\]
  is exact. Subexamples:
  \begin{itemize}
    \item If $f$ is injective, then $0 \to M \xrightarrow f N \to \mathrm{coker}(f) \to 0$ is exact.
          \item If $f$ is surjective, then $0 \to \ker(f) \to M \xrightarrow f N \to 0$ is exact.
  \end{itemize}
\end{exmp}

\begin{rem*}
  For $R$-module homomorphisms $M' \xrightarrow \a M \xrightarrow \b M''$ with $\b \circ \a = 0$, the following are equivalent:
  \begin{enumerate}[(i)]
    \item $0 \to M' \xrightarrow \a M \xrightarrow \b M'' \to 0$ is a s.e.s.
    \item $\b$ is surjective and $\a: M' \to \ker(\b)$ is an isomorphism.
          \item $\a$ is injective and the homomorphism theorem induces an isomorphism $\mathrm{coker}(\a) \cong \fak M{\mathrm{im}(\a)} \to M''$
  \end{enumerate}
  ($\b \circ \a = 0 \iff \mathrm{im}(\a) \subseteq \ker(\b)$)
\end{rem*}


\begin{prop}[Exercise]
\begin{enumerate}[(a)]
  \item Let $0 \to M_{i}' \to M_{i} \to M_{i}'' \to 0$ be short exact sequences $\forall i \in I$, then we get short exact sequences
        \[0 \to \bigoplus_{i \in I}M_{i}' \to \bigoplus_{i \in I}M_{i} \to \bigoplus_{i \in I}M_{i}'' \to 0\]
        \[0 \to \prod_{i \in I}M_{i}' \to \prod_{i \in I}M_{i} \to \prod_{i \in I}M_{i}'' \to 0\]
      \item Suppose $0 \to V_{0} \xrightarrow {f_{0}} V_{1} \xrightarrow{f_{1}} \cdots \xrightarrow {f_{n-1}} V_{n} \to 0$ is an exact sequence of finite dimensional $K$-vector spaces, then: \[\sum(-1)^{i} \dim_{K}(V_{i}) = 0.\]
\end{enumerate}

\end{prop}

\begin{nota}[Commutativity of diagrams]
A diagram of $R$-modules is a directed graph, where any vertex is an $R$-module and any arrow is an $R$-linear map from the module at its source to the module at its target. We call two arrows composable if the target of the first arrow is the source of the second; then the correspoding maps can be composed. So to any chain of composable arrows, the composition of maps defines a map from the source of the first to the target of the last arrow in the chain. A diagram is \textbf{commutative} if for any two chains of arrows with the same source and target, the resulting two maps agree.
\end{nota}
\begin{exmp*}
  \begin{enumerate}[(a)]
  \item To say that the diagram
\[\begin{tikzcd}.
M_1 \arrow[r, "f"] \arrow[d, "g"'] & M_2 \arrow[d, "g'"] \\
M_3 \arrow[r, "f'"']               & M_4
\end{tikzcd}\]
commutes means that $g' \circ f = f' \circ g$.
\item $M \overunderset fg \rightleftarrows N$ commutes $\iff g = h$
   \end{enumerate}
 \end{exmp*}

\begin{thm-defi}
  For a short exact sequence of $R$-modules %TODO fix the thing like in the VL
  \[\begin{tikzcd}
0 \arrow[r] & M' \arrow[r, "f"] & M \arrow[r, "g"] \arrow[l, "s", dashed, bend left] & M'' \arrow[r] \arrow[l, "t", dashed, bend left] & 0
\end{tikzcd} \quad (*)\]
  the following are equivalent:
  \begin{enumerate}[(a)]
    \item $\ex R$-linear map $t: M'' \to M$ such that $g \circ t = \id_{M''}$
    \item $\ex$ submodule $N \subseteq M$ such that \[\psi: \im(f) \oplus N \to M, (b,n) \mps b+n\]
          is an isomorphism.
          \item $\ex R$-linear map $s: M \to M'$ such that $s \circ f = \id_{M'}$.
  \end{enumerate}
  In this case (if (a) - (c) hold), then the sequence $(*)$ is called a split exact sequence. (simply $(*)$ is split or splits), and $t$ (or $s$) is called a splitting of $g$ (or of $f$ respectively).
  \begin{proof}
    \begin{itemize}
      \item $(a){\imp}(b)$: Given $t$, define $N:= \im(t)$ and $\psi$ as above, i.e. $\psi: \im(f) \oplus N \to M, (b,n) \mps b+n$
            \begin{itemize}
              \item $\ker(\psi) = 0$: Let $(b,n) \in \ker(\psi)$, i.e. $n = t(m'')$, for some $m'' \in M''$ and $b = f(m')$ for some $m' \in M'$ and $n+b = 0$ ($\psi(b,n) = 0$).
              \item Apply $g: M \to M''$: \[\ubr{g(n+b)}_{0} = \ubr{g(t(m''))}_{g \circ t = \id_{M''}} + \ubr{g(f(m'))}_{g \circ f = 0} = m''+0\]
                    $\imp m'' = 0 \imp n = t(m'') = 0 \underset{n+b=0}\imp b=0 \imp (b,n) = (0,0)$
                    \item $\im(\psi) = M$: Let $m \in M$, define $n = t(g(m))$ and $b = m-n$. So $n \in N = \im(f)$. $b \in \im(f)$?, to show $b \in \ker(g)$. For this $g(b) = g(m-n) = g(m) - \ubr{g(t(g(m)))}_{g\circ t = \id_{M''}} = g(m) - g(m)= 0$, so $(b,n) \in \im(f) \oplus N$ and $\psi(b,n) = b+n = m$ by definition of $b$.
            \end{itemize}
      \item $(c) {\imp} (b)$ analogous. Define $N = \ker(s)$ ($M' \overunderset fs \rightleftarrows M$). We want to show $\im(f) \oplus N \to M, (b,n) \to b+n$ is an isomorphism.
            \begin{itemize}
              \item $\ker(\psi) = 0$: Check.
              \item $\im(\psi) = M$: For $m \in M$ observe that \[\ubr{f \circ s(m)}_{\in \im(f)} + \ubr{(m-f\circ s(m))}_{\in \ker(s) \text{ check. }} = m\]
            \end{itemize}

      \item $(b) \to (a)$ and $(c)$: Consider the diagram:
            \[
\begin{tikzcd}[column sep= 20, row sep = 20]
0 \arrow[r] & M' \arrow[r, "f'"] \arrow[d, "\id_{M'}"] \arrow[r, "{\small \ \ \ \ m' \mps (f(m'),0)}"'] & \im(f) \oplus N \arrow[r, "g'"] \arrow[d, "\psi"] \arrow[l, dashed, bend right] \arrow[r, "{\small (b,n) \mps g(n)}"'] & M'' \arrow[r] \arrow[d, "\id_{M''}"]            & 0 \\
0 \arrow[r] & M' \arrow[r, "f"]                                                                 & M \arrow[r, "g"] \arrow[l, "?", dashed, bend left]                                                                     & M'' \arrow[r] \arrow[l, "?", dashed, bend left] & 0
\end{tikzcd}
            \]
            The diagram commutes. $\psi \circ f' = f, g \circ \psi = g'$, e.g: \[\psi \circ f'(m') = \psi(f(m'), 0) = f(m') + 0 = f(m')\] and
            \[g \circ \psi(b,n) = g(b+n) = \ubr{g(b)}_{=0} = g(n) = g(n) = g'(b,n)\]
            ($g(b) = 0$ is because $b \in \im(f) = \ker(g)$).

      \item For $s$: $f: M' \to \im(f)$ is an isomorphism ($f$ is injective) $\imp f^{-1}: \im(f) \to M'$ is an isomorphism. Check
            \[s = (f^{-1},0) \circ \psi^{-1}: \begin{tikzcd}[row sep = 30]
M \arrow[r, "\psi^{-1}"] & \im(f) \oplus N \arrow[r, "{\begin{matrix}(b,n) \mps f^{-1}(b) \\ \end{matrix}}"] & M'
\end{tikzcd}\]
      \item For $t$: Check that $s : N \to M''$ is an isomorphism using (b). Set $t:= i \circ g^{-1}$ for $i$ the inclusion so \[t: M'' \to N \hookrightarrow M\]Check. \qedhere
\end{itemize}
  \end{proof}
\end{thm-defi}
\begin{rem*}
  $M' \overunderset fs \rightleftarrows M $ and $M'' \overunderset tg \rightleftarrows M $ satisfy the condition from corollary 1.19, namely:
  \begin{itemize}
    \item $s \circ f = \id_{M'}$
    \item $g \circ t = \id_{M''}$
    \item $t \circ g + f \circ s = \id_{M}$
  \end{itemize}
  shows again: the sequence is split if $M \cong M' \oplus M''$ (for the ``right maps'')
\end{rem*}
\begin{rem*}
  One also has short exact sequences for groups
  \[1 \to \ker(\pi) \overset s \leftrightarrows G \overunderset t \pi \leftrightarrows \bar G \to 1\]
  Here one has to be careful what splitting means. Having a $t$ is not equivalent to having an $s$.
  \[\ex t \iff G \cong \ker(\pi) \rtimes \bar G\]
  \[\ex s \iff G \cong \ker(\pi) \tm \bar G\]
\end{rem*}

\end{document}
