\documentclass[a4paper]{report}
\usepackage{../template}
\begin{document}
\begin{defi}
  Let $R$ be a commutative ring.
  \begin{enumerate}[(a)]
    \item A category $\A$ is called $R$-linear if
          \begin{enumerate}[(i)]
            \item $\forall X, Y \in \A, \A(X,Y)$ is an $R$-module $(\A(X,Y), 0_{X,Y}, +_{X,Y}, \cd_{X,Y})$.
                  \item $\forall X, Y, Z \in \A$ the composition map \begin{align*}
                    \A(X,Y) \tm \A(Y,Z) & \to \A(X,Z) \\
                    (\ph, \psi) &\mps \psi \c \ph
                  \end{align*}
                  is $R$-bilinear. (in particular $r(\psi \c \ph) = (r \cd \psi) \c \ph = \psi \c(r \ph)$).
          \end{enumerate}
    \item A functor $F: \A \to \A'$ between $R$-linear categories is called $R$-linear if $\forall X, Y \in \A$ the map $F: \A(X,Y) \to \A'(FX,FY)$ is $R$-linear.
          \item For $\Z$-linear (preadditive) categories $\A$ and $\A'$, the full subcategory $\Add(\A,\A') \subseteq \Fun(\A,\A')$ of $\Z$-linear (additive) functors between them is again a $\Z$-linear category, i.e. there is a natural addition on natural transforations.
  \end{enumerate}
\end{defi}

\begin{exmps*}
\begin{enumerate}[(a)]
  \item $\RMod$ is $\Z$-linear and in fact ($R$ is commutative) $R$-linear.
  \item $\A$ is $R$-linear $\iff \A\op$ is also $R$-linear.
  \item If $\A$ and $\A'$ are $R$-linear, then $\A \tm \A'$ is $R$-linear.
        \item Let $S$ be a (not necessarily commutative) ring and $\underline S$ the category \[(\{*\}, S, \dom = \{*\}, \cod = \{*\}, \id_{*} = \id_{S}, \c = \cd_{S})\] associated to $S$, then $\underline S$ is $\Z$-linear $(\Hom_{\underline S}(*,*) = S)$.
\end{enumerate}
\end{exmps*}

\begin{lemm}
  If $\A$ is $\Z$-linear, then for $X \in \A$ the following are equivalent:
  \begin{enumerate}[(i)]
\item $X$ is initial.
    \item $X$ is terminal.
          \item $\A(X,X)=0$ (in particular $\id_{X} = 0$)
  \end{enumerate}
\begin{proof}
Exercise.
\end{proof}
\end{lemm}


\begin{defi}
  An $X \in \A$ satisfying conditions of Lemma 2 is called a \emph{zero object}
\end{defi}
\begin{nota*}
The zero object of $\A$ (if it exists) is unique up to unique isomorphism so we just write $0$ or $0_{A}$ for this object.
\end{nota*}

\begin{lemm}
  Let $\A$ be $\Z$-linear with a 0-object and $X, Y \in \A$, then the zero map $0_{X,Y} \in \A(X,Y)$ (remember $\A(X,Y)$ is an abelian group) is equal to the composition $X \xrightarrow{\ex !} 0\xrightarrow{\ex !} Y$
  \begin{proof}
Exercise.
  \end{proof}
\end{lemm}

\begin{prop}
  Let $\A$ be $\Z$-linear, then for $X_{1}, X_{2} \in \A$ the following are equivalent:
  \begin{enumerate}[(i)]
    \item The product $X_{1} \mathrel{\Pi} X_{2}$ exists.
    \item The coproduct $X_{1} \amalg X_{2}$ exists.
          \item $\ex Y \in \A, p_{1}, p_{2}: Y \to X_{i}$ and $\iota_{1}, \iota_{2}: X_{i} \to Y$ such that \[p_{i} \c \iota_{j} = \begin{cases}
            1_{X_{i}}, & i = j \\
            0, & i \ne j
          \end{cases}\]
  \end{enumerate}
\end{prop}
\begin{proof}
TODO.
\end{proof}
\begin{rem*}
In (iii) $(Y,\iota_{1}, \iota_{2})$ is the coproduct and $(Y, p_{1}, p_{2})$ is the product.
\end{rem*}

\begin{defi}
In a $\Z$-linear category we deonte $X \mathrel \Pi Y = X \amalg Y$ by $X \opl Y$ and call it the \emph{direct sum} of $X$ and $Y$.
\end{defi}

\begin{lemm}
  Let $\A$ and $\A'$ be $\Z$-linear and $F: \A \to \A'$ an additive functor, then
  \begin{enumerate}[(i)]
    \item If $0_{\A}$ exists then $0_{\A'}$ exists and $F(0_{\A}) \cong 0_{\A'}$.
          \item If $X, Y \in \A$ and $X \opl Y$ exists, then $FX \opl FY$ exists and $FX \opl FY \cong F(X\opl Y)$.
  \end{enumerate}
\begin{proof}
Exercise.
\end{proof}
\end{lemm}


\begin{defi}
A category $\A$ is called \emph{additive} (or $R$-linear additive) if $\A$ is $\Z$-linear (or $R$-linear), $0_{\A}$ exists and $\forall X, Y \in \A: X \opl Y$ exists.
\end{defi}

\begin{rem*}
If $\A$ is additive then the addition on $\A(X,Y)$ is determined by the composition map! (Morel II 1.2.4)
\end{rem*}

\section{Kernels and cokernels}
Recall that the equalizer of two morphisms $f, g: X \to Y$ is the limit of the diagram $X \overunderset fg \rightrightarrows Y$, so $Z \xrightarrow u X\overunderset fg \rightrightarrows Y$ such that for every $W \xrightarrow v X \overunderset fg \rightrightarrows Y$ with $f \c v = g \c v$, $v$ factors through $u$:
\[\begin{tikzcd}
Z \arrow[r, "u"]                     & X \arrow[r, "f", shift left] \arrow[r, "g"', shift right] & Y \\
W \arrow[ru, "v"'] \arrow[u, dashed] &                                                           &
\end{tikzcd}\]similarly the coequalizer is the colimit of $X\overunderset fg \rightrightarrows Y$.
\newline
In the category of $R$-modules, $\eq(f,g) = \ker(f - g)$ and $\coeq(f,g) = \cok(f - g)$. In particular $\ker f = \eq(f,0)$.

\begin{defi}
  Let $\A$ be a $\Z$-linear category and $f \in \A(X,Y)$.
  \begin{enumerate}[(a)]
    \item define $\ker f := \eq(f,0)$ and $\cok f = \coeq(f,0)$ (if they exist)
    \item If $\ker f$ exists, define the coimage as $X \to \coim f = \cok(\ker f \to X)$ (it might not exist).
          \item If $\coker f$ exists, define the image as $\im f \to Y = \ker(Y \to \cok f)$.
  \end{enumerate}
\end{defi}

\begin{exmp}
Let $X_{1}, X_{2} \in \A$ and $(X_{1} \opl X_{2}, \iota_{1}, \iota_{2}, p_{1}, p_{2})$ the direct sum. Then $\iota_{1} = \ker p_{2}, \iota_{2} = \ker p_{1}, p_{1} = \cok \iota_{2}, p_{2} = \cok \iota_{1}$
\end{exmp}

\begin{lemm}
  Kernels are monomorphisms and cokernels are epimorphisms.
\end{lemm}
\begin{proof}
  We only prove the statement for kernels. Let $\ker f \xrightarrow \iota X \overunderset f0 \rightrightarrows Y$, now assume $\iota \c \ph = \iota \c \ph'$
  \[\begin{tikzcd}
Z \arrow[r, "\varphi", shift left] \arrow[r, "\varphi'"', shift right] & \ker f \arrow[r, "\iota"] & X \arrow[r, "f", shift left] \arrow[r, "g"', shift right] & Y
\end{tikzcd}\]
By the definition of equalizer \begin{align*}
  \A(Z, \ker f) &\simeq \{\psi: Z \to X \mid f \c \psi = 0\} \\
  \ph &\mps \iota \c \ph
\end{align*}
is bijective, so $\ph = \ph'$.
\end{proof}


\begin{thm}Let $f \in \A(X,Y)$ and assume $\ker f, \cok f, \im f, \coim f$ exist. Then there exists a unique morphism $u: \coim f \to \im f$ such that $f$ is equal to the composition
  \[\begin{tikzcd}
      X \arrow[r, "c"] & \coim f \arrow[r, "\ex!u", dashed] & \im f \arrow[r, "d"] & Y
    \end{tikzcd}\]
  This is called the \emph{canonical factorization} of $f$ (or epi-mono factorization).
\end{thm}
\begin{rem*}
Note that by lemma 11, $c$ is an epimorphism and $d$ is a monomorphism.
\end{rem*}
\begin{proof}TODO.

\end{proof}
Note that in the category of $R$-modules, $u$ is an isomorphism.


\begin{defi}
  A category $\A$ is called \emph{abelian} if
  \begin{enumerate}[(i)]
    \item $\A$ is additive ($\Z$-linear, 0 exists and $X \opl Y$ exists)
    \item $\forall f \in \A(X,Y): \ker f, \coker f$ exist.
          \item $\forall f \in \A(X,Y)$ with canonical factorization $f = d \c u \c c$, $u$ is an isomorphism.
  \end{enumerate}
\end{defi}

\begin{exmp}
\begin{itemize}
  \item For any ring $R$ the category $\RMod$ is abelian.
        \item For rings $R, R'$ the category $\RMod_{R'}$ is abelian $( \cong {}_{R \otm_{\Z}(R')\op}\Mod )$.
\end{itemize}
\end{exmp}

\begin{exmp}
    \item Let $\A \subseteq \Ab$ be the fullsubcategory of finitely generated free $\Z$-modules, then $\A$ is additive.
    \item For $f: X \to Y$ in $\A$ the usual $\ker_{\Ab} f$ as abelian group is again a finitely generated free abelian group and is also the kernel in $\A$.
    \item The cokernel however might have torsion. In fact \[\coker_{\A} f = \fak{\coker_{\Ab} f}{\Tor(\coker_{\Ab} f)}\]
          So kernels and cokernels exists in $\A$, now let $f: \Z \to \Z, n \mps 2n$, the canonical factorization is \[\begin{tikzcd}
\Z \arrow[r, "\id"] & \Z \arrow[r, "u"]    & \Z \arrow[r, "\id"] & \Z \\[-20]
                    & n \arrow[r, maps to] & 2n                  &
\end{tikzcd}\]since $u$ is not an isomorphism, $\A$ is not abelian.
\end{exmp}


\begin{prop}
  Let $\C$ be a category, then
 \begin{enumerate}[(i)]
   \item $\C$ is $\Z$-linear $\iff \C\op$ is $\Z$-linear.
   \item $\C$ is additive $\iff \C\op$ is additive.
         \item $\C$ is abelian $\iff \C\op$ is abelian.
 \end{enumerate}
 Moreover (if they exist) $\ker_{\C} f = \coker _{\C\op} f$ and $\im_{\C} f = \coim_{\C\op} f$.
\end{prop}


\section{Abelian categories}
From now on let $\A$ be an abelian category.
\begin{rem*}
Let $f: X \to Y$ and $X \twoheadrightarrow \coim f \overunderset \cong u \rightarrow \im f \hookrightarrow Y$ be the canonical factorization. Then either $X \twoheadrightarrow \coim f \hookrightarrow Y$ or $X \twoheadrightarrow \im f \hookrightarrow Y$ (or anything else isomorphic to them) is called ``the'' canonical factorization for $f$.
\end{rem*}

\begin{prop}
  Let $X, Y \in \A$
  \begin{enumerate}[(a)]
    \item $\coker (0 \to X) = X \xrightarrow \id X$ and $\ker(X \to 0) = X \xrightarrow \id X$.
    \item $f: X \to Y$ is a monomorphism $\iff \ker f = 0 \iff$ the canonical factorization of $f$ is $X \xrightarrow \id X \overset f \hookrightarrow Y \iff f$ is a kernel.
    \item $g: C \to Y$ is an epimorphism $\iff \coker g= 0 \iff$ the canonical factorization of $g$ is $X \xrightarrow g Y \xrightarrow \id Y \iff g$ is a cokernel.
          \item $u$ is an isomorphism $\iff u$ is a monomorphism and an epimorphism.
  \end{enumerate}
\end{prop}


\begin{cor}
$X \xrightarrow a Z \xrightarrow b Y$ is the canonical factorization for $f = b \c a \iff a$  ia a monomorphism and $b$ an epimorphism.
\end{cor}

\begin{defi}
In an abelian category $f \in \A(X,Y)$ is called injective if $\ker f = 0$ and surjective if $\coker f = 0$.
\end{defi}

\begin{cor}
  Let $f \in \A(X,Y)$, then
 \begin{enumerate}[(a)]
\item $f$ is injective $\iff f$ is a monomorphism.
\item $f$ is surjective $\iff f$ is a epimorphism.
\item $f$ is injective and surjective $\iff f$ is a isomorphism.
 \end{enumerate}
\end{cor}


\section{Exactness}
\begin{lemm}
  Let $(g,f)$ be a composable pair of morphisms in $\A$ such that $g \c f = 0$, then $\ex$ a canonical injection $\im f \hookrightarrow \ker g$.
\end{lemm}
\begin{proof}
  Let $f: X \to Y$ and $g: Y \to Z$ and write the epi-mono factorization of $f$:
  \[\begin{tikzcd}
X \arrow[r, "c", two heads] & \coim f \arrow[r, "\cong"] & \im f \arrow[r, "d", hook] & Y \arrow[r, "g"] & Z
\end{tikzcd}\]and call the isomorphism in the middle $u: \coim f \cong \im f$.
Now $g \c d \c u \c c = 0 \imp g \c d \c u = 0$ since $c$ is an epimorphism and $u$ isomorphism $\imp g \c d = 0$. By the definition of the kernel $d$ must factor through $\ker g$:
\[\begin{tikzcd}
\im f \arrow[r, "e"] & \ker g \arrow[r, "\iota", hook] & Y \arrow[r, "g"] & Z
\end{tikzcd}\]
Since $d$ is injective: if $e \c \ph = c \c \ph ' \imp \iota \c c \c \ph = \iota \c c \c \ph' \imp d \c \ph = d \c \ph' \imp \ph = \ph ' \imp e$ is a monomorphism $\imp \im f \hookrightarrow \ker g$.
\end{proof}
\end{document}
