\documentclass[a4paper]{report}
\usepackage{../template}
\begin{document}
\addcontentsline{toc}{subsection}{Ring/Einheitengruppe}
\begin{whg*}
  $(R, 0, 1, +, \cd)$ ist ein \textbf{Ring} $\iff (R, 0, +)$ ist eine Gruppe, $(R, 1, \cd)$ ist ein Monoid und es gelten die Distributivgesetze.
  \[R^{\tm} = \{r \in R \mid \ex s \in R : rs = sr = 1\}\]
  ist die Einheitengruppe von $R$
\end{whg*}
\begin{bsp*}
(Übung) $\Zn\en = \{\bar a \mid \ggt(a, n) = 1\}$, wobei $\Zn = \fak \Z{n\Z} = \fak {\Z} {(n)}$
\end{bsp*}

\addcontentsline{toc}{subsection}{Ringhomomorphismus}
\begin{defi}[\textbf{Ringhomomorphismus}]
  Seien $R, R'$ Ringe, eine Abbildung $\ph : R \to R'$ heißt Ringhomomorphismus wenn:
  \begin{itemize}
    \item $\ph : (R, 0, +) \to (R', 0', +')$ ist ein Gruppenhomomorphismus.
    \item $\ph : (R, 1, \cd) \to (R', 1', \cd')$ ist ein Monoidhomomorphismus.
  \end{itemize}
  $\ph$ ist ein Ringisomorphismus $\iff \ph$ ist bijektiver Ringhomomorphismus $\underset{\text{Übung}}\iff \ex \ph': R' \xrightarrow{\text{Ringhom.}} R$, sodass $\ph \circ \ph' = \id_{R'}$ und $\ph' \circ \ph = \id_{R}$. In diesem Fall schreibe $R \cong R'$ ($R$ isomorph zu $R'$).
\end{defi}
\begin{bsp*} $R$ heißt Nullring $\iff 0_{R} = 1_{R} \ueb\iff R = \{0_{R}\}$ (alle Nullringe sind isomorph.)
\end{bsp*}
\begin{bsp*}(Übung)
  Sei $R$ beliebig $\imp \ex!$ Ringhomomorphismus $\ph: \Z \to R$ nämlich
  \[\ph : \Z \to R, n \mapsto \ph(n) = n\cdot 1_{R}\]
  (wegen $\ph(1) = 1_{R}$)
\end{bsp*}

\addcontentsline{toc}{subsection}{Unterring}
\begin{defi}[\textbf{Unterring}]
  $S \subseteq R$ heißt Unterring, falls
  \begin{itemize}
    \item $1 \in S$
    \item $S - S = \{s_{1}-s_{2} \mid s_{1}, s_{2} \in S\} \subseteq S$
    \item $S + S = \{s_{1}+s_{2} \mid s_{1}, s_{2} \in S\} \subseteq S$
  \end{itemize}
\end{defi}
\addcontentsline{toc}{subsection}{Produkt von Ringen}
\begin{defi*}[\textbf{Produkt von Ringen}]
  Seien $R_{1}, R_{2}$ Ringe, dann ist $(R_{1} \tm R_{2}, (0,0), (1,1), +, \cd)$ ein Ring mit komponentenweiser Addition und Multiplikation.
  \[+ : (R_{1} \times R_{2})^{2} \to R_{1} \times R_{2}, (r_{1}, r_{2}) + (s_{1}, s_{2}) = (r_{1} + s_{1}, r_{2} + s_{2})\]
  \[\cd : (R_{1} \times R_{2})^{2} \to R_{1} \times R_{2}, (r_{1}, r_{2}) \cd (s_{1}, s_{2}) = (r_{1} \cd s_{1}, r_{2} \cd s_{2})\]
\end{defi*}

\begin{bem*}[Übung]\item
  \begin{enumerate}[(a)]
    \item Sei $R$ ein kommutativer Ring, $S \subseteq R$ ein Unterring, dann ist $S$ kommutativ.
    \item Seien $R_{1}, R_{2}$ kommutative Ringe, so ist auch $R_{1} \tm R_{2}$ kommutativ.
  \end{enumerate}
\end{bem*}

\begin{whg*}
Seien $I, X$ Mengen. Eine Folge/Familie in $X$ über (Indexmenge ) $I$, geschrieben $(x_{i})_{i \in I}$ ist eine Abbildung $x: I \to X, i \mapsto x_{i}$. Schreibe $X^{I}$ für die Menge aller Folgen in $X$ über $I$ ($=\Abb(I,X)$)
\end{whg*}

\addcontentsline{toc}{subsection}{Monoidring}
\begin{bsp}[\textbf{Monoidring}]
  Sei $R = (R, 0, 1, +, \cd)$ ein kommutativer Ring und $M = (M, e, \circ)$ ein Monoid. Definiere
  \begin{enumerate}[(i)]
    \item $R[M] := \{(a_{m})_{m \in M} \in R^{M} \mid (E): \# \{m \in M : a_{m} \ne 0\} < \infty\}$
    \item $\underline 0 = $ die Abbildung $M \to \{0\} \subseteq R$
    \item $\underline 1 = $ die Folge $(\delta_{em})_{m \in M}$ mit $\delta_{em} = \begin{cases} 1, & m = e, \\ 0, & m \ne e.\end{cases}$
    \item Verknüpfungen $+, \cd : R[M] \tm R[M] \to R[M]$ durch:
          \[(a_{m})_{m \in M} + (b_{m})_{m \in M} := (a_{m}+b_{m})_{m \in M}\]
          und
          \[(a_{m})_{m \in M} \cd (b_{m})_{m \in M} := (c_{m})_{m \in M}\]
          mit (Übung)
          \[c_{m}:= \sum_{\substack{(m', m'') \in M \tm M \\ m' \cd m'' = m}}a_{m'} \cd b_{m''}\]
          die Summe ist endlich wegen $(E)$ und wegen $(E)$ gilt: $\#\{m \mid c_{m \ne 0}\} < \infty$
  \end{enumerate}
\end{bsp}

\begin{nota*}
\[\sum_{m \in M}a_{m} \cd m \text{ für } (a_{m})_{m \in M} \in R[M]\]
\end{nota*}

\begin{ubng}\item
\begin{enumerate}[ a)]
\addcontentsline{toc}{subsection}{Ringhomomorphismus}
  \item $(R[M], \underline 0, \underline 1, +, \cd)$ ist ein Ring, ($R[M]$ heißt \textbf{Monoidring} zu $M$ über $R$)
  \item Ist $M$ abelsch, so ist $R[M]$  kommutativ.
  \item Ist $\ph: R \to S$ ein Ringhomomorphismus und $\s : M \to (S, 1, \cd)$ ein Monoidhomomorphismus, so $\ex !$ Ringhomomorphismus $\psi : R[M] \to S$ mit $\psi|_{R} = \ph$ und $\psi|_{M} = \s$. (dabei wir identifizieren $R$ mit $R \cd e = R \cd 1$ (1-Folge) und $M$ mit $1_{R} \cd M$), nämlich:
        \[\psi\ubr{\l(\sum a_{m}\cd m\r)}_{\text{in }R[M]} = \ubr{\sum \ph(a_{m})\cd \s(m)}_{\text{in } S}\]
\end{enumerate}
\end{ubng}

\begin{kon*}
Ab nun seien alle Ringe $R, R', S, R_{i}$ kommutativ, (und es Seien in §3 stets Ringe)
\end{kon*}

\section{Polynomringe}
\begin{bsp}
  Die folgenden Strukturen sind abelsche Monoide:
  \begin{enumerate}[(i)]
    \item $(\Nn, 0, +) = \Nn$
    \item $(\Nn^{n}, (0,...,0), +) = \bigtimes_{i \in \eb n} \Nn$ (Komponentenweise Addition)
    \item Für $I$ eine beliebige Menge:
          \((\Nn^{(I)}, \underline 0, \underline +)\) mit
          \[\Nn^{(I)}= \{(a_{i})_{i \in I} \in \Nn \text{ Folgen über }I \mid \#\{i \in I : a_{i} \ne 0\} < \infty\}\]
          $\underline 0 = 0$-Folge und $\underline +$ komponentenweise Addition in $\Nn^{(I)}$.
  \end{enumerate}
\end{bsp}
\begin{facts}[Übung]\item
\begin{enumerate}[(i)]
  \item $\Nn^{n} \cong \Nn^{(\eb n)}, (a_{i})_{i \in \eb n} \mps (a_{i})_{i \in \eb n}$
  \item Für $i \in I$ sei $e_{i} \in \Nn^{(I)}$ die Folge mit $e_{i}(j) =
        \begin{cases}
          1, & j = i, \\ 0 & j \ne i.
        \end{cases}$

        (betrachte $e_{i} : I \to \Nn$ als Abbildung) Damit ist jede Folge $\underline a = (a_{i})_{i \in I} \in \Nn^{(I)}$ eindeutige Linearkombination mit Koeffizienten in $\Nn$, nämlich:
        \[\underline a = \sum_{i \in I}a_{i} \cd e_{i} = \sum_{i \in I, a_{i} \ne 0}a_{i} \cd e_{i}\]
        Beachte: $\Nn^{(I)} \subseteq \Q^{(I)}$ (analog definiert, Folgen in $\Q$ über $I$) mit Endlichkeitsbedingung $(E)$. Und $(e_{i})_{i \in I}$ ist eine Basis von $\Q^{(I)}$ als $\Q$-Vektorraum. Man sagt auch $\Nn^{(I)}$ ist freies abelsches Monoid über der Basis $(e_{i})_{i \in I}$.
    \item Ist $M$ ein abelsches Monoid und $(m_{i})_{i \in I}$ eine Folge in $M$, so $\ex!$ Monoidhomomorphismus \[\ph: \Nn^{(I)} \to M, \ph(e_{i}) = m_{i}\]
\end{enumerate}
\end{facts}

\begin{whg*}
  $R[X]$ ist der Polynomring über $R$ in Variablen $X$. Elemente sind $\sum_{n \ge 0} a_{n}X^{n}, (a_{n} \in R)$ nur endlich viele $a_{n} \ne 0$. $+, \cd$ auf $R[X]$ sind definiert durch
  \[\sum a_{i}X^{i} + \sum b_{i}x^{i} = \sum(a_{i}+b_{i})X^{i}\]
  \[\l(\sum a_{i}X^{i}\r)\l(\sum b_{i}X^{i}\r) = \sum_{i}\l(\sum_{j=0}^{i}a_{j}b_{i-j}\r)X^{i}\]
\end{whg*}

\begin{prop}
  Die folgende Abbildung ist ein Ringisomorphismus.
  \[\psi : R[\Nn] \to R[X], \sum_{i \in \Nn} r_{i}i \mapsto \sum_{i \in \Nn}r_{i}X^{i}\]
  \begin{proof}[Beweis]\item
    \begin{itemize}
      \item $\psi$ wohldefiniert und bijektiv:
            \[R[\Nn] = \text{ Folgen } (r_{i})_{i \in \Nn} \text{ mit } \#\{i \mid r_{i} \ne 0\} < \infty\]
            \[R[X] = \text{ analog }\]
      \item Ringstruktur:
            \begin{itemize}
              \item Addition (Übung)
              \item Multiplikation
                    \[\ubr{\l(\sum_{i \in \Nn} r_{i}\cd i\r)}_{f \in R[\Nn]}\ubr{\l(\sum_{j \in \Nn} s_{j}\cd j\r)}_{g} \underset{\text{Nach Def.}}= \sum_{k \in \Nn} s_{k}\cd k, \quad s_{k}\]
                    \[= \sum_{0 \le i,j, i+j = k}r_{i}s_{j} = \sum_{j=0}^{k}r_{j}s_{k-j}\]
                    \[\imp \psi(f \cd g) = \psi\l(\sum_{k} s_{k}\cd k\r) = \sum_{k}g_kX^{k}\]
                    \[ = \sum_{i}a_{i}\cd  \sum_{j}b_{j}X^{j} = \psi(f)\psi(g).\qedhere\]
            \end{itemize}
    \end{itemize}
  \end{proof}
  Formal: $\{0, 1, \cdots\} \to \{X^{i} \mid i \in \Nn\}$.
\end{prop}

\begin{prop}[Universelle Eigenschaft von {$R[X] \cong R[\Nn]$}] $\forall \psi : R \to S$ Ringhomomorphismen und $\forall s \in S \ex !$ Ringhomomorphismus $\hat \psi : R[X] \to S$ mit $\hat \psi|_{R} = \psi$ und $\hat \psi(X) = s$
\begin{proof}[1. Beweis]
  Definiere $\hat\psi(\sum_{i \ge 0}r_{i}X^{i}) := \sum_{i \ge 0}\underbrace{\psi(r_{i})}_{\in S}s^{i}$. Dann die Behauptung nachprüfen.
\end{proof}
\begin{proof}[2. Beweis]
  Facts 6(iii) $\ex !$ Monoidhomomorphismus $\s : \Nn \to (S, 1, \cd)$ mit $\s(1) = s$ und Übung 4(c) (universelle Eigenschaft des Monoidrings) $\ex !$ Ringhomomorphismus $\hat \psi : R[\Nn] \to S$ mit $\hat \psi|_{R} = \psi$ und $\hat\psi|_{\Nn} = 0$. Dieser erfüllt die Aussagen in Prop 8, denn $\hat \psi(X) = \hat \psi(1) = s$, $X$ entspricht $1 \in \Nn $ (Unter Isomorphismus von Proposition 7).
\end{proof}
\end{prop}
\begin{defi*}
  Für $n \ge 1$ Variable: ($n \in \N$)
  \[R[X_{1}, \ldots, X_{n}] := (R[X_{1}, \ldots, X_{n-1}])[X_{n}] = \cdots = (\cdots((R[X_{1}])[X_{2}]) \cdots)[X_{n}]\]
\end{defi*}
\begin{satz}
  Sei $\ph : \Nn^{n} \to (R[X_{1}, \ldots X_{n}], 1, \cd)$ der eindeutige Monoidhomomorphismus mit $\ph(e_{i}) = X_{i}$, wobei $e_{i} = (\delta_{i,j})_{j} = (0, \cdots, 1, \cdots 0)$ für $i \in \eb n$. Dann ist (nach 4(c) eindeutige) Ringhomomorphismus $\hat \psi : R[\Nn^{n}] \to R[X_{1}, \ldots, X_{n}]$ mit $\hat \psi|_{R} = \id_{R}$ und $\hat \psi|_{\Nn^{n}} = \ph$ ein Ringisomorphismus.
  \begin{proof}[Beweis]
    (Übung) Hierbei wird $m = (m_{1}, ..., m_{n}) \in \Nn^{n}$ identifiziert (unter $\hat \psi$) mit $X_{1}^{m_{1}}\cdot \ldots \cd X_{n}^{m_{n}}$
  \end{proof}
\end{satz}

\addcontentsline{toc}{subsection}{Polynomring}
\begin{defi}[\textbf{Polynomring}]
  Der \textbf{Polynomring} in den Variablen $(X_{i})_{i \in I}$ ($I$ beliebige Menge) ist definiert als
  \[R[X_{i} \mid i \in I] := R[\Nn^{(I)}]\]
  Elemente in diesem Ring sind
  \[\sum_{a \in \Nn^{(I)}}r_{a}\cd a\]
  mit $r_{a} \in R$ und es gilt $\#\{a \in \Nn^{(I)} \mid r_{a} \ne 0\} < \infty$.
\end{defi}
\begin{nota*}
  Andere Notation: Für $a \in \Nn^{(I)}$ schreibe für $a$ \[X^{a} \text{ oder } \prod_{i \in I, a_{i} \ne 0} X_{i}^{a_{i}}\]
  Insbesondere ist \(X^{e_{i}} = X_{i}\), wobei $e_{i}$ die Folge in $\Nn^{(I)}$ mit $e_{i}(j) = \delta_{i,j}$ ist.
  \item Monoidaddition $a + b$ entspricht \[X^{a} \cd X^{b} = X^{a + b}\]
  (bilden $a + b$ in $(\Nn^{(I)}, \underline 0, +)$ und $(a_{i})_{i \in I} + (b_{i})_{i \in I} = (a_{i} +_{\Nn} b_{i})_{i \in I}$) Also $+$ ist nicht die Addition im Ring.
\end{nota*}
\addcontentsline{toc}{subsection}{Primitives Monom}
\begin{defi*}[\textbf{Primitives Monom}]
Die Elemete in $R[\Nn^{(I)}]$ sind Summen \[\sum_{a \in \Nn^{(I)}}r_{a} \cd X^{a}\] (Polynome wie gewohnt.) Die Elemente $X^{a}, a \in \Nn^{(I)}$ heißen \textbf{primitive Monome}. Jedes Element in $R[X_{i} \mid i \in I]$ ist eine eindeutige Linearkombination in den Monomen $X^{a}, a \in \Nn^{(I)}$, mit Koeffizienten $r_{a}$ aus $R$, sodass $\#\{a \in \Nn^{(I)} \mid r_{a} \ne 0\} < \infty$, d.h. als $R$-Modul ist $R[X_{i} \mid i \in I]$ frei über $R$ mit Basis $X^{a}, a \in \Nn^{(I)}$
\end{defi*}
\begin{bsp*}
$(2, 5, 3) \in \Nn^{3}$ entspricht $X_{1}^{2}X_{2}^{5}X_{3}^{3}$
\end{bsp*}

\begin{satz}[Universelle Eigenschaft von {$R[X_{i} \mid i \in I]$}] Zu Ringhomomorphismus $\psi : R \to S$ und einer Folge $(s_{i})_{i \in I}$ aus $S$ über $I$ $\ex !$ Ringhomomorphismus $\hat \psi : R[X_{i}\mid i \in I] \to S$ mit $\hat \psi|_{R} = \psi$ und $\hat \psi(X_{i}) = s_{i}$
\end{satz}

\begin{facts*}\item
\begin{enumerate}[(a)]
  \item Für $J \subseteq I$ existiert eindeutiger Monoidhomomorphismus $\Nn^{(J)} \to \Nn^{(I)}$ mit $e_{j} \mps e_{j}$ und ein induzierter Ringhomomorphismus (für $j \in J$) \[\hat \psi: R[\Nn^{(J)}] = R[X_{j} \mid j \in J] \to R[\Nn^{(I)}] = R[X_{i} \mid i \in I]\]
        mit $\hat \psi|_{R} = \id_{R}$ und $\hat \psi (X_{j}) = X_{j}$ ($j \in J$). Die Abbildung $\hat \psi$ ist injektiv deswegen betrachten wir $R[X_{j} \mid j \in J]$ als Unterring von $R[X_{i} \mid i \in I]$
  \item Es gilt: \[R[X_{i} \mid i \in I] = \bigcup_{J \subseteq I \text{ endl.}} R[X_{j} \mid j \in J]\]
        d.h. jedes Polynom im Ring ist Polynom in nur endlich vielen Variablen.
\end{enumerate}
\end{facts*}

\addcontentsline{toc}{subsection}{Grad, Lietkoeffizient, normiertes Polynom}
\begin{defi}\item
\begin{enumerate}[(a)]
  \item $\grad : R[X] \to \Nn \cup \{- \infty\}$ ist die eindeutige Abbildung mit
        \[\grad(f) = \grad\l( \sum_{i \ge 0}r_{i}X^{i} \r) = \begin{cases} -\infty, & f = 0, \\
                                                          \max\{i \in \Nn \mid r_{i} \ne 0\}, & f \ne 0
                                                        \end{cases}\]
  \item Der \textbf{Leitkoeffizient} von $f \ne 0$ ist $a_{\grad(f)}$.
  \item $f \ne 0$ heißt \textbf{normiert} $\iff a_{\grad(f)} = 1$.
  \item Ist $R = K$ ein Körper, so gelten außerdem
        \[\grad(fg) = \grad(f) + \grad(g)\]
        wobei $-\infty + n = n + -\infty = - \infty + (- \infty) = - \infty$ für $n \in \Nn$. Genügt: $R$ ist Integritätsbereich.
  \item Falls $R$ ein Körper (oder Integritätsbereich), so gilt
        \[(R[X])\en = \{f \in R[X] \mid \ex g \in R[X] : fg = 1\}\]
        \[\ueb = \{f \in R[X] \mid \grad(f) = 0, \ex g \in R[X] : \grad g = 0 : fg = 1\} \]
        \[ = \{f \in R \mid \ex g \in R : fg = 1\} = R\en\]
\end{enumerate}
\end{defi}
\section{Symmetrische Polynome}%
Sei $R$ ein kommutativer Ring, $n \in \N$ fest.
\begin{bez*}
\begin{enumerate}[(a)]
  \item Ein Monom in $R[X_{1}, ..., X_{n}]$ ist ein Polynom der Form $aX^{m} = aX_{1}^{m_{1}}\cdot\ldots\cdot X_{n}^{m_{n}}$ für $a \in R \setminus \{0\}$ und $m = (m_{i})_{i \in \eb n} \in \Nn^{n}$ und $X^{m}$ (falls $a = 1$) heißt primitives Monom.
  \item Der (Total-)Grad des Monoms $aX^{m}$ für $a \in R \setminus \{0\}$ und $m = (m_{i})$ ist $|m| := \sum_{i} m_{i}$. Der (Total-)Grad von $f = \sum a_{m}X^{m}$ ist $\grad(f) = \max\{|m| : a_{m} \ne 0\}$. ($\max(\emptyset) := - \infty$)
  \item $f \in R[X_{1}, \ldots X_{n}]$ heißt homogen vom Grad $t \iff f$ ist Summe von Monomen $aX^{m}$, die alle vom Grad $|m| = t$ sind.
\end{enumerate}
\end{bez*}

\begin{bsp*}
\begin{enumerate}[(a)]
  \item $f = X_{1}^{3}X_{2}^{2}X_{3}$ ist primitiver Monom mit $\grad(f) = 11$
  \item $g = X_{1}^{3}X_{2}^{2} + X_{1}X_{2}^{4}$ ist homogen vom Grad $5$
\end{enumerate}
\end{bsp*}

\begin{lemm}
\begin{enumerate}[(a)]
  \item $\forall \s \in \Sn \ex !$ Ringhomomorphismus $\tilde \s : R[X_{1}, \ldots, X_{n}] \to R[X_{1}, \ldots, X_{n}]$ mit $\tilde s|_{R} = \id_{R}$ und $\tilde \s(X_{i}) = X_{\s(i)}$ für $i \in \eb n$
  \item $\tilde{\id} = \id_{R[X_{1}, \ldots, X_{n}]}$ (für $\id \in \Sn$ die Eins).
  \item $\forall \s, \tau \in \Sn : \tilde{\s \circ \tau} = \tilde \s \circ \tau$ Ringhomomorphismen.
\end{enumerate}
\begin{proof}[Beweis]
\begin{enumerate}[(a)]
  \item $\tilde \s$ existiert und ist eindeutig nach universeller Eigenschaft (Satz 10) für $R[X_{1}, \ldots X_{n}]$.
  \item $\alpha:= \id_{R[X_{1}, \ldots, X_{n}]}$ ist ein Ringhomomorphismus $R[X_{1}, \ldots, X_{n}] \to R[X_{1}, \ldots, X_{n}]$ mit $\alpha|_{R} = \id_{R}$ und $\alpha(X_{i}) = X_i \overset{(a)}\imp \alpha = \tilde \id$.
  \item Wende universelle Eigenschaft von $R[X_{1}, \ldots, X_{n}]$ an. Wir haben:
        \[\tilde{\s \circ \tau}|_{R} \underset{\text{Def. in (a)}}= \id_{R} = \id_{R} \circ \id_{R} = \tilde \s|_{R} \circ \tilde \tau|_{R} = \tilde \s \circ \tilde \tau |_{R}\]
        und
        \[\tilde{\s \circ \tau}(X_{i})= X_{\s \circ \tau(i)} = X_{\s(\tau(i))} = \tilde \s(X_{\tau(i)}) = \tilde \s(\tilde \tau(X_{i})) = (\tilde \s \circ \tilde \tau)(X_{i})\]
        \[ \overunderset{\text{Eindeutigkeit}}{\text{in (a)}} \imp \tilde{\s \circ \tau} = \tilde \s \circ \tilde \tau. \qedhere\]
\end{enumerate}
\end{proof}
\end{lemm}

\begin{bem*}[Übung] Ist $\alpha : R \to R$ ein Ringhomomorphismus, so ist $R^{\alpha} := \{r \in R \mid \alpha(r) = r\}$ ein Unterring von $R$.

\end{bem*}

\begin{kor}
$R[X_{1}, \ldots, X_{n}]^{S_{n}} := \{f \in R[X_{1}, \ldots, X_{n}] \mid \tilde \s (f) = f, \forall \s \in \Sn\} = \bigcap_{\s \in \Sn}R[X_{1}, \ldots, X_{n}]^{\tilde \s}$ ist ein Unterring von $R[X_{1}, \ldots, X_{n}]$.
\end{kor}

\begin{defi}[Symmetrische Polynom]
Die Elemente in $R[X_{1}, \ldots, X_{n}]^{\Sn}$ heißen symmetrische Polynome.
\end{defi}

\begin{kor} Die Abbildung
  \[\tilde \cd : \Sn \to \Aut(R[X_{1}, \ldots, X_{n}]), \s \mps \tilde \s\]
  ist wohl-definiert und ein injektiver Gruppenhomomorphismus.

 \begin{proof}[Beweis]\item
\begin{enumerate}[1)]
  \item $\tilde \cd$ wohl-definiert: Zu zeigen $\tilde \s$ ist Automorphismus (bijektiver Ringhomomorphismus). Dazu beachte
        \[\tilde \s \circ \tilde {\s^{-1}} \underset{12} = \tilde{\s \circ \s^{-1}} = \tilde \id = \id_{R[X_{1}, \ldots, X_{n}]} = \cdots = \tilde {\s^{-1}} \circ \tilde \s\]
        folglich: $\tilde \s$ ist Ringautomorphismus.
  \item Gruppenhomomorphismus: folgt aus 12(c)
  \item $\s \mps \tilde \s$ injektiv: Denn verschiedene $\s, \tau$ wirken unterschiedlich auf $\{X_{1}, \ldots, X_{n}\}$
\end{enumerate}
 \end{proof}
\end{kor}

\begin{bem*}[Ziel von diesem Abschnitt]
Explizite Beschreibung von $R[X_{1}, \ldots, X_{n}]^{\Sn}$
\end{bem*}
\section{Elementar symmetrische Polynome}%
\begin{prop*}
Zu $\s \in \Sn$ erweitern $\tilde \s$ zu $\s'$ Ringautomorphismus von $R[X_{1}, \ldots, X_{n}][X]$ durch \[\s'|_{R} = \id_{R}, \s'(X_{i}) = X_{\s(i)} \text{ und } \s'(X) := X\]
Behauptung: $g:= \prod_{i = 1}^{n}(X - X_{i}) \overset ! \in R[X_{1}, \ldots, X_{n}]^{S_{n}} \ueb = R[X_{1}, \ldots, X_{n}]^{\Sn}[X]$.
\begin{proof}[Beweis]
  $\s'(g) = \prod_{i=1}^n (\s'(X) - \s'(X_{i})) = \prod_{i=1}^n (X - X_{\s(i)}) = \prod_{i=1}^n (X-X_{i}) = g$ da $\tilde \s$ eine Bijektion auf $\{X_{1}, ..., X_{n}\}$ definiert.
\end{proof}
\end{prop*}

\begin{bem*}
  Schreibe $g$ als Polynom in $X$ mit Koeffizienten $s_{i}$ in
  \[R[X_{1}, \ldots, X_{n}] \imp g = \sum_{i=0}^n (-1)^{n-i}X^{i}s_{n-i}(X_{1},...,X_{n})\]
  \[ = X^{n} - s_{1}(X_{1}, ..., X_{n})X^{n-1}i + s_{2}(X_{1}, ..., X_{n})X^{n-2} \mp \cdots + (-1)^{n}s_{n}(X_{1}, ..., X_{n})\]
  Das definiert $s_{1}, ..., s_{n} \in R[X_{1}, \ldots, X_{n}]^{S_{n}}$
  \item Insbesondere:
  \begin{enumerate}[(i)]
    \item $s_{1}, ..., s_{n} \in R[X_{1}, \ldots, X_{n}]^{S_{n}}$
    \item $s_{i}$ ist homogen vom Grad $i$, denn $g$ ist homogen vom Grad $n \imp $ Koeffizient von $X^{n-i}$ in $g$ ist homogen vom Grad $i$.
  \end{enumerate}
\end{bem*}

\begin{ubng}
  Es gelten:
  \[s_{1} = \sum_{i=1}^n X_{i}, \quad s_{n} \prod_{i=1}^n X_{i}\]
  \[s_{i}(X_{1}, ..., X_{n}) = \sum_{1 \le j_{1} < j_{2} < \cdots < j_{i} \le n} X_{j_{1}}X_{j_{2}}\cdots X_{j_{i}}\]
  ($n=3, i = 2 \rightsquigarrow s_{2} = X_{1}X_{2} + X_{1}X_{3} + X_{2}X_{3}$)
\end{ubng}

\begin{defi}
Die Polynome $s_{1}, ..., s_{n} \in R[X_{1}, \ldots, X_{n}]^{S_{n}}$ sind die elementar symmetrischen Polynome in $X_{1}, ..., X_{n}$ (homogen vom Grad $1, 2, ..., n$) ($s_{i} = i$-tes elementar symmetrisches Polynom)
\end{defi}

\begin{satz}
  Sei $\psi: R[Y_{1}, \ldots, Y_{n}] \to R[X_{1}, \ldots, X_{n}]$ der Ringhomomorphismus \[h(Y_{1}, ..., Y_{n}) \mps h(s_{1}, ..., s_{n})\]
  Dann gilt
  \begin{enumerate}[(a)]
    \item $\psi$ ist Ringhomomorphismus mit $\psi|_{R} = \id_{R}$ und $\psi(Y_{i}) = s_{i}$ und $\bil(\psi) \subseteq R[X_{1}, ..., X_{n}]^{S_{n}}$
    \item $\psi$ definiert einen Ringisomorphismus
          \[R[Y_{1}, \ldots, Y_{n}] \to R[X_{1}, \ldots, X_{n}]^{\Sn}\]
  \end{enumerate}
\end{satz}

\begin{bsp*}
  $n = 4, f = X_{1}^{2} + X_{2}^{2} + X_{3}^{2} + X_{4}^{2}$
  \[(\underbrace{X_{1} + \cdots + X_{4}}_{s_{1}})^{2} - 2(\ubr{X_{1}X_{2} + X_{1}X_{3} + X_{2}X_{3} + X_{1}X_{4} + X_{2}X_{4} + X_{3}X_{4}}_{s_{2}})\]
  \[ = s_{1}^{2} - 2s^{2} = h(s_{1}, s_{2}), h = Y_{1}^{2} - 2Y_{2}\]
\end{bsp*}

\begin{whg*}\item
\begin{enumerate}[(a)]
  \item $R[X_{1}, \ldots, X_{n}] \subseteq R[X_{1}, \ldots, X_{n}]^{\Sn}$ symmetrische Polynome.
  \item Elementar symmetrische Polynome $s_{1}, \ldots, s_{n} \in K[X_{1}, \ldots, X_{n}]^{\Sn}$ mit \[s_{i}(X_{1}, \ldots, X_{n}) = \sum_{1 \le j_{1} < \cdots < j_{i} \le n}\prod_{1 \le k \le i}X_{j_{k}} = \sum_{1 \le j_{1} < \cdots < j_{i} \le n}X_{j_{1}}\cdot \ldots \cdot X_{j_{i}}\]
\end{enumerate}
\end{whg*}

\begin{proof}[Beweis](zu Satz 3.19)
\begin{enumerate}[{Teil} (a)]
  \item Klar
        \[\bil(\psi) = \l\{ \sum_{m \in \Nn} \ubr{a_{m}}_{\in R} \cd \ubr{s_{1}^{m_{1}}\cdot \ldots \cd s_{n}^{m_{n}}}_{\text{symm. Pol.}} \r\}\]
  \item benötigt Vorbereitungen.
\end{enumerate}
\end{proof}

\begin{bem*}
Sei $R = K$ ein Körper, $\a_{1}, \ldots, \a_{n}$ die Nullstellen von $f = X^n - \a_{1}X^{n-1} + a_{2}X^{n-2} \mp \cdots + (-1)^{n}a_{n} \in K[X]$, dann gilt $\a_{i} = s_{i}(\a_{1}, \ldots, \a_{n})$, denn: $f = (X-\a_{1})\cdot \ldots \cdot (X-\a_{n})$. (hatten $s_{i}$ erhalten als die Koeffizienten von $(-1)^iX^{n-i}$ in $(X-X_{1})\cdot \ldots \cd (X-X_{n})$)
\end{bem*}

\begin{defi}[Lex-Ordnung]\item
\begin{enumerate}[(a)]
  \item Definiere auf $\Nn^{n}$ die Relation $\le$ durch $\ell = (\ell_{1}, \ldots, \ell_{n}) \le m = (m_{1}, \ldots, m_{n}) :\iff \ell = m$ oder $\ex i \in \eb n$ mit $\ell_{1} = m_{1}, \ldots \ell_{i-1} = m_{i-1}, \ell_{i} < m_{i}$. Dies definiert eine Totalordnung auf $\Nn^{n}$, die lexikographische Ordnung. Schreibe $\ell < m$ für $\ell \le m$ und $\ell \ne m$. Für primitive Monome schreibe
        \[X^{\ell} \le X^{m} \iff \ell \le m\]
  \item Der leitgrad von $f = \sum_{m \in \Nn^{n}}a_{m}X^{m}$ ist $\mathrm{in}(f):= \max \{m\in \Nn^{n} \mid a_{m} \ne 0\} \in \Nn^{n} \cup \{-\infty\}$ (mit der Konvention $\mathrm{in}(0) = -\infty$) der Leitkoeffizient von $f \ne 0$ ist $a_{\mathrm{in}(f)}$.
\end{enumerate}
\end{defi}
\begin{bsp*}
$\mathrm{in}(\ubr{X_{1}^{3} X_{2}^{2} + X_{1}^{4}X_{3}}_{\in R[X_{1}, X_{2}, X_{3}]}) = (4, 0, 1) \in \Nn^{3}$
\end{bsp*}

\begin{prop}
  Seien $f = \sum_{\ell \in \Nn^{n}}a_{\ell} X^{\ell}, g = \sum_{m \in \Nn^{n}}b_{m} X^{m}, \ell_{0} = \mathrm{in}(f), m_{0} = \mathrm{in}(g)$. Dann:
  \begin{enumerate}[(a)]
    \item Für $m, \ell, m', \ell' \in \Nn^{n}$ gilt
          \[m \ge \ell, m' \ge \ell' \imp m+m' \ge \ell + \ell'\]
          (gilt dabei $m \ne \ell$ oder $m' \ne \ell'$, so folgt $m+m' > \ell + \ell'$)
    \item $\mathrm{in}(f \cd g) \le \ell_{0}+m_{0}$ und es gilt $\mathrm{in}(f \cd g) = \ell_{0} + m_{0}$ falls die Leitkoeffizienten $a_{\ell_{0}} \cd b_{m_{0}} \ne 0$.
    \item $\mathrm{in}(f \cd g) \le \max{\mathrm{in}(f), \mathrm{in}(g)}$ und es gilt Gleichheit falls $\mathrm{in}(f) \ne \mathrm{in}(g)$.
    \item $\mathrm{in}(s_{i}) = (\ubr{1, \ldots, 1}_{i \text{ Terme}}\ubr{0, \ldots, 0}_{n-i \text{ Terme}}) =: \xi_{i} \in \Nn^{n}$ für $i \in \eb n$.
    \item $\xi_{1}, \ldots, \xi_{n}$ sind linear unabhängig als Elemente von $\Q^{n}$, und also ist $\ph_{i}: \Nn^{n} \to \Nn^{n}, (a_{i}) \mps \sum a_{i} \xi_{i}$ injektiv und $\ph^{-1}$ ist durch die Formel (für Elemente im Bild) \[(b_{i}) \mps (b_{1}-b_{2}, b_{2}-b_{3}, \ldots, b_{n-1}-b_{n}, b_{n})\]
  \end{enumerate}
\begin{proof}[Beweis]
\begin{enumerate}[(a)]
  \item (Übung) Es genügt zu zeigen $m \ge \ell \imp m+m' \ge \ell+m'$ (mit $> \imp >$) genügt mit Induktion zu zeigen: $m \ge \ell \imp m + e_{j} \ge \ell + e_{j}$, ($e_{j} = (0, \ldots, 0,1,0, \ldots 0)$)
  \item $f \cd g = (\sum a_{\ell}X^{\ell})(\sum b_{m}X^{m}) = \sum_{\ell, m}a_{\ell}b_{m}X^{\ell+m}$ falls $a_{\ell} b_{m} \ne 0$ (nur solche Terme tragen zu $f \cd g$ bei), so folgt $\ell \le \ell_{0}$ und $m \le m_{0}$, $\ell_{0}, m_{0}$ die Leitkoeffizienten. $\underset{(a)}\imp \ell + m \ge \ell_{0} + m_{0} \imp \mathrm{in}(f \cd g) \le \ell_{0} + m_{0}$.

  Außerdem: (Koeffizient von $X^{\ell_{0}+m_{0}} = ?$) gilt $\ell + m = \ell_{0}=m_{0}$, so muss wegen (a) $\ell = \ell_{0}$ und $m = m_{0}$ gelten, falls $a_{\ell} \ne 0$ und $b_{m} \ne 0 \imp $ Koeffizient von $X^{\ell_{0}+m_{0}}$ ist $a_{\ell_{0}}\cd b_{m_{0}}$. Also $\mathrm{in}(fg) = m_{0} + \ell_{0}$, falls $a_{\ell_{0}} b_{m_{0}} \ne 0$.
  \item $f + g = \sum_{m}(a_{m} + b_{m})X^{m}$: Im Fall $a_{m} + b_{m} \ne 0$, so folgt $a_{m} \ne 0$ oder $b_{m} \ne 0 \imp m \le \ell_{0}$ oder $m \le m_{0} \imp m \le \max\{\ell_{0}, m_{0}\}$.

  Für Zusatz: Gelte o.E. $\ell_{0} < m_{0}$, dann ist der Koeffizient von $X^{m_{0}}$ gleich $a_{m_{0}} + b_{m_{0}} \ne 0$, wobei $a_{m_{0}} = 0$ wegen $m_{0} \ge \mathrm {in}(f)$, und $b_{m_{0}} \ne 0$, da $m_{0} = \mathrm{in}(f)$. Also folgt $\mathrm{in}(f+g) = \max\{\ell_{0}, m_{0}\}$.

  \item $s_{i} = \sum_{i \le j_{1} < j_{2} < \cdots < j_{i} \le n} X_{j_{1}}\cdot \ldots \cdot X_{j_{i}}$ größtes Monom (mit Koeffizient $\ne 0$) in der Summe ist $X_{1} \cdot \ldots \cdot X_{i} \imp \mathrm{in}(s_{i}) = (1, \ldots, 1, 0, \ldots 0) = (\delta_{j \le i})_{1 \le j \le n}$.
  \item (Übung) zur linearen Algebra, $\ph$ hat Darstellungsmatrix
        \[
        \begin{pmatrix}
           1 & 1 & \cdots & 1 \\
           0 & 1 & \cdots & 1 \\
           \vdots & \vdots & \ddots & \vdots \\
           0 & \cdots& \cdots & 1
        \end{pmatrix}\]
        und $\ph^{-1}$
        \[
        \begin{pmatrix}
          1 & -1 \\
            & 1 & -1 \\
            & & \ddots & \ddots \\
            & & &  1 & -1 \\
            & & & & 1
        \end{pmatrix} :
        \begin{pmatrix}
          t_{1} \\ t_{2} \\ \vdots  \\ t_{n-1} \\ t_{n}
        \end{pmatrix} \mps
        \begin{pmatrix}
          t_{1} - t_{2} \\
          t_{2} - t_{3} \\
          \vdots \\
          t_{n-1} - t_{n} \\
          t_{n}\\
        \end{pmatrix}. \qedhere\]
\end{enumerate}
\end{proof}
\end{prop}

\begin{proof}[Beweis von Satz 3.19] %TODO

\end{proof}
\begin{defi}
Die Diskriminante von $f(X) = X^{n} - a_{1}X^{n-1} + a_{2}X^{n-2} \mp \cdots + (-1)^{n}a_{n} \in R[T]$ ist $D(f):= d_{n}(a_{1}, \ldots, a_{n})$ Polynom in $n$-Variablen über $R$.
\end{defi}
\begin{bed*}
  Sei $R$ ein Körper und seien $\a_{1}, \ldots \a_{n}$ die Nullstellen von $f$, so dass $\a_{i} = s_{i}(\a_{1}, \ldots, \a_{n})$, dann folgt:
  \[D(f) = d_{n}(s_{1}(\a_{1}, \ldots, \a_{n}), \ldots, s_{n}(\a_{1}, \ldots, \a_{n}))\]
  \[= D_{n}(\a_{1}, \ldots, \a_{n}) = \prod_{1 \le i < j \le n}(\a_{i} - \a_{j})^{2}.\]
  d.h. $D(f)$ erkennt ob merhfache Nullstelle vorliegt. Jedes symmetrische Polynom in den Nullstellen von $f$ lässt sich schreiben als ein Polynom in den Koeffizienten von $f$.
\end{bed*}
\begin{whg}
  Sei $R$ ein kommutativer Ring (im Weiteren), $I \subseteq R$ ist ein Ideal von $R$, falls $RI \subseteq I, I+I \subseteq I$.
\end{whg}
\begin{nota*}
  Für $a \in R$ sei $(a) = Ra$ das Hauptideal in $R$, Erzeuger $a$. Für $a_{1}, \ldots, a_{n} \in R$ sei $(a_{1}, \ldots, a_{n}) = Ra_{1} + R{a_{2}} + \ldots + Ra_{n} \subseteq R$ Ideal.
\end{nota*}
\begin{bem*}[Übung]
Für $ I \subseteq R$ ein Ideal: $1 \in I \iff I = R$, für $S \subseteq R$ Unterring: $S = R \iff RS \subseteq S$.
\end{bem*}


\begin{prop}
  Sei $\ph: R \to R'$ ein Ringhomomorphismus, dann gelten:
  \begin{enumerate}[(i)]
    \item Ist $I' \subseteq R'$ ein Ideal, so ist $\ph^{-1}(I') \subseteq R$ ein Ideal.
    \item $\ker \ph = \ph^{-1}(\{0\}) \subseteq R$ ist ein Ideal.
    \item $\bil(\ph) = \{\ph(r) \mid r \in R\} \subseteq R'$ ist ein Unterring.
    \item Ist $\ph$ surjektiv und $I \subseteq R$ ein Ideal, so ist $\ph(I) \subseteq R'$ ein Ideal.
  \end{enumerate}
  \begin{proof}[Beweis] nur (iv)
    \begin{enumerate}[(iv)]
      \item \[\ph(I) + \ph(I) = \{\ubr{\ph(a) + \ph(b)}_{\ph(a+b)} \mid a, b \in I\} = \ph(I + I) \underset{I + I \subseteq I} \subseteq \ph(I)\]
            (benötigt nicht, dass $\ph$ surjektiv)
            \[R' \cd \ph(I) \underset{\ph \text{ surj.}} = \ph(R)\ph(I) = \{\ph(r)\ph(a) \mid r \in R, a \in I\} = \ph(RI) \underset{RI \subseteq I} \subseteq \ph(I)\]
            Also $\ph(I) \subseteq R'$ ist ein Ideal.\qedhere
    \end{enumerate}
  \end{proof}
\end{prop}

\begin{defi}[Charakteristik]
  Die Charakteristik von $R$ ist
  \[\chr(R) :=
  \begin{cases}
    0, & n \cd 1_{R} \ne 0_{R}, \forall n \in N \\
    \min\{n \in \N \mid n \cd 1_{R} = 0_{R}\}, & \ex n \in \N : n\cd 1_{R} = 0_{R}
  \end{cases}\]
\end{defi}

\begin{bsp*}
\item $\chr(\Z) = 0, \chr(\fak Z{nZ}) = n, n \in \N$
\end{bsp*}

\begin{bem*}[Übung]
\begin{enumerate}[(a)]
  \item Sei $\ord(1_{R})$ die Ordnung von $1_{R}$ in $(R, 0_{R}, +)$, dann
        \[\chr(R) =
        \begin{cases}
          \ord(1_{R}), & \ord(1_{R}) \ne \infty \\
          0, & \ord(1_{R}) = \infty \\
        \end{cases}\]
  \item Sei $\ph: \Z \to R$ der eindeutige Ringhomomorphismus
        \[\ph(1_{\Z}) := 1_{R} \imp \ph (n_{\Z}) = n \cd 1_{R}, \forall n \in \Z\]
        Dann gilt: $\chr (R)$ ist der (eindeutige) Erzeuger in $\N$ von $\ker (\ph) \subseteq \Z$ (ein Ideal) (``Grund für die Definition von $\chr (R)$'')
\end{enumerate}
\end{bem*}

\begin{prop}
  Ist $K$ ein Körper, so ist $\chr K$ Null oder eine Primzahl.
  \begin{proof}[Beweis]Annahme: $\chr K \in \N$ und ist keine Primzahl $\imp \ex n, m \in \N$ mit $n > 1, m > 1$, sodass $\chr K = n \cd m > \max\{n,m\}$

    Definition der Charakteristik gibt:
    \[n \cd m \cd 1_{K} = 0_{K} \imp \ubr{n\cd 1_{K}}_{\ne 0\ (*)} \cd \ubr{m \cd 1_{K}}_{\ne 0\ (*)} = 0\]
    $(*)$ da $n, m < n \cd m = \chr K$. Da $K$ ein Körper $\imp K$ ist nullteilerfrei $\imp n \cd 1_{K} = 0$ oder $m \cd 1_{K} = 0$. Widerspruch zu $(*)$. \qedhere
  \end{proof}
\end{prop}

\begin{bsp*}[Übung]
  Sei $R$ ein Ring mit $\chr (R) = p$ eine Primzahl, dann gelten:
  \begin{enumerate}[(a)]
    \item $\ph_{R} = R \to R, a \mps a^{p}$ ist ein Ringhomomorphismus.
    \item Es gilt $\ph_{\Fp} = \id_{\Fp}$, wobei $\Fp = \fak \Z {p\Z}$, d.h. $\forall a \in \Fp$ gilt $a^{p} = a$.
  \end{enumerate}
\end{bsp*}

\begin{whg*}
Für $I \subseteq R$ ein Ideal, hatten Faktorring $\fak RI$ und Faktorabbildung $\pi : R \to \fak RI, r \mps r+I$ (vgl. Satz 1.49)
\end{whg*}

\begin{satz}[Homomorphiesatz für Ringe]
  Sei $\ph : R \to R'$ ein Ringhomomorphismus und $I \subseteq \ker(\ph)$ ein Ideal von $R$, dann:
  \begin{enumerate}[(a)]
          \item $\ex !$ Ringhomomorphismus $\bar \ph : \fak RI \to R'$ mit $\bar \ph (r+I) = \ph(r)$, d.h. folgendes Diagramm kommutiert:
\[\begin{tikzcd}
R \arrow[d, "\pi"'] \arrow[r, "\ph"] & R' \\
\fak RI \arrow[ru, "\bar \ph"']       &
\end{tikzcd}\]
          \item Ist $I = \ker(\varphi)$, so definiert $\bar \ph$ aus (a) einen Ringisomorphismus \[\fak R{\ker(\ph)} \to \bil(\ph) \subseteq R', r+\ker(\varphi) \mps \ph(r)\]
  \end{enumerate}
  \begin{proof}[Beweis] (Übung) analog zum Beweis vom Homomorphiesatz für Gruppen (Satz 1.45).
  \end{proof}
\end{satz}
\begin{satz}[Isomorphiesatz für Ringe]
  Sei $\ph : R \to R'$ ein surjektiver Ringhomomorphismus $\l(R' \cong \fak R{\ker(\ph)}\r)$, seien $X = \{I \subseteq R \text{ Ideal} \mid  \ker(\ph) \subseteq I\}, X' = \{I' \subseteq R' \mid I' \text{ Ideal }\}$. Dann gelten:
  \begin{enumerate}[(a)]
    \item Die Abbildung $X' \to X, I' \to \ph^{-1}(I')$ ist eine Bijektion mit Umkehrabbildung $X \to X', I \mps \ph(I)$.
    \item Für $I \subseteq R'$ einIdeal und $I = \ph^{-1}(I')$ ist die Abbildung
          \[\fak RI \to \fak{R'}{I'}, r+I \mps \ph(r) + I'\]
          ein Ringisomorphismus.
  \end{enumerate}
  \begin{proof}[Beweis]
  (Übung) analog zum Beweis vom 2. Isomorphiesatz für Gruppen (Satz 1.51).
  \end{proof}
\end{satz}

\begin{nota*}
Für $I, J \subseteq R$ sei $I \cd J = \{\sum_{i}a_{i}b_{i} \mid a_{i} \in I, b_{i} \in J\}$, d.h. (Übung) $I \cd J$ ist das kleinste Ideal in $R$, das $\{a \cd b \mid a \in I, b \in J\}$ enthält.
\end{nota*}

\begin{satz}[Chinesischer Restsatz]
  Seien $I_{1}, \ldots, I_{t} \subseteq R$ Ideale, die ``paarweise Koprim'' sind, d.h. $I_{i} + I_{j} = R$ für $i \ne j \in \eb t$. Dann gelten:
  \begin{enumerate}[(a)]
    \item $I_{i}$ und $\prod_{j \ne i \in \eb t}I_{j}$ sind Koprim.
    \item $I_{1}\cd \ldots \cd I_{t} = \bigcap_{i \in \eb t} I_{i}$.
    \item Die Abbildung
          \[\fak R{\textstyle \prod_{i \in \eb t}I_{i}} = \fak R{I_{1}\cd \ldots \cd I_{t}} \overset \cong \longrightarrow \bigtimes_{i \in \eb t} \fak R{I_{i}} = \fak R{I_{1}} \tm \cdots \tm \fak R{I_{t}}\]
          \[r+I_{1}\cd \ldots I_{t} \mps (r+I_{1}, \ldots r+I_{t})\]
          ist wohl-definiert und ein Ringisomorphismus. Also gilt
          \[\fak R{\textstyle \prod_{i \in \eb t}I_{i}} \cong\bigtimes_{i \in \eb t} \fak R{I_{i}}\]
  \end{enumerate}

  \begin{proof}[Beweis]
    In der LA2 für $R$ ein Hauptidealring, allgemein: siehe Jantzen-Schwermer, Satz III.3.10
  \end{proof}
\end{satz}


\section{Ringe von Brüchen/Lokalisierung}%
\begin{defi}
Eine Teilmenge $S \subseteq R$ heißt multiplikativ abgeschlossen $\iff S$ ist ein Untermonoid von $(R, 1, \cd)$.
\end{defi}
\begin{bsp*}
  \begin{enumerate}[(i)]
    \item $S = \Z \setminus \{0\} \subseteq \Z$ ist multiplikativ abgeschlossen.
    \item $S^{p} = \Z \setminus p \Z \subseteq \Z$ ist multiplikativ abgeschlossen.
    \item $S_{p} = \{p^{n} \mid n \in \Nn\} \subseteq \Z$ ist multiplikativ abgeschlossen.
  \end{enumerate}
  \item Es gilt $S = S^{P} \cd S_{p}$
\end{bsp*}

\begin{defi}
  Definiere eine Äquivalenzrelation auf $R \tm S$ ($S \subseteq R$ multiplikativ abgeschlossen) durch \[(r,s) \sim(r',s') :\iff \ex t \in S : t (rs' - r's)\]
  Denn:
  \item $\sim$ reflexiv: $(r,s) \sim r,s$, da $1 \cd (rs-rs) = 0$.
  \item $\sim$ symmetrisch: Gelte $(r,s) \sim (r', s')$, d.h. $\ex t \in S : t(rs' - r's) = 0 \imp t(r's - rs') = 0 \imp (r',s') \sim (r,s)$.
  \item $\sim$ transitiv: Gelte $(r,s) \sim (r',s')$ und $(r',s') \sim (r'',s'')$, d.h. $\ex t, t' \in S : t(rs' - r's) = 0$ und $t'(r's'' - r''s') = 0$. Gemeinsamer Nenner $tt'ss's''$
  \[\imp tt's''(rs'-r's) = 0, tt's(r's''-r''s) = 0\]
  \[\imp tt's''rs' - tt'sr{''}s' = 0 = tt's'(rs'' - r''s) \imp (r,s) \sim (r'',s'')\]
  Schreibe: $\frac rs$ für die Äquivalenzklasse von $(r,s)$ und $S^{-1}R$ für $\fak {R \tm S}\sim$.
  \item Beachte: $\frac rs = \frac{r'}{s'} \iff \ex t \in S : \frac{ts'r}{tss'} = \frac{tsr'}{tss'}$ gilt $ts'r = tsr'$, beachte zudem $\frac rs \ueb = \frac {tr}{ts}$, für $t \in S$.
\end{defi}

\begin{satz}
  Sei $S \subseteq R$ multiplikativ abgeschlossen, dann:
  \begin{enumerate}[(a)]
    \item Die Verknüpfungen $+, \cd$ auf $S^{-1}R$ definiert durch
          \[\frac rs + \frac {r'}{s'} = \frac{rs'+r's}{ss'}, \quad \frac rs \cd \frac {r'}{s'} = \frac{rr'}{ss'}\]
          sind wohl-definiert.
    \item $S^{-1}R = (S^{-1}R, \frac 01, \frac 11, +, \cd)$ ist ein kommutativer Ring.
    \item Die Lokalisierung von $R$ an $S$
          \[\ph : R \to S^{-1}R, r \mps \frac r1\]
          ist ein Ringhomomorphismus. (Klar aus dem Definition von $+$ und $\cd$)
    \item (Universelle Eigenschaft) Ist $\psi : R \to R'$ ein Ringhomomorphismus, sodass $\psi(S) \le (R')\en$, so existiert ein eindeutiger Ringhomomorphismus $\hat \psi : S^{-1}R \to R'$ mit $\hat \psi |_{R} = \psi$, nämlich $\hat \psi (\frac rs) = \psi(r) \cd \psi(s)^{-1}$.
  \end{enumerate}
\end{satz}
\begin{bsp*}$(\Z \setminus \{0\})^{-1}\Z = \Q, \Z^{-1}\Z = 0$-Ring.
\end{bsp*}
\begin{proof}[Beweis]\item
\begin{enumerate}[(a)]
  \item $+$ und $\cd$ sind wohldefiniert:
        Gelte $\frac rs = \frac ab$ und $\frac {r'}{s'} = \frac {a'}{b'}$ mit $r, r', a, a' \in R, s, s', b, b' \in S$, zu zeigen ist: \[\frac{rs' + r's}{ss'} = \frac{ab' + a'b}{bb'}\]
        Voraussetzung: $\ex t, t' \in S : t(rb - as) = 0, t'(r'b' - a's') = 0$. Gemeinsamer Nenner: $ss'bb'tt'$, also
        \[tt'b's'(rb - as) = 0, \quad tt'sb(r's'-a's) = 0\]
        \[\imp tt'b's'rb - tt'b's'as + tt'sbr'b' - tt'sba's' = 0\]
        \[ = tt'b'b(rs'+r's) - tt'ss'(ab'-a'b) \imp \frac{rs' + r's}{ss'} = \frac{ab' + a'b}{bb'}\]
  \item - (d) Siehe Jantzen Schwermer III.4.2 oder Übung.
\end{enumerate}
\end{proof}

\begin{defi}[Nullteiler]
\begin{enumerate}[(a)]
  \item $x \in R$ heißt Nullteiler $\iff \ex y \in R \setminus \{0\}$ mit $xy = 0$
  \item $R$ heißt Integritätsbereich (IB) $\iff 0_{R} \ne 1_{R}$ und $0_{R}$ ist der einzige Nullteiler.
\end{enumerate}
\end{defi}

\begin{bem}
  $R$ ist Itdentitätsbereich $\iff$ man darf in $R$ kürzen und $0_{R} \ne 1_{R}$
  \[\ueb \iff \forall a, b, c \in R : a \ne 0 : a\cd b = a\cd c \imp b = c\]
  Denn $ab = ac \iff a(b-c) = 0$
\end{bem}
\begin{bsp*}
\begin{enumerate}[(i)]
  \item Jeder Körper ist ein Integritätsbereich.
  \item $\Z, K[X]$ sind Integritätsbereich.
  \item Jeder Unterring eines Körpers ist ein Integritätsbereich.
  \item Jeder Unterring eines Integritätsbereichs ist ein Integritätsbereich.
\end{enumerate}
\end{bsp*}

\begin{lemm}
  Sei $S \subseteq R$ multiplikativ abgeschlossen, dann gilt: enthält $S$ keine Nullteiler, so ist \[\ph : R \hookrightarrow S^{-1}R, r \mps \frac r1\] injektiv.
  \begin{proof}[Beweis]
    Für $r \in R: \ph(r) = 0 \iff \frac r1 = \frac 01 \iff \ex t \in S : t(r \cd 1 - 0\cd 1) = 0 = tr$, da $S$ nullteilerfrei $\iff r = 0.$
  \end{proof}
\end{lemm}

\begin{kor}
  Sei $R$ ein Integritätsbereich, dann:
  \begin{enumerate}[(a)]
    \item $S = R \setminus \{0\}$ multiplikativ abgeschlossen.
    \item $S^{-1}R$ ist ein Körper.
    \item $R \to S^{-1}R$ ist injektiv (also ist $R$ Unterring des Körpers $S^{-1}R$)
  \end{enumerate}
\begin{proof}[Beweis]
  \begin{enumerate}[(a)]
    \item Klar, $a,b \ne 0 \imp a\cd b \ne 0$ ($a,b$ keine Nullteiler)
    \item Sei $\frac rs \in S^{-1}R \setminus \{\frac 01\}$, Behauptung: $r \ne 0$ (also $r \in S$) $\imp \frac sr$ ist Inverses von $\frac rs$
          Beweis der Behauptung: Angenommen $r = 0 \imp \frac 0s \ne \frac 01$, Widerspruch, da $\frac 01 = \frac 01$ ($1 \cd(0\cd1 - 0 \cd s) = 0$)
    \item Folgt aus Lemma 3.35.
  \end{enumerate}

\end{proof}
\end{kor}

\begin{defi}[Quotientenkörper]
$S^{-1}R = \Quot(R)$ heißt Quotientenkörper von $R$.
\end{defi}

\begin{bem}
Jeder Integritätsbereich ist Unterring eines Körpers (seinem Quotientenkörpers).
\end{bem}
\section{Spezielle Ideale}%
\begin{defi}
  Sei $I \subseteq R$ ein echtes Ideal (d.h. $I \subsetneq R$), dann
  \begin{enumerate}[(a)]
    \item $I$ ist \textbf{Primideal} $\iff \forall a, b \in R$ gilt:
          \[a \cd b \in I \imp a \in I \vee b \in I\]
    \item $I$ heißt \textbf{maximales Ideal} $\iff \forall J \subsetneq R$ Ideale mit $I \subseteq J$ gilt $I = J$
  \end{enumerate}
\end{defi}

\begin{prop}
  Seien $P, M \subseteq R$ Ideale, dann:
  \begin{enumerate}[(a)]
    \item $P$ ist ein Primideal $\iff \fak RP$ ist ein Integritätsbereich.
    \item $R$ ist ein Körper $\iff \{0\}$ und $R$ sind die einzigen Ideale von $R$ und $0_{R} \ne 1_{R}$.
    \item $M$ ist ein maximales Ideal $\iff \fak RM$ ist ein Körper.
    \item Jedes maximale Ideal ist ein Primideal.
  \end{enumerate}

  \begin{proof}[Beweis] \item
    \begin{enumerate}[(a)]
      \item $P$ Primideal $\imp P \subsetneq R$ und $\forall a, b \in R : a \cd b \in P \imp a \in P \vee b \in P \imp \fak RP \ne 0$-Ring und $\forall a, b \in R :$
      \[(a+P)(b + P) \subseteq P \imp a+P = P \vee b+P = P\]
      $\imp \fak RP \ne 0$-Ring und $\forall \bar a, \bar b \in \fak RP :$
      \[\bar a \cd \bar b = 0 \imp \bar a = 0 \vee \bar b = 0\]
      $\imp \fak RP$ ist ein Integritätsbereich. Man kann auch ''rückwarts laufen'' $\imp$ die Äquivalenz in (a).
      \item Übung.
      \item Folgt aus (b) und dem Isomorphiesatz für Ringe. (der postuliert Bijektion : $\{$Ideale $J \subseteq R \mid M \subseteq J \subseteq R\}$ und $\{$Ideale $\bar J \subseteq \fak RM \mid \{\bar 0\} \subseteq \bar J \subseteq \fak RM\}$).
      \item Folgt aus (c) und (a), da Körper Integritätsbereiche sind. \qedhere
    \end{enumerate}
\end{proof}
\end{prop}
\begin{bsp}
  \begin{enumerate}[(a)]
    \item Ist $R$ ein Integritätsbereich und kein Körper, so ist $\{0\}$ ein Primideal, aber nicht maximal.
    \item In $R = K[X, Y]$ sind $\{0\} \subsetneq X \subsetneq (X, Y) \subsetneq K$ Primideal.
   \end{enumerate}
 \end{bsp}

 \begin{whg*}[Grundlagen]
   Eine Relation $\le$ auf einer Menge $M$ heißt \textbf[Halbordnung] $\iff \le$ ist reflexiv, transitiv und antisymmetrisch. ($\le$ antisymmetrisch bedeutet: $x \le y \wedge y \le x \imp x = y$). Eine Halbordnung heißt \textbf{Totalordnung} $\iff \forall x, y \in M : x \le y \vee y \le x$.
 \end{whg*}

 \begin{defi}
   Sei $(M \le)$ eine halbgeordnete Menge.
   \begin{enumerate}[(a)]
     \item Eine Teilmenge $N \subseteq M$ heißt \textbf{Kette} $\iff (N, \le|_{N \times N})$ ist eine Totalordnung.
     \item Eine Teilmenge $P \subseteq M$ besitzt eine obere Schranke (in $M$) $\iff \ex m \in M$, sodass $\forall p \in P : p \le m$.
     \item $m \in M$ heißt maximales Element $\iff \nexists m' \in M : m' > m$
   \end{enumerate}
 \end{defi}
 \begin{bsp*}
\begin{enumerate}[(a)]
  \item Ist $M$ eine beliebige Menge und $\mathcal P(M)$ eine Potenzmenge, so ist $(\mathcal P(M), \subseteq)$ eine Halbordnung. $M$ ist obere Schranke für jede Teilmenge von $\mathcal P(M)$.
  \item $\Nn$ besitzt keine obere Schranke.
\end{enumerate}
 \end{bsp*}
 Wir betrachten nun die folgenden beiden Axiome der axiomatischen Mengenlehre:
\begin{axiom*}[Zorn's Lemma]
  Sei $(M, \le)$ eine Halbordnung ($M \ne \emptyset$). Besitzt jede Kette in $M$ eine obere Schranke, so besitzt $M$ ein maximales Element.
  Dies nehmen wir als Axiom an.
\end{axiom*}
\begin{axiom*}[Auswahlaxiom]
  Ist $I$ eine Menge und $(A_{i})_{i \in I}$ eine Familie von nichtleeren Mengen (indiziert mit $I$), so $\ex$ Funktion $f: I \to \bigcup_{i \in I}A_{i}$, mit $f(i) \in A_{i}$.
\end{axiom*}

\begin{satz*}[Halmos, Naive Set Theory, 62-65]
Zorn's Lemma $(\forall (M, \le))$ und das Auswahlaxiom $(\forall I, \forall (A_{i})_{i \in I})$ sind äquivalent.
\end{satz*}
\begin{satz}
  Sei $I \subseteq R$ ein echtes Ideal, dann $\ex$ maximales Ideal $M \subsetneq R$ mit $I \subseteq M$. Insbesondere hat $R$ maximale Ideale (Satz für $I = (0)$)
  \begin{proof}[Beweis]
    Sei $X$ die Menge aller Ideale $J \subsetneq R$ mit $I \subseteq J$. Wegen $I \in X$ gilt $X \ne \emptyset$. $(X, \subseteq)$ ist halbgeordnete Menge.
    \item Behauptung: Zorn's Lemma ist anwendbar.
    \item Denn: Sei $X_{0} \subseteq X$ eine Kette (o.E. $X_{0} \ne 0$). Definiere $J_{\infty} := \bigcup_{J \in X_{0}}J$.
    \item Zu zeigen: $J_{\infty} \in X \imp J_{\infty}$ ist obere Schranke von $X_{0}$. Klar ist $I \subseteq J_{\infty}$ und $1 \notin J_{\infty}$ ($\imp J_{\infty} \subsetneq R$)
    \item Zu zeigen: $J_{\infty}$ ist ein Ideal. Seien $a, b \in J_{\infty}$ und $r \in R$. Nach Definition von $J_{\infty} \ex J, J' \in X_{0}$ mit $a \in J, b \in J'$. Nun ist aber $X_{0}$ totalgeordnet unter $\subseteq$. D.h. $J \subseteq J'$ oder $J' \subseteq J$. o.E. $J' \subseteq J \imp a, b \in J \imp a + b, r \cd a \in J \imp a+b , ra \in J_{\infty}$, da $J \subseteq J_{\infty}$, damit ist die Behauptung gezeigt.
    \item Sei $M$ ein maximales Element von $X$. Dann ist $M$ ein maximales Ideal von $R$ (mit $I \subseteq M$) sonst $\ex J'' \subsetneq R$ ideal mit $M \subsetneq J''$, Widerspruch zu $M$ maximales Element in $X$.
  \end{proof}
\end{satz}
\begin{ubng*} (Plenarübung 3)
Zorn's Lemma $\imp$ jeder Vektorraum hat eine Basis.
\end{ubng*}
\section{Teilbarkeit in Integritätsbereichen}%
\begin{defi}
  Sei bis auf Weiteres $R$ ein Integritätsbereich, $a, b \in R$.
  \begin{enumerate}[(a)]
    \item $a$ ist Teiler von $b$ ($a$ teilt $b$, $a \mid b$)$:\iff \ex c \in R : a \cd c = b$.
    \item $a, b$ sind assoziiert $(a \simeq b):\iff \ex c \in R\en : a \cd c = b$.
    \item $a$ heißt irreduzibel (bzw. unzerlegbar) $:\iff a \in R \setminus (R \en \cup \{0\})$ und $\forall c \in R : c \mid a \imp c \simeq a \vee c \simeq 1$.
    \item $a$ heißt Primelement $:\iff a \in R \setminus (R\en \cup \{0\})$ und $\forall b, c \in R : a \mid bc \imp a \mid b \vee a \mid c$.
  \end{enumerate}
\end{defi}

\begin{bem}[Übung]\item
\begin{enumerate}[(a)]
  \item $a \mid b \iff (b) \subseteq (a)$.
  \item $a \simeq b \iff a \mid b \wedge b \mid a \iff (a) = (b)$ und $\simeq$ ist eine Äquivalenzrelation. (``denn:'' $(R\en, 1, \cd)$ ist eine Gruppe).
  \item $Ra = (a) = (0) \iff a = 0$. $Ra = (1) \iff a \in R\en \iff a \simeq 1$
  \item $a$ Primelement $\iff (a)$ ist ein Primideal.
  \item $a$ ist irreduzibel $\iff (0) \subsetneq (a) \subsetneq R$ und $\nexists b \in R : (a) \subsetneq (b) \subsetneq R$. (d.h. $(a)$ ist maximal unter den Hauptidealen) und $a \in R \setminus (R\en \cup \{0\})$ ist reduzibel $\iff \ex b, c \in R \setminus (R\en \cup \{0\}): a = b \cd c \iff \ex b, c \in R : a = b \cd c$ und $(a) \subsetneq (b), (c) \subsetneq R$.
  \item $a$ Primelement $\imp a$ ist irreduzibel.
\end{enumerate}
\end{bem}
\begin{proof}[Beweis zu (f)]
  Annahme: $a$ ist reduzibel. Nach letzte Formulierung von (c) $\ex b, c \in R : a = b \cd c \wedge (a) \subsetneq (b), (c) \subsetneq R \imp$ in $\fak R{(a)}$ gilt $\bar 0 = \bar a = \bar b \cd \bar c$ und $\bar 0 \ne \bar b, \bar c \imp \fak R {(a)}$ kein Integritätsbereich $\imp (a)$ kein Primideal, Widerspruch zu (d), da $a$ Primelement.
\end{proof}

\begin{defi}[\textbf{Hauptidealring}]
Ein Integritätsbereich heißt \textbf{Hauptidealring} (HI-Ring), wenn jedes Ideal ein Hauptideal ist.
\end{defi}
\begin{defi}
  Ein Integritätsbereich $R$ heißt \textbf{euklidischer Ring}, wenn $\ex \lambda : R \setminus \{0\} \to \Nn$, sodass gilt:
  \[\forall a, b \in R \ex q, r \in R : a = qb + r, (r = 0 \vee \lambda (r) < \lambda(b)) \quad (*)\]
\end{defi}
\begin{bez*}\item
\begin{enumerate}[(a)]
  \item $(*)$ heißt Division mit Rest.
  \item $\lambda$ heißt euklidische Funktion.
\end{enumerate}
\end{bez*}

\begin{bsp}
\begin{enumerate}[(a)]
  \item $\Z$ ist ein euklidischer Ring mit \[\lambda : | \cd | : \Z \setminus \{0\} \to \Nn, n \mps |n|\]
  \item $K[X]$ ist ein euklidischer Ring mit \[\lambda = \grad : K[X] \setminus \{0\} \to \Nn, f \mps \grad f\]
\end{enumerate}
\end{bsp}

\begin{prop}
  Ist $R$ ein euklidischer Ringe $\imp R$ ist ein Hauptidealring.
  \begin{proof}[Beweis]
    Sei $I \ne \{0\}$ ein Ideal. Sei $a \in I \setminus \{0\}$ ein Element, sodass \[\lambda(a) = \min \{\lambda(b) \mid b \in I \setminus \{0\}\} \subsetneq \Nn \]
    \item Behauptung: $I = (a)$ ($I \supseteq$ klar, da $a \in I$).
    \item Dazu: Sei $b \in I$ beliebig. Wende Division mit Rest an
    \[b = qa + r, \quad r = 0 \vee \lambda(r) < \lambda(a)\]
    \[\imp r = b - qa \in I - I \subseteq I \imp b = qa \in (a). \qedhere\]
  \end{proof}
\end{prop}

\begin{prop}
  Sei $R$ ein Hauptidealring und $a \in R \setminus (R\en \cup \{0\})$, dann sind äquivalent:
  \begin{enumerate}[(i)]
    \item $a$ irreduzibel.
    \item $a$ Primelement.
    \item $(a)$ ist Primideal.
    \item $(a)$ ist maximales Ideal.
    \item $\fak R{(a)}$ ist ein Körper.
  \end{enumerate}
  \begin{proof}[Beweis]
    \begin{itemize}
\item (iv) $\iff$ (v) folgt aus Bemerkung 40(c)
\item (i) $\imp$ (iv) folgt aus Bemerkung 45(e) ($a$ irreduzibel $\imp (a)$ ist maximal unter Hauptidealen)
\item (iv) $\imp$ (iii) folgt aus Bemerkung 40(d)
\item (iii) $\imp$ (ii) folgt aus Bemerkung 45(d)
\item (ii) $\imp$ (i) folgt aus Bemerkung 45(f)
    \end{itemize}
\end{proof}
\end{prop}

\begin{defi}
  Seien $a, b \in R$
  \begin{enumerate}[(a)]
    \item $d \in R$ heißt ggT (\textbf{größter gemeinsamer Teiler}) von $a, b$, wenn $d \mid a, d \mid b$ und $\forall c \in R : c \mid a, c \mid b \imp c \mid d$.
    \item $d \in R$ heißt kgV (\textbf{kleinstes gemeinsames Vielfaches}) von $a, b$, wenn $a \mid d, b \mid d$ und $\forall c \in R: a \mid c, b \mid c \imp d \mid c$.
    \item $a, b$ heißen \textbf{teilerfremd}, wenn $\ggt(a, b) \simeq 1$.
  \end{enumerate}
\end{defi}

\begin{bem*}[Übung]
ggT und kgV sind (sofern sie existieren) eindeutig bis auf Assoziiertheit.
\end{bem*}
\begin{nota*}
$d \simeq \ggt(a, b)$ bedeutet $d$ ist ggT von $a, b$.
$d \simeq \kgv(a, b)$ bedeutet $d$ ist kgV von $a, b$.
\end{nota*}

\begin{kon}
  Sei $K[X]_{+} = \{f \in K[X] \setminus \{0\} \mid f \text{ normiert}\} \cup \{0\}$. In
  $\l\{ \begin{matrix}
          \Z \\ K[X]
        \end{matrix} \r\}$ ist jedes Element zu einem eindeutigen Element in
$\l\{ \begin{matrix}
          \Nn \\ K[X]_{+}
        \end{matrix} \r\}$ assoziiert. Für $f, g \in \l\{ \begin{matrix}
          \Z \\ K[X]
        \end{matrix} \r\}$ schreibe $d = \ggt(f, g)$ bzw. $d = \kgv(f, g)$ $\iff d \simeq \ggt(f,g)$ bzw. $d \simeq \kgv(f,g)$ und $d \in \l\{ \begin{matrix}
          \Nn \\ K[X]_{+}
        \end{matrix} \r\}$.
\end{kon}

\begin{satz} Sei $R$ ein Hauptidealring, dann gelten für $a, b, c \in R$:
  \begin{enumerate}[(a)]
    \item $(i) \quad c \simeq \ggt(a, b) \iff (ii) \quad (c) = (a) + (b)$
    \item $(i) \quad c = \kgv(a, b) \iff (ii) \quad (c) = (a) \cap (b)$
    \item $\ggt(a, b)$ und $\kgv(a, b)$ existieren $\forall a, b \in R$
    \item Es sind Äquivalent: $(i) \quad \ggt(a,b) \simeq 1$ ($a,b$ teilerfremd) $\iff (ii) \quad (a)+(b) = R \iff (iii) \quad \ex \a, \beta \in R : \a a + \beta b = 1$
  \end{enumerate}
\begin{proof}[Beweis]
(Übung)
\end{proof}
\end{satz}

\begin{bem*}
\begin{enumerate}[(a)]
  \item Hauptidealringe haben die Bezout-Eigenschaft, d.h. zu $a, b \in R \ex \alpha, \beta \in R : \a \a + \beta b \simeq \ggt(a,b)$.
  \item In euklidischen Ringen kann man den ggT mit dem euklidischen Algorithmus berechnen und $\alpha, \beta$ wie in (a) mit dem erweiterten euklidischen Algorithmus.
\end{enumerate}
\end{bem*}


\begin{satzdef}
  Für einen Integritätsbereich $R$ sind äquivalent:
  \begin{enumerate}[(i)]
    \item $\forall a \in R \setminus (R\en \cup \{0\}) \ex t \in \N \ex$ Primelemente $p_{1}, \ldots, p_{t} \in R$ mit $a \simeq p_{1}\cdot  \ldots \cd p_{t}$
    \item $\forall a \in R \setminus (R\en \cup \{0\}) \ex t \in \N \ex$ irreduzible Elemente $p_{1}, \ldots, p_{t} \in R$ mit $a \simeq p_{1}\cdot  \ldots \cd p_{t}$ und diese Darstellung ist eindeutig bis auf Indexpermutation und Assoziiertheit, d.h. gilt $a \simeq q_{1}\cd \ldots \cd q_{s}$ mit $q_{1}, \ldots, q_{s}$ irred., so gilt $s = t$ und nach Indexpermutation $q_{i} \simeq p_{i}$ für $i = 1, \ldots, t$.
  \end{enumerate}
  Ein Integritätsbereich, der (i) und (ii) erfüllt heißt faktorieller Ring (oder EPZ-Ring: Ring mit eindeutiger Primfaktorzerlgung)
  \begin{bem*}\item
\begin{enumerate}[(a)]
  \item (i) $\imp$ irred. Elemente in $R$ sind Primelemente. Denn: Sei $q$ irred. in $R$, schreibe Faktorisierung wie in (i) für $q$, d.h. $q \simeq p_{1}\cd \ldots \cd p_{t} \underset{q \text{ irred.}}\imp t = 1$ also $q \simeq p_{1}$. Primelement.
  \item In (b) ist $R$ beliebiger Integritätsbereich (so zwigt man mit Induktion) Ist $p$ ein Primelement in $R$ und ein Teiler von $a_{1}\cd \ldots \cd a_{t}$ (mit $a_{i} \in R$), so $\ex i \in \eb t$ mit $p | a_{i}$.
  \item Für $R$ wie in 54 ist $R$ faktoriell, so ist die Länge $r$ einer Primfaktorzerlgung $r \simeq p_{1} \cld p_{t}$ ($p_{i}$ prim) von $r \in R \setminus \{0\}$ unabhängig von der Faktorisierung (vgl (ii)). Schreibe $r(r) \in \Nn$ für diese Länge.
\end{enumerate}
\end{bem*}
\begin{proof}[Beweis](von Satz 54)
  \item (i) $\imp$ (ii): Existenz der Faktorisierung in (ii) ist klar nach (i), da Primelemente irreduzibel sind.
  \item Eindeutigkeit: Gelte $p_{1} \cld p_{t} \overset{(*)}\simeq q_{1} \cld q_{t}, s \in \N, p_{i}$ prim, $q_{i}$ irred. Zeige mit Induktion über $t$: $s = t$ und nach Indexpermutation $q_{i} \simeq p_{i}$
  \begin{itemize}
    \item $t = 1$: $p_{1} \simeq q_{1}\cld q_{s} \imp s = 1$ ($p_{1}$ prim, also irred.)
    \item $t-1 \to t$: $(*)$ und Bemerkung (b) $\imp \ex j \in \eb s$ mit $p_{t} \mid q_{j}$, o.E. $j = s$ (Umindizieren) und also $p_{t} \simeq q_{s}$ ($q_{s}$ irreduzibel). teile beide Seiten durch $p_{t}$
          \[p_{1} \cld p_{t-1} \simeq q_{1} \cld q_{s-1} \ubr{u}_{\in R\en} \simeq q_{1} \cld q_{s-1}\]
          Nun: wende Induktionsvoraussetzung an. $\implies s-1 = t-1$ (also $s=t$) und nach Indexpermutation: $q_{i} \simeq p_{i}$ für $i = \eb{t-1}$
  \end{itemize}
  \item (ii) $\implies$ (i): Zeige irred. Elemente in $R$ sind Primelemente. Sei also $q \in R$ irreduzibel. Seien weiter $a, b \in R$, sodass $q \mid ab$.
  \item Zu zeigen: $q \mid a$ oder $q \mid b$: o.E. $a, b \ne 0$ (sonst $q \mid a \vee q \mid b$), o.E. $a, b \notin R\en$, ist z.B. $a \in R\en$, so folgt aus $q \mid ab$ direkt $q \mid b$. Sei $c \in R$ mit $qc = ab$. Schreibe $c , a, b$ einer Faktorisierung wie in (ii) geg.
  \[a \simeq p_{1} \cld p_{t}, \quad b \simeq q_{1} \cld q_{s},\quad c \simeq r_{1}\cld r_{u}\]
  ($p_{i}, q_{j}, r_{k}$ irred. $t, s \in \N, u \in \Nn$)
  \[qr_{1}\cld r_{u} \simeq p_{1}\cld p_{t}\cd q_{1}\cld q_{s}\]
  Wende Eindeutigkeitsaussage von (ii), um zu folgen:
  \item $q \simeq p_{i}, i \in \eb t$ oder $q \simeq q_{j}, j \in \eb s \imp q \mid a$ oder $q \mid b.\qedhere$
\end{proof}
\end{satzdef}

\begin{kor}[Übung]
  Sei $R$ ein faktorieller Ring und sei $\P$ ein Repräsentantensystem der Primelemente von $R$ modulo Assoziiertheit, dann gelten:
  \begin{enumerate}[(a)]
    \item $\forall p \in \P$ ist die Abbildung $v_{p}: R \setminus \{0\} \to \Nn, r \mps \max\{n \in \Nn : p^{n} \mid r\}$ wohl-definiert (genauer $v_{p}(r) \le t(r)$) und $v_{p}$ ist ein Monoidhomomorphismus (für $(R, 1, \cd)$)
    \item $\forall r \in R \setminus \{0\}$ gilt $\#\{p \in \P : p \mid r\} \le t(r) < \infty$
    \item $\forall r \in R \setminus \{0\} \ex ! u \in R\en :$ \[r = u \prod_{p \in \P}p^{v_{p}(r)} = u\prod_{p \in P, v_{p} > 0}p^{v_{p}(r)}\] (Primfaktorzerlgung von $r$)
    \item Für $r, s \in R \setminus \{0\}$ gilt: \[r \mid s \iff \bigforall_{p \in \P} r_{p}(r) \le v_{p}(s)\]
    \item Für $r, s \in R \setminus \{0\}$ gelten:
          \[\ggt(r, s) \simeq \prod_{p \in \P}p^{\min(v_{p}(r), v_{p}(s))},\quad \kgv(r, s) \simeq \prod_{p \in \P}p^{\max(v_{p}(r), v_{p}(s))}\]
          ($\ggt$ und $\kgv$ existieren also in faktoriellen Ringen)
    \item Sei $K = \Quot(R)$, dann $\ex!$ Fortsetzung $v_{p} : K\en \to \Z$ (ein Gruppenhomomorphismus, der den Monoidhomomorphismus $v_{p} : R \setminus \{0\} \to \Nn$ fortsetzt) und $\forall r \in K\en \ex ! u \in R\en:$
          \[r = u \prod_{p \in \P}p^{v_{p}(r)}\]
          (dabei $\#\{p : v_{p}(r) \ne 0\}$ endlich.)
  \end{enumerate}
\end{kor}

\begin{ubng}[vgl. LA2] Für einen (beliebigen kommutativen) Ring $R$ sind äquivalent:
  \begin{enumerate}[(a)]
    \item Jede aufsteigende Kette von Idealen \[I_{0} \subseteq I_{1} \subseteq I_{2} \subseteq \cdots \subseteq I_{n} \subseteq \cdots \subseteq R\]
          ($\Nn$ indiziert) wird stationär, d.h. $\ex n_{0} : \forall n \in n_{0} : I_{n} = I_{n_{0}}$
    \item Jede nichtleere Teilmenge $\mathscr M$ von Idealen enthält ein bzgl. der Inklusion maximales Element.
    \item Jedes Ideal $I \subseteq R$ ist endlich erzeugt, d.h. $\ex n \in \N \ex a_{1}, \ldots, a_{n} \in R$ mit $I = (a_{1}, \ldots, a_{n})$
  \end{enumerate}
\begin{defi*}[Noetherscher Ring]
Gelten (a)-(c) für $R$, so heißt $R$ \textbf{noethersch}
\end{defi*}
\end{ubng}
\begin{bem*} $R$ noethersch $\underset{\text{ohne Zorn's Lemma}}\imp \forall$ echte Ideal $I \subsetneq R \ex$ maximales Ideal mit $I \subseteq \mathscr M$ denn ein solches findet sich in $\mathscr M = \{J \subsetneq R \mid I \subseteq J\}$
\end{bem*}

\begin{kor}
  Ist $R$ ein HI-Ring, so gelten:
  \begin{enumerate}[(a)]
    \item $R$ ist noethersch.
    \item $\forall r \in R \setminus (R\en \cup \{0\}) \ex t \in \N \ex$ Primelemente $p_{1}, \ldots, p_{t} \in R$, sodass $r \simeq p_{1} \cld p_{t}$
    \item $R$ ist faktoriell.
  \end{enumerate}
\begin{proof}[Beweis]
  \begin{enumerate}[(a)]
    \item folgt aus 56, da jedes Ideal von $R$ ein Hauptideal ist.
    \item folgt aus (b).
    \item Bei HI-Ringen, wissen schon Primelemente sind irred. und umgekehrt. (Prop. 50).
          Zu zeigen: Sei $r \in R \setminus (R\en \cup \{0\})$, dann existiert eine Faktorisierung wie in (b). Definiere: \[\mathscr M_{a}:=\{(b) \subseteq R \mid \ex t \in \Nn \ex \ubr{q_{1}, \ldots, q_{t}}_{\text{irred.}} \in R : bq_{1}\cld q_{t} \simeq a\}\]
          $\mathscr M_{a} \ne \emptyset$: denn $(a) \in \mathscr M_{a}$ für $t = 0$. Sei nun $(b) \in \mathscr M_{a}$ ein maximales Element (bzgl. $\subseteq$). Behauptung: $(b) = R$ (Dann $b \simeq 1 \imp q_{1} \cld q_{t} \simeq a$ für Faktorisierung in Definition von $\mathscr M_{a}$ zu $b$). Dazu: Nehme an: $(b) \subsetneq R$, dann ist wegen $R$ noethersch $\ex$ maximales Ideal $M \subsetneq R$ mit $(b) \subseteq M$, da $R$ HI-Ring ist $M= (p)$ für $p$ ein Primelement (Proposition 50) und aus $(b) \subseteq (p)$ folgt $p \mid b$ und $(\fak bp) \supsetneq (b)$ ($p$ keine Einheit) und $(\fak bp) \in \mathscr M_{a}$, da $\fak bp \cd p \cd q_{1} \cld q_{t}\simeq a$. Widerspruch zur Maximalität von $(b)$.
  \end{enumerate}
\end{proof}
\end{kor}
\begin{bem*}
Nächstes Ziel $R$ faktoriell $\imp R[X]$ faktoriell.
\end{bem*}
\begin{prop}[Übung]\item
  \begin{enumerate}[(a)]
    \item Für einen kommutativen Ring $R$ sind äquivalent:
          \begin{enumerate}[(i)]
            \item $R$ ist ein Integritätsbereich.
            \item $R[X]$ ist ein Integritätsbereich
                  \item $\forall f, g \in R[X]$ gilt: $\grad(fg) = \grad f + \grad g$
          \end{enumerate}
          \item Ist $R$ ein Integritätsbereich, so gilt $(R[X])\en = R\en$
  \end{enumerate}
\begin{proof}[Beweis]
\begin{enumerate}[(a)]
  \item Zeige (i) $\imp$ (iii) $\imp $ (ii) $\imp$ (i)
  \item $u \in R[X]\en \imp \ex r : vu = 1$, dann Grad Identität anwenden...$\qedhere$
\end{enumerate}
\end{proof}
\end{prop}
\begin{bsp*}
$\Z[X]\en = \Z\en = \{\pm 1\}, K[X_{1}, X_{2}]\en = K\en$.
\end{bsp*}
\begin{defi}
  Sei bis auf Widerruf $R$ ein faktorieller Ring mit Quotientenkörper $K = \Quot(R) \supseteq R$, $f = \sum_{i=0}^{n}a_{i}X^{i} \in K{X} \setminus \{0\}$ heißt primitiv wenn:
  \begin{enumerate}[(a)]
    \item $f \in R[X]$ (alle $a_{i} \in R$)
    \item $\ggt(a_{0}, \ldots, a_{n}) \simeq 1$
  \end{enumerate}
\end{defi}

\begin{lemm} Sei $f = \sum_{i=0}^n a_{i}X^{i} \in K[X] \setminus \{0\}$, dann:
  \begin{enumerate}[(a)]
    \item $\ex c \in K\en \ex g \in K[X] \setminus \{0\}$ primitiv, so dass $f = cg$
          \item Gelte $cg = c'g'$ mit $c, c' \in R\en, g, g' \in K[X] \setminus \{0\}$ primitive Polynome, so folgt $\frac cc \in R\en$, d.h. $c$ in (a) ist eindeutig bis auf Faktor in $R\en$
  \end{enumerate}
\begin{proof}{Beweis}
  \begin{enumerate}[(a)]
    \item Schreibe $a_{i} = \frac {b_{i}}{d_{i}}$ it $b_{i} \in R, d_{i} \in R \setminus \{0\}$ als gekürzten Bruch (d.h. $\ggt(b_{i}, d_{i}) \simeq 1$ geht $R$ faktoriell.) \newline
          Sei $d = \kgv(d_{0}, \ldots, d_{n})$ (Hauptnenner), $b = \ggt(b_{0}, \ldots, b_{n})$. Es folgt $g:= f \cd \frac ab = \sum_{i=0}^n \l( \frac{b_{i}}b \cd \frac d {d_{i}} \r)X^{i} \in R[X] (\setminus \{0\})$.\newline
          Behauptung: $g$ ist primitiv. $\l(\implies c= \frac bd = \frac{\ggt(b_{0}, ..., b_{n})}{\kgv(d_{0}, ..., d_{n})}\r)$.\newline
          Annahme: $g$ ist nicht primitiv. Dann $\ex$ Primelement $p \in R$, sodass $p \mid \frac {b_{i}}b \cd \frac d{d_{i}}, \forall i$. Nach der Wahl von $b$ gilt $\frac {b_{0}}b, \ldots ,\frac {b_{n}}b$ sind insgesamt teilerfremd $\implies \ex i : p \not |\  \frac {b_{i}}b \imp \ex i : p \mid \frac d{d_{i}} \imp p \mid d$. Sei $k = v_{p}(d)$, d.h. $p^{k}$ teilt $d, p^{k+1} \not | \ d \underset{d = \kgv(...)} \imp \ex i : p^{k} \mid d_{i}, p^{k+1} \not | \ d_{j}, j \in \{0, \ldots, n\}$, sei dieses $i_{0}$. Insbesondere ist $p \mid d_{i_{0}}$.\newline
          Beachte: $\frac {b_{i_{0}}}{d_{i_{0}}}$ gekürzter Bruch $\imp p \not | \ b_{i_{0}}$, aber nach Voraussetsung ($g$ nicht primitiv) $p$ teilt $\frac{b_{i_{0}}} b \cd \frac d{d_{i_{0}}}$. Widerspruch, da $p$ kein Teiler von $b$ oder $d_{i_{0}}$ ist.
    \item Haben $c \cd g = c' \cd g'$ für $c, c' \in K\en, g,g'$ primitiv. Schreibe $u = \frac {c'}c$ als gekürzter Bruch $u = \frac {d'}d \imp d \cd h = d' \cd g' = d \cd \sum b_{i}X^{i} = d' \cd \sum b_{i}'X^{i}$. $g, g'$ primitiv heißt $\imp d = \ggt(b_{0}d, \ldots, b_{n}d) = \ggt(b'_{0}d', \ldots, b'_{n}d') = d' \imp \frac {d'}d \in R\en$ und $\frac{c'}c = \frac{d'}d.$
  \end{enumerate}

\end{proof}

\end{lemm}
\end{document}
