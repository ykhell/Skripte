\documentclass[a4paper]{report}
\usepackage{amsfonts}
\usepackage{amssymb}
\usepackage{amsmath}
\usepackage{amsthm}
\usepackage{mathabx}
\usepackage{graphicx}
\usepackage{enumerate}
\usepackage{faktor}
\usepackage{tikz-cd}
\usepackage{color}   %May be necessary if you want to color links
\usepackage{hyperref}
\hypersetup{
    colorlinks=true, %set true if you want colored links
    linktoc=all,     %set to all if you want both sections and subsections linked
    linkcolor=black,  %choose some color if you want links to stand out
  }
%\usepackage{geometry}
% \geometry{
% a4paper,
% top=40mm,
% }

\theoremstyle{plain}
\newtheorem{thm}{Theorem}%[subsection]
\newtheorem{lemm}[thm]{Lemma}
\newtheorem{prop}[thm]{Proposition}
\newtheorem*{prop*}{Proposition}
\newtheorem{satz}[thm]{Satz}
\newtheorem*{satz45'}{Satz 45'}
\newtheorem{kor}[thm]{Korollar}
\newtheorem{cor}[thm]{Corollary}
\newtheorem{thm-defi}[thm]{Theorem-Definition}

\renewcommand*\contentsname{Inhaltsverzeichnis}
\renewcommand*\chaptername{Kapitel}

\theoremstyle{definition}
%\renewcommand{\thethm}{\arabic{thm}}
\newtheorem{defi}[thm]{Definition}
\newtheorem{rem}[thm]{Remark}
\newtheorem{bem}[thm]{Bemerkung}
\newtheorem*{bem*}{Bemerkung}
\newtheorem*{hin*}{Hinweis}
\newtheorem{exmp}[thm]{Example}
\newtheorem{bsp}[thm]{Beispiel}
\newtheorem*{bsp*}{Beispiel}
\newtheorem*{bspe*}{Beispiele}
\newtheorem{Conc}[thm]{Conclusion}
\newtheorem*{nota*}{Notation}
\newtheorem*{rech*}{Rechenregeln}
\newtheorem*{whg*}{Wiederholung}
\newtheorem{auf}[thm]{Aufgabe}
\newtheorem{ubng}[thm]{Übung}
\newtheorem*{ubng*}{Übung}
\newtheorem{loes}[thm]{Lösung}
\newtheorem{exer}[thm]{Exercise}
\newtheorem{soln}[thm]{Solution}

\let\ph\varphi
\newcommand{\Z}{\mathbb{Z}}
\newcommand{\R}{\mathbb{R}}
\newcommand{\C}{\mathbb{C}}
\newcommand{\Rn}{\mathbb{R}^n}

\DeclareMathOperator{\id}{id}
\DeclareMathOperator{\kn}{Kern}
\DeclareMathOperator{\bil}{Bild}
\DeclareMathOperator{\End}{End}
\DeclareMathOperator{\Aut}{Aut}
\DeclareMathOperator{\Iso}{Iso}
\DeclareMathOperator{\GL}{GL}
\DeclareMathOperator{\SL}{SL}
\DeclareMathOperator{\OR}{O}
\DeclareMathOperator{\SO}{SO}

\graphicspath{ {./graphics/} }
\title{\vspace{-2cm} Algebra 1 Vorlesungsmitschrieb}
\author{Yousef Khell}
\begin{document}
\maketitle
\tableofcontents
\chapter{Gruppentheorie}
\section{Gruppen und Monoide}%
\label{sec:Gruppen und Monoide}
\begin{nota*} \item
\begin{itemize}
  \item $\Bbb N = \{1, 2, ...\}$
  \item $\Bbb N_{0} = \Bbb N \cup \{0\}$
\item $\#X = $ die Kardinalität/Mächtigkeit einer Menge $X$
\end{itemize}
\end{nota*}

\begin{defi}[\textbf{Monoid}] %def-1
  Ein Tripel $(M, e, \circ)$ mit
  \begin{itemize}
 \item $M$ einer Menge.
 \item $e$ einem Element aus $M$,
 \item $\circ : M \times M \to M$ einer zweistelligen Verknüpfung
  \end{itemize}
  heißt \textbf{Monoid} falls gilt
  \begin{enumerate}[(M1)]
 \item Assoziativität: $$\forall a, b, c \in M : (a \circ b) \circ c = a \circ (b \circ c)$$
 \item Neutrales Element: $$\forall a \in M : a \circ e = a = e \circ a$$
  \end{enumerate}
  Wir nennen ein $a \in M$ \textbf{invertierbar}, falls $$\exists b, b' \in M : b \circ a = e = a \circ b'$$ (b bzw. b' heißen dann Links- bzw. Rechtsinverse)
 \begin{bem*}
$b = b'$, denn $$b' = e \circ b' = (b\circ a) \circ b' = b \circ (a \circ b') = b \circ e = b$$
 \end{bem*}
\end{defi}

\begin{defi}[\textbf{Gruppe}]%def-2
Eine \textbf{Gruppe} ist ein Monoid, in dem alle Elemente invertierbar sind.
\end{defi}
\begin{bem}[zur Assoziativität]%bem-3
Seien $a_{1}, ..., a_{n} \in M$, und setzt man in
$$a_{1} \circ \cdots \circ a_{n}$$
Klammern, sodass $\circ$ jeweils 2 Elemente verknüpft, so ist wegen (M1) das Ergebnis unabhängig von der Wahl der Klammerung, and also lässt man i.a. die Klammern weg. (Die Reihenfolge ist aber schon wichtig!)
\end{bem}
\begin{defi}[\textbf{Abelsche Gruppe/Monoid}] %def-4
Ein Monoid bzw. eine Gruppe $M$ heißt \textbf{abelsch} (oder kommutativ) $:\iff \forall a, b \in M:$ $$a \circ b = b \circ a$$
\end{defi}
\begin{prop}[Eindeutigkeit des neutralen Elements bzw. der neutralen Elementen]
  % TODO Check Grammar in Title
  % TODO Check Numbering in Vorlesung
  Sei $M$ ein Monoid, dann
  \begin{enumerate}[(a)]
    \item Erfüllt $e' \in M$ die Bedingung $e' \circ a = a \forall a \in M$, so gilt $e' = e$.
    \item Ist $a \in M$ invertierbar, so ist sein Inverses eindeutig.
  \end{enumerate}
\end{prop}
\begin{proof}[Beweis] \item
\begin{enumerate}[(a)]
  \item Nach Konstruktion $e = e' \circ e = e'$.
\item Gelte $a \circ b' = e$ und $b$ sei ein Inverses von $a$, dann:
$$b' = e \circ b' = (b \circ a) \circ b' = b \circ (a \circ b') = b \circ e = b. $$
\end{enumerate}
\end{proof}

\begin{satz}[ohne Beweis]%satz-6
  Sei $(G, e, \circ)$ ein Tripel  mit $G$ eine Menge, $e \in G$, $\circ : G \times G \to G$ eine assoziative Verknüpfung sodass:
  \begin{itemize}
    \item  $e$ ist Linkseins, d.h.
    $$\forall g \in G : e \circ g = g$$
    \item jedes $g$ hat ein Linksinverses $$\forall g \in G \exists h \in G : h \circ g = e$$
    So ist $(G, e, \circ)$ eine Gruppe.
  \end{itemize}
\end{satz}
\begin{hin*}[Nutzen von Satz 6]
  Es müssen weniger Axiome geprüft werden.
\end{hin*}
\begin{nota*}
  \item
  \begin{enumerate}[(i)]
\item $ab := a \circ b$
\item $a^{0} = e, a^{1}=a, a^{n+1}=a^{n}a, n \in N$
    \item $a^{n} = (a^{-n})^{-1}, n < 0$

\item  Ist $\circ$ kommutativ, so schreibt man oft $+$
  \end{enumerate}
\end{nota*}
\begin{ubng*}[Rechenregeln]
  \item
  \begin{enumerate}[(i)]
\item  $a^{n}a^{m}=a^{n+m}, (a^{n})^{m}=a^{nm}, \forall m,n \in \Bbb N_{0}$
\item  Ist $a$ invertierbar, so gelten die Regeln $\forall n,m \in \Bbb Z$
  \end{enumerate}
\end{ubng*}

\begin{prop}[Übung]
  Sei $G$ eine Gruppe, seien $g,h \in G$, dann:
  \begin{enumerate}[(a)]
    \item Die Glecihung $xg = h$ besitzt genau eine Lösung (in $G$), nämlich $x=hg^{-1}$.
    \item Es gilt $(gh)^{-1} = h^{-1}g^{-1}$
    \item Die Rechtstranslation (um $g$) $r_{g}: G \to G, x \mapsto xg$ und die Linkstranslationen (um $g$) $\ell_{g}: G \to G, x \mapsto gx$
    sind bijektiv.

  \end{enumerate}
\end{prop}

\begin{bsp*}
  % TODO Nummerierung wie in der Vorlesung
  \begin{enumerate}[1)]
\item $(\Bbb N_{0}, 0, +), (\Bbb N_{0}, 1, \cdot)$ sind kommutative Monoide.
\item Jede Gruppe ist ein Monoid.
\item Ist $X$ eine Menge, $\mathrm{Abb}(X,X)$ bzw. $\mathrm{Bij}(X,X)$ die Menge aller Abbildungen bzw. Bijektionen von $X$ in sich, so gilt:
\begin{enumerate}[(a)]
    \item $(\mathrm{Abb}(X,X), \mathrm{id}_{X}, \circ)$ ist ein Monoid.
    \item $(\mathrm{Bij}(X,X), \mathrm{id}_{X}, \circ)$ ist eine Gruppe.
\end{enumerate}
    Schreibe $S_{n}:=\mathrm{Bij}(\{1, ..., n\}, \{1, ..., n\})$ für die Gruppe der Permutationen von $\{1, ..., n\}$.
    \item Ist $(V, \langle \cdot, \cdot \rangle)$ ein Euklidischer Raum, so sind
    \begin{enumerate}[(i)]
    \item $O(V):= \{\varphi \in \mathrm{End}_{\Bbb R}(V) | \varphi \text{ orthogonal}\}$ und $SO(V):= \{\varphi \in O(V) | \det(\varphi) = 1\}$ Gruppen.
    \item Ist $V = \Bbb R^{2}$ und $P_{n}:=\{\cos \frac{2\pi j}{n}, \sin \frac{2\pi j}{n} \mid j = 0, ..., n-1\}$, dann ist
    \begin{enumerate}[(a)]
        \item $C_{n}:= \{\varphi \in SO(V) \mid \varphi(P_{n}) = P\}$ die Gruppe der Drehungen um 0 von Winkel $\frac{2\pi j}{n}, (j=0, ..., n=1)$ und
        \item $D_{n}:= \{\varphi \in O(V) \mid \varphi(P_{n})=P\}$ die [[Diedergruppe]] der Ordnung $2n$

    \end{enumerate}
        (Übung) $\#C_{n} = n, \#D_{n} = 2n$.
    \end{enumerate}
    Gruppen beschreiben oft Symmetrien eines geometrischen Objekts.
\item Ist $M$ ein Monoid, so ist $M^\times:=\{a \in M \mid a \text{ invertierbar}\}$ eine Gruppe, also $(M^{\times}, e, \circ)$.

  \end{enumerate}


\end{bsp*}
\begin{defi}[\textbf{Ring}]
  Ein [[Ring]] ist ein [[Tupel]] $(R, 0, 1, +, \cdot)$, sodass
  \begin{enumerate}[(R1)]
    \item $(R, 0, +)$ eine [[abelsche Gruppe]],
    \item $(R, 1, \cdot)$ ein Monoid,
    \item Es gelten die Distributivgesetze
  \end{enumerate}
\end{defi}

\begin{defi}[\textbf{Ordnung einer Gruppe}]
  Ist $M$ ein Monoid oder eine Gruppe, so heißt $$\mathrm{ord}(M):=\#M$$
  die Ordnung von $M$.
\end{defi}

\begin{defi}[\textbf{Untermonoid/Untergruppe}]
  Seien $M$ ein Monoid, $G$ eine Gruppe, dann
  \begin{enumerate}[(a)] %TODO Check nummer
    \item $N \subseteq M$ heißt Untermonoid (UM) wenn:
          \begin{itemize}
            \item $e \in N$
            \item $\forall n, n' \in N : n \circ n' \in N$
          \end{itemize}
    \item $H \subseteq G$ heißt Untergruppe (UG) wenn:
          \begin{itemize}
            \item $e \in H$
            \item $\forall h, h' \in H : h \circ h' \in H$
          \end{itemize}
  \end{enumerate}

So schreiben wir $N \leq M, H \leq G$.
\end{defi}

\begin{ubng}%übung-11
  \begin{enumerate}[(i)] %TODO Check nummer
    \item $N \leq M \implies (N, e, \cdot \mid_{N \times N}:N\times N \to N)$ ist Monoid
    \item $H \leq G \implies (H, e, \cdot \mid_{H \times H}:H\times H \to H)$ ist Monoid
  \end{enumerate}

\end{ubng}
\begin{bsp*}
  Sei $K$ ein Körper, dann ist
  \begin{enumerate}[(i)] %TODO Check nummer
    \item $SL_{n}(K) \leq GL_{n}(K)$
    \item $SO(V) \leq O(V) \le \mathrm{Aut}_{\Bbb R}(V)$
  \end{enumerate}
\end{bsp*}

\begin{prop}[Übung]
Sind $(H_{i})_{i \in I}$ Untergruppen von $G$, so ist
$$\bigcap_{i \in I} H_{i} \leq G.$$
\end{prop}

\begin{bsp*}
  Sei $G$ eine Gruppe, $g  \in G, S \le G$, dann:
  \begin{enumerate}[(i)] %TODO Check num
\item $C_{G}(g)$ \textbf{Zentralisator} von $g \in G$, also $$C_{G}(g) = \{h \in G \mid hg = gh\} \le G$$
\item $C_{G}(S)$ \textbf{Zentralisator} von $S$, also $$C_{G}(S) = \{h \in G \mid hs = sh \forall s \in S\} = \bigcap_{s \in S} C_{G}(s) \le G$$
\item $Z(G)$ \textbf{Zentrum} von $G$, also $$Z(G)=C_{G}(G) \underset{\text{komm.}}\le G$$
\item (Übung) $Z(GL_{n}(K)) = K^{\times}\mathbf{1}_{n}$
  \end{enumerate}

\end{bsp*}

\begin{lemm}
  Sei $G$ eine Gruppe und $S \subseteq G$ eine Teilmenge, dann $\exists$ kleinste Untergruppe $\langle S \rangle \le G$, die $S$ umfasst.
  \begin{proof}[Beweis]
    Definiere $$\langle S \rangle := \bigcap \{H \leq G \mid S \subseteq H\}.$$
  \end{proof}

\end{lemm}

\begin{ubng}
Sei $M$ ein Monoid, $S \subseteq M$ eine Teilmenge, ein Wort aus $S$ ist ein Ausdruck $$s_{1} \cdot \cdots \cdot s_{n}, s_{i} \in S, n\in N$$
Dann gilt: $\{\text{Worte in } S \cup \{e\}\} = \langle S \rangle \le M$ ist das kleinste Untermodnoid von M, das $S$ umfasst.
Und ist $G$ eine Gruppe, so gilt $\{\text{Worte in } S \cup S^{-1} \cup \{e\}\} = \langle S \rangle \le G$ ist die kleinste Untergruppe von G, die $S$ umfasst.
\end{ubng}

\begin{defi}[\textbf{Erzeugendensystem}]
Sei $G$ eine Gruppe und $S \subseteq G$ eine Teilmenge. $S$ heißt Erzeugendensystem von $G \iff \langle S \rangle = G$.
\end{defi}

\begin{bsp*}[Übung]
Seien $E_{ij} \in M_{n \times n}(K)$ die Elementarmatrizen mit $1$ an der Stelle $(i,j)$ und $0$ sonst. Dann ist
$$\{\mathbf 1_{n} + aE_{ij} \mid a \in K, i,j \in \{1, ..., n\}, i \neq j\}$$
ein Erzeugendensystem von $SL_{n}(K)$ (Gauß-Algorithmus)
\end{bsp*}

\begin{lemm}
Sei $G$ eine Gruppe, $g \in G$, dann gilt
$$\langle g \rangle = \langle \{g\} \rangle = \{g^{n} \mid n \in \Bbb Z\}$$
\begin{proof}[Beweis]
(Nach Übung 14) $$\langle \{g\} \rangle = \{\text{Worte in } \{g, g^{-1}, e\}\}$$
$$ = \{g^{i_{1}}, ..., g^{i_{n}} \mid n \in \Bbb N i_{1}, ..., i_{n} \in \{0, \pm 1\}\}$$
$$= \{g^{i_{1}+\cdots + i_{n}} \mid n \in \Bbb N i_{1}, ..., i_{n} \in \{0, \pm 1\}\}$$
$$= \{g^{n} | n \in \Bbb Z\}$$
\end{proof}
\end{lemm}
\begin{bem*}$\langle g \rangle$ ist abelsch.\end{bem*}
\begin{defi}[\textbf{Ordnung eines Gruppenelements, Zyklische Gruppe}]
  % TODO NUmmer
  Sei $G$ eine Gruppe, $g \in G$

\begin{enumerate}[(a)]

 \item Die Ordnung von $g$ ist
$$\mathrm{ord}(g) = \# \langle g \rangle = \# \{ g^{n} \mid n \in \Bbb{Z} \} \in \Bbb{N} \cup \{\infty\}$$
 \item$g$ hat endliche Ordnung $\iff \mathrm{ord}(g) \in \Bbb N$
 \item $G$ ist zyklisch $\iff \exists g \in G : G = \langle g \rangle$
\end{enumerate}
\end{defi}

% Vorlesung 2

\begin{prop}
  Zyklische Gruppen sind abelsch.
  \begin{proof}[Beweis]
$G$ zyklisch $\implies \exists g \in G : G = \langle g \rangle = \{ g^{n} \mid n \in \Bbb{Z} \}$. Dann: $$g^{n}g^{m} = g^{n+m} \overset{+ \text{komm. in } \Bbb Z}= g^{m+n} = g^{m}g^{n}.$$
  \end{proof}
\end{prop}

\begin{prop} %prop-19 %TODO Format
Sei $G$ eine Gruppe, $g \in G, n := \mathrm{ord}(g)$ und
$$n' = \sup\{m \in \Bbb N \mid e, g, g^{2} ..., g^{m-1} \text{ paarw. versch.}\}$$
Dann gelten:
\begin{enumerate}[(a)]
\item $n' = \infty = \sup \Bbb N$ oder $g^{n'} = e$ und $\langle g \rangle= \{e, g, g^{2} ..., g^{n'-1}\}$. Insbesondere ist $n'=n$
\item Falls $n = \mathrm{ord}(g) < \infty$, so gilt für $m,m' \in \Bbb Z$: $$g^{m} = g^{m'} \iff m \equiv m' \mod n$$
    Insbesondere ist $g^{m} = e \iff n \mid m$
\item Für $s \in \Bbb Z$ gilt $$\mathrm{ord}(g^{s}) = \frac{n}{\mathrm{ggT}(n,s)}$$
\end{enumerate}
\begin{proof}[Beweis]
  \item
  \begin{enumerate}[(a)]
    \item Gelte $n' < \infty$:

Definition von $n' \implies g^{n'} \in \{e, g, ..., g^{n'-1}\}$
Annahme: $g^{n'} = g^{i}$ für ein $i \in \{1, ..., n'-1\}$
Multipliziere mit $g^{-i} \implies g^{n'-i} = g^{0} = e$ und
$0 < n' - i < n'$, d.h. $g^{n'-i} \in \{e, ..., g^{n'-1}\}$
$\implies \{g^{0}, ..., g^{n'-1}\}$ nicht paarweise verschieden (Widerspruch)
Sei schließlich $m \in \Bbb Z$ beliebig, Division mit Rest: $$m = qn' + r : q,r \in \Bbb Z , 0 \le r \le n' - 1$$
$$\implies g^{m} = g^{qn'+r} = (g^{n'})^{q}g^{r} = g^{r} \in \{g^{0}, ..., g^{n-1}\}$$
Also: $\langle g \rangle = \{e, ..., g^{n'-1}\}$ sind paarweise verschieden. $\implies \mathrm{ord}(g) = \# \langle g \rangle = n'$
\item Seien $m, m' \in \Bbb Z$, schreibe $m'-m = qn'+r, (q, r \in \Bbb Z, 0 \le r \le n'-1)$, dann:
$$g^{m'} = g^{m} \iff g^{m'-m} = g^{0} = e \iff g^{qn'+r} = e$$
$$\iff g^{} = e \overunderset{1.\  n = n'}{e, ..., g^{n-1} \text{paarw. versch.}}\iff r = 0$$
$\iff m' - m$ ist Vielfaches von $n=n' \iff m \equiv m \mod n$
\item Bestime die $m \in \Bbb Z$ mit $(g^{s})^{m} = e$
$$(g^{s})^{m} = e \iff g^{sm} \underset{2.}= e \iff n \mid sm$$
$$\underset{\mathrm{ggT}(n,s) \mid n,s}\iff \frac n{\mathrm{ggT}(n,s)} \mid \frac s{\mathrm{ggT}(n,s)}m \iff \frac{n}{\mathrm{ggT}(n,s)} \mid m$$

  \end{enumerate}

Da $\frac n{\mathrm{ggT}(n,s)},  \frac s{\mathrm{ggT}(n,s)}$ teilerfremd sind
$$\overset{2. }\iff \mathrm{ord}(g^{s}) = \frac n{\mathrm{ggT}(n,s)} \ \square.$$
\end{proof}
\end{prop}

\begin{bsp*}
$$\mathrm{ord}(g) = 6 \implies \mathrm{ord}(g^{2}) = 3 = 6/\mathrm{ggT}(6,2) = 6/2$$
\end{bsp*}

\begin{kor}
  Sei $G$ eine Gruppe, dann
  \begin{enumerate}[(a)]
    \item Für $g \in G$ gilt: $$\mathrm{ord}(g)= \infty \iff g^{n}, n \in \Bbb Z \text{ sind paarw. verschieden}$$
    \item Ist $G$ zyklisch und $G \le G$ eine Untergruppe, so ist $H$ zyklisch.
  \end{enumerate}
\begin{proof}[Beweis]
\item
  \begin{enumerate}[(a)]
\item $\impliedby$ vgl. 19(a)
$\implies$ wissen nach 19(a), dass $e, g, ..., g^{n}, ...$ paarw. versch. sind. Multipliziere mit $g^{-m},(m \in \Bbb N)$
$\implies g^{-m}, g^{-m+1}, ..., g^{0}, g^{1}, ...$ sind paarw. versch.
\item Sei $g \in G$ ein Erzeuger von $G, H \le G$ eine UG von G und ohne Einschränkung $H \supsetneq \{e\}$
$$\implies \exists m \in \Bbb Z \setminus \{0\}:g^{m} \in H \setminus \{e\}$$
$H$ ist Gruppe $\implies g^{m}, (g^{m})^{-1} = g^{-m} \in H$

Sei $t \in \min\{m \in \Bbb N \mid g^{m} \in H\}$. Behauptung: $\langle g^{t} \rangle = H$.
\begin{itemize}
\item ``$\subseteq$'': Klar, da $g^{t} \in H$ also auch $\langle g^{t} \rangle \subseteq H$ ($H$ ist UG die $t$ enthält)
\item ``$\supseteq$'': Sei $g^{m} \in H$, Division mit Rest: $m = tq + r:  q, r \in \Bbb Z, 0 \leq r \leq t-1$
\end{itemize}
$\implies H \ni g^{m} = g^{tq + r} = {\underbrace{(g^{t})}_{\in H}}^{q}g^{r} \implies g^{r} = (g^{m})((g^{t})^{q})^{-1} \in H$

          Nach Def von $t$ muss gelten: $r = 0$, da $r = 1, ..., t-1$ verboten.
Also ist $g^{m} = (g^{t})^{q} \in \langle g^{t} \rangle$.
  \end{enumerate}

\end{proof}


\end{kor}
\begin{kor}[Übung]
Untergruppen von $\Bbb Z$ sind die Mengen $\Bbb Zn = \{an \mid a \in \Bbb Z\}, (n \in \Bbb N_{0})$
\end{kor}

\begin{whg*}[Vorbereitung]
  \item
\begin{itemize}
\item Äquivalenzrelationen
\item Äquivalenzklassen
\item Repräsentantensysteme
\end{itemize}
\end{whg*}

\begin{bem*} \item
  \begin{itemize}
        \item $X = \bigsqcup_{r \in \mathcal R} [r]_{\sim}$
        \item Falls $\#X < \infty : \# = \sum_{r \in \mathcal R} \# [r]_{\sim}$ %TODO what is this
  \end{itemize}
\end{bem*}

\begin{satz}[Satz von Lagrange]
Sei $G$ eine endliche Gruppe und $H \le G$ eine Untergruppe, dann gilt $\#H \mid \#G$.
\end{satz}
\begin{proof}[Beweis] \item
  \begin{enumerate} [1)]
    \item Definiere $\sim$ auf $G$ durch $g \sim g' :\iff \exists h \in H : g' = gh$
          $\sim$ ist eine Äquivalenzrelation:
          \begin{itemize}
            \item reflexiv: $g \sim g$ denn $g = ge, e \in H$
            \item symmetrisch: gelte $g' = gh$ für ein $h \in H$ $$\underset{\_\cdot h^{-1}}{\implies} g'h^{-1}=g \underset{H \text{ Gruppe}} \implies h^{-1} \in H \implies g' \sim g$$
            \item transitiv: gelte $g \sim g', g' \sim g''$, d.h. $\exists h \in H : g' = gh, \exists h' \in H " g'' = g'h$
          \end{itemize}
            $$\implies g'' = g'h' = (gh)h' = g(hh') \implies g \sim g''$$
    \item Äquivalenzklassen: Für $g \in G$ ist
    $$[g]_{\sim} = \{g' \in G \mid \exists h \in H : g' = gh\} = \{gh \mid h \in H\} =: gH$$
    \item Beachte $G$ endlich $\implies H \subseteq G$ endlich (und ebenso jede Teilmenge von $G$)
        Behauptung: $\#gH = \#H \forall g \in G$
        Grund: Die Abbildungen $$\ell_{g}: H \to gH, h \mapsto gh, \ell_{g^{-1}}: gH \to H, x \mapsto g^{-1}x$$ sind zueinander invers (Übung) und also biijektiv. $\implies \#H = \#gH$.
    \item Sei $\mathcal R \subseteq G$ ein Repräsentantensystem zu $\sim$

          $\implies \#G = \sum_{g \in \mathcal R} \# [g]_{\sim} = \sum_{g \in \mathcal R} \#gH = \sum_{g \in \mathcal R} \#H \overset{3)}= \# \mathcal R \# H$
  \end{enumerate}
$\implies \#H$ teilt $\#G$.
\end{proof}

\begin{nota*}
Seien $G$ eine Gruppe, $H \le G$ eine Untergruppe und $\sim$ wie im Beweis vom Satz 22.
\begin{itemize}
\item Schreibe $\faktor GH$ für die Menge aller Äquivalenzklassen also für $\{gH \mid g \in G\}$
\item Schreibe $[G : H]:= \#  \faktor GH = \# \mathcal R$ (Index von $H$ in $G$)
\end{itemize}
Lagrange sagt: $\#G = \# \faktor GH \cdot \#H = [G : H] \cdot \#H$
\end{nota*}

\begin{ubng}
Seien $H' \le H \le G$ Untergruppen, dann ist $H' \le G$ und
$$[G: H'] = [G:H]\cdot[H:H']$$
\end{ubng}

\begin{kor} %kor-24
  Sei $G$ eine endliche Gruppe, dann gelten:
  \begin{enumerate}[(a)]
    \item $\forall g \in G : \mathrm{ord}(g) \mid \mathrm{ord}(G) = \#G$
    \item Ist $\mathrm{ord}(G)$ eine Primzahl, so ist $G$ zyklisch
  \end{enumerate}
  \begin{proof}[Beweis] \item
    \begin{enumerate}[(a)]
\item $\langle g \rangle \le G$ ist eine Untergruppe $\underset{\text{Lagrange}}\implies \mathrm{ord}(g) = \#\langle g \rangle \mid \#G = \mathrm{ord}(G)$
\item Sei $p = \mathrm{ord}(G) \in \mathbb P$ eine Primzahl, sei $g \in G \setminus \{e\}$ ($\#G \ge 2$)
Nach 1. gilt $\underbrace{\mathrm{ord}(g)}_{\neq 1 \text{ da } g \neq e} \mid \mathrm{ord}(G) = p$
    \end{enumerate}
Folglich: $p = \mathrm{ord}(g) = \mathrm{ord}(G)$, d.h. $\langle g \rangle \le G$ ist Inklusion gleichmächtiger endlicher Mengen, also $\langle g \rangle = G.$
\end{proof}
\end{kor}

\begin{defi}[\textbf{Gruppenexponent}]
Sei $G$ eine Gruppe, der Exponent von $G$ ist $\exp(G) = \min \{n \in \Bbb N \mid \forall g \in G : g^{n} = e\}$ (wobei $\min \emptyset = \infty$).
\end{defi}

\begin{bsp*}[Übung] \item
\begin{enumerate}[(i)]
\item $\exp (C_{n}) = n$
\item $\exp D_{n} = \mathrm{kgV}(2,n)$
\item $\exp(S_{3}) = 6$
\item $\exp(S_{4}) = 12$
\item $\exp(G) = 2 \implies G$ abelsch
\item $\Bbb F_{p}$ Körper mit $p$ Elementen und $0 \neq V$ ein $\Bbb F_{p}$-[[Vektorraum]], so gilt $\exp(V, 0, +) = p$
\end{enumerate}
\end{bsp*}

\begin{satz}
  Sei $G$ eine endliche Gruppe, es gelten
  \begin{enumerate}[(a)]
    \item $\exp(G) \mid \mathrm{card}(G)$
    \item $\exp(G) = \mathrm{kgV}(\{\mathrm{ord}(g)\mid g \in G\})$
  \end{enumerate}
\begin{proof}[Beweis] \item
\begin{enumerate}[(a)]
\item Folgt aus (b) und $\mathrm{ord}(g) \mid \mathrm{ord}(G) \forall g \in G$ nach Korollar 24.
\item $\mathrm{ord}(g) \mid \exp(G), \forall g \in G$, denn nach Definition gilt:
$$g^{\exp(G)} = e \underset{19} \implies \mathrm{ord}(g ) \mid \exp(G)$$
\end{enumerate}
folglich: $N:= \mathrm{kgV}(\{\mathrm{ord}(g) \mid g \in G\})$ teilt $\exp G$.

        Behauptung: $\exp G \le N$, (dann fertig)

        Wir zeigen: $g^{N} = e \implies \exp G \le N$.
Dies folgt aus $g^{\mathrm{ord}(g)} = e$ und $\mathrm{ord}(g) \mid N = \mathrm{kgV}(...).$
\end{proof}
\end{satz}

\begin{ubng}%ubng-27
  Sei $G$ eine endliche Gruppe, dann gelten:
  \begin{enumerate}[(a)]
    \item Sind $g,h \in G : gh = hg$ und gilt $\mathrm{ggT}(\mathrm{ord}(g), \mathrm{ord}(h)) = 1$, so gilt $$\mathrm{ord}(gh) = \mathrm{ord}(g)\mathrm{ord}(h)$$
    \item Gelte $p^{f} \mid \exp G$ für $p$ eine Primzahl und $f \in \Bbb N$, dann $\exists g \in G : \mathrm{ord}(g) = p^{f}$
    \item Ist $G$ abelsch, so $\exists g \in G : \exp (G) = \mathrm{ord}(g)$
  \end{enumerate}
\end{ubng}

\begin{satz}
Sei $G$ eine endliche abelsche Gruppe, dann ist $G$ genau dann zyklisch, wenn $\mathrm{ord}(G) = \exp(G)$
\begin{proof}[Beweis] \item
\begin{itemize}
\item ``$\implies$'': Sei $g \in G$ Erzeuger $\underset{19}\implies \mathrm{ord}(G) = \mathrm{ord}(g)$
$$\mathrm{ord}(g) \mid \exp G, \exp G \mid \mathrm{ord}(G) \implies \exp G = \mathrm{ord}(G)$$
\item ``$\impliedby$'': Wähle nach 27.3 ein $g \in G$ mit $\mathrm{ord}(g) = \exp(G)$, nach Voraussetzung ist $\exp(G) = \mathrm{ord}(g) \implies \mathrm{ord}(g) = \mathrm{ord}(G) \implies \langle g \rangle \subseteq G$ ist Gleichheit, d.h. $\langle g \rangle = G.$
\end{itemize}
\end{proof}
\end{satz}
\section{Gruppenhomomorphismen}%
\label{sec:Gruppenhomomorphismen}
Seien im Weiteren $M, M'$ Monoide und $G, G'$ Gruppen.
\begin{defi}[\textbf{Monoid-/Gruppenhomomorphismus}] \item
  \begin{enumerate}[(a)]
    \item Eine Abbildung $\varphi : M \to M'$ heißt \textbf{Monoidhomomorphismus}, falls
          \begin{enumerate}[(i)]
            \item $\varphi(e) = e'$ und
                  \item $\forall m , \tilde m \in M : \varphi(m \circ \tilde m) = \varphi(m) \circ' \varphi(\tilde m)$
          \end{enumerate}
          \item Sind $M, M'$ Gruppen, so heißt ein Gruppenhomomorphismus $\iff$ (ii) gilt.
  \end{enumerate}
\end{defi}
\begin{bem} \item
 \begin{enumerate}[(a)]
   \item Ist $\varphi : M \to M'$ ein Gruppenhomomorphismus, so gilt $\varphi(e)=e'$ und $\varphi(m^{-1}) = \varphi(m)^{-1}, \forall m \in M$.
         \item (Übung) Die Verkettung von Monoid- bzw. Gruppenhomomorphismen ist wieder ein solcher.
 \end{enumerate}
 \begin{proof}[Beweis]
Zu (a): $$e' \circ' \varphi(e) = \varphi(e) = \varphi(e \circ e) = \varphi(e) \circ' \varphi(e)$$
Kürzen $\implies e' = \varphi(e)$. Und $$\varphi(m^{-1})\circ' \varphi(m) = \varphi(m^{-1} \circ m) = \varphi(e) = e'$$
Eindeutigkeit des Inverses $\implies \varphi(m^{-1}) = \varphi(m)^{-1}$.
 \end{proof}
\end{bem}
\begin{bsp}
  \begin{enumerate}[(a)]
    \item Für $g \in G$ ist die Abbildung
          $$\varphi: \Bbb Z \to G, n \mapsto g^{n}$$
          ein Gruppenhomomorphismus mit $\mathrm{Bild}(\varphi) = \langle g \rangle$.
    \item Sei $K$ ein Körper, $V, W$ $K$-Vektorräume, $\varphi : V \to W$ ein Vektorraumhomomorphismus, dann ist
          $$\varphi : (V, 0_{V}, +_{V}) \to (W, 0_{W}, +_{W})$$
          ein Gruppenhomomorphismus.
    \item Die Vorzeichenfunktion (Aus der linearen Algebra)
          $$\mathrm{sgn}: S_{n} \to \{\pm 1\}, \sigma \mapsto \mathrm{sgn}(\sigma)$$
          ist ein Gruppenhomomorphismus.
  \end{enumerate}
\end{bsp}

\begin{defi}[\textbf{Kern/Bild}]
  Sei $\varphi : G \to G'$ ein Gruppenhomomorphismus.
  \begin{enumerate}[(a)]
    \item Der Kern von $\varphi$ ist $\mathrm{Kern}(\varphi) := \{g \in G \mid \varphi(g) = e'\}$
    \item Das Bild von $\varphi$ ist $\mathrm{Bild}(\varphi) := \{\varphi(g) \in G'\mid g \in G\}$
  \end{enumerate}
\end{defi}

\begin{prop}[Übung]
  Sei $\varphi : G \to G'$ ein Gruppenhomomorphismus, dann
  \begin{enumerate}[(a)]
          \item Für $H \le G$ eine Untergruppe ist $\varphi(G) \le G'$ eine Untergruppe.
    \item Für $H' \le G'$ eine Untergruppe ist $\varphi^{-1}(H') \le G$ eine Untergruppe.

          Insbesondere sind $\mathrm{Bild}(\varphi) \le G', \mathrm{Kern}(\varphi) \le G$ Untergruppen.
    \item $\varphi$ ist injektiv (ein Gruppenmonomorphismus) $\iff \mathrm{Kern}(\varphi) = \{e\}$.
    \item $\varphi$ ist surjektiv (ein Gruppenepimorphismus) $\iff \mathrm{Bild}(\varphi) = G'$
  \end{enumerate}
  \begin{bem*}
(a), (b) und (d) gelten auch für Monoide.
  \end{bem*}
\end{prop}

\begin{defi}[\textbf{Gruppenisomorphismus}]
Ein Gruppenhomomorphismus $\varphi$ ist ein Gruppenisomorphismus, wenn $\varphi$ bijektiv ist. ($\iff \mathrm{Kern}(\varphi) = \{e\}$ und $\mathrm{Bild}(\varphi) =G'$).
\end{defi}
\begin{bem*}[Übung]
  Definiere ein Monoidhomomorphismus analog zu Definition 24.
\end{bem*}

\begin{nota*}
  Wir schreiben $G \cong G'$ ($G$ ist isomorph zu $G'$) wenn $\exists$ Gruppenisomorphismus $\varphi : G \to G'$.
\end{nota*}

\begin{defi}[\textbf{Gruppenautomorphismus}]
\begin{enumerate}[(a)]
  \item Ein Gruppenisomorphismus $\varphi : G \to G$ heißt Gruppenautomorphismus.
  \item $\mathrm{Aut}(G):= \{\varphi : G \to G \mid \varphi \text{ ist ein Gruppenautomorphismus}\}$.
\end{enumerate}
\end{defi}


\begin{bem}[Übung]
  \begin{enumerate}[(a)]
    \item $\mathrm{id}_{G} : G \to G \in \mathrm{Aut}(G)$
    \item Verkettung von Gruppenisomorphismen (oder Automorphismen) ist wieder ein solcher.
    \item Ist $\varphi : G \to G'$ ein Gruppenisomorphismus, so gelten
          \begin{enumerate}[(i)]
            \item $\#G = \#G'$.
            \item $G$ abelsch $\iff G'$ abelsch.
            \item $S \subseteq G$ ein Erzeugendensystem $\iff \varphi(S) \subseteq G'$ ein Erzeugendensystem.
          \end{enumerate}
  \end{enumerate}
\end{bem}


\begin{prop} $(\mathrm{Aut}(G), \mathrm{id}_{G}, \circ)$ und $(\mathrm{Aut}(M), \mathrm{id}_{M}, \circ)$ sind Gruppen.
  \begin{proof}[Beweis](Übung) Zeige: $$\mathrm{Aut}(G) \le \mathrm{Bij}(G) , \mathrm{Aut}(M) \le \mathrm{Bij}(M)$$
sind Untergruppen.
  \end{proof}
\end{prop}

\begin{bsp}[Übung]\item
  \begin{enumerate}[(a)]
    \item $\mathrm{Aut}((\Bbb Z, 0, +)) = \{\mathrm{id}_{\Bbb Z}, - \mathrm{id}_{\Bbb Z}\} \cong C_{2}$
    \item Für $\Bbb Z_{n}:= \faktor{\Bbb Z}{(n)}$ der Ring der Restklassen modulo $n$ gilt
          $$(\Bbb Z_{n}, \overline 0, +) \cong C_{n} \text{ und } \mathrm{Aut}(\Bbb Z_{n}, \overline 0, +) \cong \Bbb Z_{n}^{\times}$$
          z.B. Erzeuger von $\Bbb Z_{n}$ sind Reste $\overline a$, sodass $\mathrm{ggT}(a,n) = 1$
    \item Sei $G$ beliebig, zu $g \in G$ definiere den \textbf{Konjugationsautomorphismus} (\textbf{Konjugation} mit $g$)
          $$c_{g}: G \to G, h \mapsto g\circ h \circ g^{-1}$$
\begin{enumerate}[(i)]
  \item $c_{g} \circ c_{g'} = g_{g\circ g'}, \forall g, g' \in G$
  \item $c_{e} = \mathrm{id}_{G}$ und $c_{g} \in \mathrm{Aut}(G), \forall g \in G$
  \item $c_{\cdot}: G \to \mathrm{Aut}(G), g \mapsto c_{g}$ ist ein Gruppenhomomorphismus.
  \item $\mathrm{Kern}(c_{\cdot}) = Z(G)$ (Zentrum von $G$).
\end{enumerate}
  \end{enumerate}
\end{bsp}
\begin{bem*}
$\mathrm{Bild}(c_{\cdot}) =: \mathrm{Inn}(G)$ die Gruppe der \textbf{inneren Automorphismen} von $G$
\end{bem*}

\begin{lemm}
  Seien $\varphi, \varphi' : G \to G'$ Gruppenhomomorphismen. Sei $S \subseteq G$ ein Erzeugendensystem. Dann gilt
  $$\varphi(s) = \varphi'(s) \forall s \in S \iff \varphi = \varphi'\quad (*)$$
  Analoge Aussage gilt für Monoide
\end{lemm}
\begin{proof}[Beweisskizze](Übung)
  \begin{itemize}
          \item ``$\impliedby$'': Klar.
    \item ``$\implies$'':
          \begin{enumerate}[1)]
            \item Zeige $H := \{g \in G \mid \varphi(g) = \varphi'(g)\}i \le G$ ist eine Untergruppe.
            \item Da $S \subseteq$ nach Definition von $H$ und Voraussetzung von ``$\implies$'', folgt $G = \langle S \rangle \subseteq H \le G$
          \end{enumerate}
  \end{itemize}
\end{proof}
\section*{Normalteiler (Normal Subgroup)}
\begin{nota*}
  Für $X \subseteq G$ und $g \in G$ setze $$\ell_{g}(X) = \{gx \mid x \in X\} = gX \text{ und } r_{g}(X) = \{xg \mid x \in X\} = Xg$$
  Gruppenverknüpfung assoziaativ $\implies$
  \begin{enumerate}[(i)]
    \item $c_{g}(X) = \{gxg^{-1} \mid x \in X\} = (gX)g^{-1} = g(Xg^{-1})$.
    \item $g(hX) = (gh)X$ und $(Xg)h = X(gh)$.
  \end{enumerate}
\end{nota*}

\begin{bem*} Ist $H \le G$ eine Untergruppe, dann heißt $gH$ \textbf{Linksnebenklasse} und $Hg$ \textbf{Rechtsnebenklasse}.
\end{bem*}

\begin{defi}[\textbf{Normalteiler}]
  Eine Untergruppe $N \le G$ heißt Normalteiler (N.T.) $\iff \forall g \in G : Ng = gN$. (Diese Definition ist auch für Monoide sinnvoll)
\end{defi}

\begin{lemm}
  Für eine Untergruppe $N \le G$ sind äquivalent:
  \begin{enumerate}[(i)]
    \item $\forall g \in G : gN = nG$
    \item $\forall g \in G : gNg^{-1} = N$
    \item $\forall g \in G : gNg^{-1} \subseteq N$
  \end{enumerate}
  \begin{proof}[Beweis]
\begin{itemize}
\item ``$(ii)\implies (iii)$'': Klar.
  \item ``$(iii)\implies (i)$'': Rechtsmultiplikation mit $g$ liefert aus $(iii)$:
        $$(gNg^{-1})g = gN(g^{-1}g) = gNe = gN \subseteq Ng$$
        Für die andere Inklusion betrachte $(iii)$ für $g^{-1}$:
        $$g^{-1}Ng \subseteq N \underset{\text{Linksmult. mit }g}\implies Ng \subseteq gN$$
\item ``$(i)\implies (ii)$'': Wende auf $(i)$ Rechtsmultiplikation mit $g^{-1}$ an. ($r_{g^{-1}}: G \to G$ ist eine bijektive Abbildung.)
\end{itemize}
  \end{proof}
\end{lemm}

\begin{nota*}
  \item $H \le G$ bedeuteg $H \subseteq G$ ist eine Untergruppe.
  \item $H \trianglelefteq G$ bedeuteg $H \subseteq G$ ist ein Normailteiler.
\end{nota*}

\begin{satz}
  Ist $\varphi : G \to G'$ ein Gruppenhomomorphismus, so ist $\mathrm{Kern}(\varphi) \trianglelefteq G$ ein Normalteiler.
  \begin{proof}[Beweis]
    Sei $g \in G$ beliebig, zu zeigen ist $g \circ \mathrm{Kern}(\varphi) \circ g^{-1} \subseteq \mathrm{Kern}(\varphi)$
    \item Sei $h \in \mathrm{Kern}(\varphi)$, zu zeigen ist $ghg^{-1} \in \mathrm{Kern}(\varphi)$.
    Damit:
    $$\varphi(ghg^{-1}) = \varphi(g)\varphi(h)\varphi(g^{-1}) \underset{h \in \mathrm{Kern}(\varphi)} = \varphi(g) \circ e' \circ \varphi(g^{-1}) = \varphi(g)\varphi(g^{-1})$$
    $$= \varphi(gg^{-1}) = \varphi(e) = e'.$$
    $\implies \mathrm{Kern}(\varphi) \trianglelefteq G.$
  \end{proof}
\end{satz}

\begin{ubng} \item
  \begin{enumerate}[(a)]
          \item Ist $N' \trianglelefteq G'$ und $\varphi : G \to G'$ Gruppenhomomorphismus, so gilt $\varphi^{-1}(N') \trianglelefteq G$.
    \item Ist $h \le G$ eine Untergruppe mit $[G:H] = \# \faktor GH= 2$, so folgt $H \trianglelefteq G$.
    \item Ist $G$ abelsch, so ist jede Untergruppe $H \le G$ ein Normalteiler.
    \item Der \textbf{Kommutator} zu $g, h \in G$ ist $ghg^{-1}h^{-1}$, die \textbf{Kommutatoruntergruppe} von $G$ ist $$[G, G]:= \langle ghg^{-1}h^{-1} \mid g, h \in G \rangle$$
\end{enumerate}
Es gilt $[G, G] \trianglelefteq G.$
\end{ubng}

\begin{bsp*} Es gibt Beispiele für folgende Aussagen:
  \begin{enumerate}[(i)]
    \item $\exists H \le G : H \not \trianglelefteq G$
    \item $\varphi : G \to G'$ ein Gruppenhomomorphismus und $N \trianglelefteq G$ mit $\varphi(G) \not \trianglelefteq G'$
    \item $\exists N \trianglelefteq G$ und $H \trianglelefteq N$, so dass $H \not \trianglelefteq G$.
  \end{enumerate}
\end{bsp*}
  \begin{proof}[Beweis] \item
  \begin{enumerate}[(i)]
    \item $G = S_{3} = \mathrm{Bij}(\{1,2,3\}) \supseteq H = \{\mathrm{id}, \sigma\}$ mit $\sigma = \begin{pmatrix} 1 & 2 & 3 \\ 2 & 1 & 3\end{pmatrix}$. Dann $H \le G$ Klar, aber $H \not \trianglelefteq G$, denn für $\tau = \begin{pmatrix} 1 & 2 & 3 \\ 1 & 3 & 2\end{pmatrix}$ gilt $\tau \sigma \tau^{-1}$ (Übung) $\implies \tau H\tau^{-1} \not \subseteq H$
    \item Betrachte $\varphi: H \to G$ Inklusion mit $G, H$ aus (i), dann gilt $H \trianglelefteq H$ aber $\varphi(H) = H$ nein Nullteiler von $G = S_{3}$.
    \item Später.
   \end{enumerate}
  \end{proof}
\begin{satz}
  Sei $N \trianglelefteq G$ ein Nullteiler, dann gelten:
  \begin{enumerate}[(a)]
    \item Aus $gN = g'N$ und $hN = h'N$ für $g, g', h, h' \in G$ folgt $ghN = g'h'N$ und insbesondere ist die Verknüpfung $$\circ : \underbrace{\faktor GN}_{\{gN \mid g \in G\}} \times \faktor GN \longrightarrow \faktor GN,\  (gN, hN) \longmapsto gN \circ hN = ghN$$
          wohl-definiert.
    \item $\faktor GN, \underbrace{N}_{=eN}, \circ)$ ist eine Gruppe.
    \item $gN = g'N \iff g^-1g' \in N$.
    \item $\pi : G \to \faktor GN, g \mapsto gN$ ist ein Gruppenhomomorphismus mit $\mathrm{Kern}(\pi) = N$.
  \end{enumerate}
\begin{proof}[Beweis]
\begin{enumerate}[(a)]
  \item Es gelten (Formeln von Definition 40)
        $$(gh)N = g(hN) \overset{N \trianglelefteq G}= g(Nh) = (gN)h$$
        $$= (g'N)h = g'(Nh) = g'(hN) = g'(h'N) = (g'h')N \implies (a)$$
  \item Überlege Gruppenaxiome.
\begin{itemize}
  \item Assoziativität (Übung)
  \item Linkseins ist $N = eN$, denn
        $$N \circ(gN) = eN\circ gN \overset{\text{wohl-def.}} = (e \circ g)N = gN$$
  \item Linksinverses zu $gN$ ist $g^{-1}N$, denn
        $$(g^{-1}N)\circ gN \underset{\text{nach Def.}}= (g^{-1}g)N \underset{\text{Gruppe}}= eN = N$$
\end{itemize}
  \item  $gN = g'N \underset{g^{-1}\circ \underline{\ }}= N = g^{-1}g'N \underset{e \in N}\implies N \ni g^{-1}g'e$, d.h. $g^{-1}g' \in G$.
        $$g^{-1}g' \in N \underset{\substack{\ell_{g^{-1}g'} : N \to N\\ \text{ist bijektiv.}}}\implies N = g^{-1}N \underset{g^{-1}\circ \underline{\ }}\implies gN= g'N$$
  \item $\pi : G \to \faktor GN, g \mapsto gN$ ist Gruppenhomomorphismus, denn $$\pi(gg') = gg'N \underset{\text{Def. von } \circ} = gN \circ g'N = \pi(g)\circ \pi(g')$$
        $$g \in \mathrm{Kern}(\pi) \iff gN = eN \underset{(c)}\iff e^{-1}g = g \in N$$
\end{enumerate}
\end{proof}
\end{satz}
\begin{bem*}[Bezeichnung] $\faktor GN$ (bzw. $(\faktor GN, eN, \circ)$) heißt \textbf{Faktorgruppe} von $G$ modulo $N$.
\end{bem*}
\begin{bem*}[Übung]
$G$ abelsch $\implies \faktor{G}{N}$ abelsch.
\end{bem*}

\begin{satz}[Homomorphiesatz für Gruppen]
  Sei $\varphi: G \to G'$ ein Gruppenhomomorphismus mit $N = \mathrm{Kern}(\varphi)$, dann existiert genau ein Gruppenhomomorphismus $\overline \varphi : \faktor GN \longrightarrow G'$, sodass
  $$
\begin{tikzcd}
G \arrow[r, "\varphi"] \arrow[d, "\pi"'] & G' \\
\faktor GN \arrow[ru, "\overline \varphi"']               &
\end{tikzcd}
$$
kommutiert, d.h. $\overline \varphi \circ \pi = \varphi$. (wobei $\pi : G \longrightarrow \faktor GN, g \mapsto gN$ aus Satz 44). Die Abbildung $\overline \varphi$ ist injektiv und $\overline \varphi$ bijektiv $\iff \varphi$ surjektiv.
\begin{proof}[Beweis]
\begin{itemize}
  \item Existenz von $\overline \varphi$: Definiere $\overline\varphi(gN) = \varphi(g), \forall g \in G$.
  \item $\overline\varphi$ wohl-definiert: Es gilt: $gN = g'N \iff N = g^{-1}g'N \underset{44c}\iff g^{-1}g' \in N.$ Damit $$\implies \varphi(g') = \varphi(gg^{-1}g') = \varphi(g)\varphi(\underbrace{g^{-1} \circ g'}_{\in N = \mathrm{Kern}(\varphi)}) = \varphi(g)e = \varphi(g).$$
  \item $\overline\varphi$ Gruppenhomomorphismus: $$\overline\varphi(gN \circ g'N)\underset{\text{Def. von } \circ} = \overline\varphi(gg'N) \underset{\text{Def. von } \overline\varphi} =  \varphi(gg') \underset{\varphi \text{ Hom.}} = \varphi(g)\varphi(g')$$
        $$\underset{\text{Def. von } \overline\varphi} =\overline\varphi(gN)\overline\varphi(g'N).$$
\item $\overline\varphi \circ \pi = \varphi$: (Aus der Definition von $\overline\varphi$): $$\underbrace{\overline\varphi(gN)}_{\overline\varphi(\pi(g))} = \varphi(g)$$
\item $\overline\varphi$ injektiv: $\overline\varphi(gN) = e \iff \varphi(g)=e \iff g \in N = \mathrm{Kern}(\varphi) \underset{44c} \iff gN= eN = N$.
\item $\overline\varphi$ eindeutig: Folgt aus der Surjektivität von $\pi$.
\item Zusatz $\varphi$ surjektiv $\iff \overline\varphi$ Isomorphismus (Übung): Verwende $\mathrm{Bild}(\varphi) = \mathrm{Bild}(\overline\varphi)$ und $\overline\varphi$ injektiv.
\end{itemize}
\end{proof}
\end{satz}
\begin{satz45'}[Homomorphiesatz'](Übung)
  Ist $\varphi : G \to G'$ ein Gruppenhomomorphismus und $N \trianglelefteq G$, so dass $N \subseteq \mathrm{Kern}(\varphi)$, dann existiert genau ein Gruppenhomomorphismus $$\overline\varphi : \faktor GN \longrightarrow G'\text{ mit }\ \overline\varphi \circ \pi = \varphi.$$
  wobei $\pi : G \to \faktor GN, g \mapsto gN$
\end{satz45'}
\begin{nota*}
Für $n \in \Bbb N$ sei $\Bbb Z_{n} = \faktor {\Bbb Z}{(n)} = \faktor{\Bbb Z}{n\Bbb Z}$ der Restklassenring. ($n\Bbb Z \subseteq \Bbb Z $ eine Untergruppe)
\end{nota*}

\begin{kor}
  Sei $G$ eine zyklische Gruppe,
  \begin{enumerate}[(a)]
    \item Falls $m:= \mathrm{ord}(G) \in \Bbb N \implies G \cong \Bbb Z_{m} = \faktor{\Bbb Z}{(m)}$.
          \item Falls $\mathrm{ord}(G) = \infty \implies G \cong \Bbb Z$.
  \end{enumerate}
\begin{proof}[Beweis]
  Sei $g \in G$ ein Erzeuger und betrachte $$\varphi : \Bbb Z \to G, n \mapsto g^{n}$$
  $\varphi$ ist surjektiv, da $\mathrm{Bild}(\varphi) = \langle g^{n} \mid n \in \Bbb Z \rangle = G$.
  $$\underset{\text{Satz 45}}\implies \overline\varphi : \faktor{\Bbb Z}{\Bbb Zm} \overset \cong \longrightarrow{} G$$
  für $m \in \Bbb N_{0}$, so dass $\mathrm{Kern}(\varphi) = \Bbb Zm$.
  \begin{itemize}
\item Fall (b): $\mathrm{ord}(G) = \infty \implies \mathrm{Kern}(\varphi) = \{0\} \implies \varphi : \Bbb Z \to G$ ist ein Isomorphismus.
\item Fall (a): $\mathrm{ord}(G) = m \in \Bbb N$ dann ist $\overline\varphi$ der gewünschte Isomorphismus.
 \end{itemize}
\end{proof}
\end{kor}
\begin{kor}Für zyklische Gruppen $G, H$ gilt $G = H \iff \#G = \#H$
\end{kor}
\begin{ubng*}
\begin{enumerate}[(a)]
  \item $\faktor G{[G, G]}$ ist eine abelsche Gruppe.
  \item Für $N \trianglelefteq G$ gilt:
        $$\faktor GN \text{ abelsch} \iff [G, G] \le N$$
\end{enumerate}
\end{ubng*}
\subsection*{Einschub: Faktorringe}
\begin{defi}[\textbf{Ideal}]
  Sei $R$ ein kommutativer Ring. $I \subseteq R$ heißt Ideal wenn
  \begin{enumerate}[(i)]
    \item $I$ ist Untergruppe von $(R, 0, +)$
    \item $RI := \{ri \mid r \in R, i \in I\} \subseteq I$
  \end{enumerate}
\end{defi}
\begin{bsp*}
  \begin{enumerate}[1)]
    \item $\Bbb Zn \subseteq \Bbb Z$ ist ein Ideal $\forall n \in \Bbb Z$.
    \item $Ra \subseteq R$ für $a \in R$ ist ein $Ideal$ von $R$.
  \end{enumerate}
\end{bsp*}

\begin{satz}
  Sei $R$ ein kommutativer Ring, $I \subseteq R$ ein Ideal, und $\faktor RI = \{r + I \mid r \in R\}$ die Nebenklassenmenge von $R$ modulo $I$ (für die Gruppe $(R, 0, +)$). Dann:
  \begin{enumerate}[(a)]
    \item Die Verknüpfungen $$+ : \faktor RI \times \faktor RI \longrightarrow \faktor RI, (r+I, s+I) \longmapsto (r+s)+I$$
          $$\cdot : \faktor RI \times \faktor RI \longrightarrow \faktor RI, (r+I, s+I) \longmapsto rs+I$$
          sind wohl-definiert auf $\faktor RI$

    \item $(\faktor RI, \overline 0, \overline 1, +, \cdot)$ ist ein kommutativer Ring ($\overline r := r + I$ Notation für die Klasse von $r$) der Restklassenring von $R$ modulo $I$.
    \item $\pi : R \longrightarrow \faktor RI, r \longmapsto r+I$ ist ein surjektiver Ringhomomorphismus.
  \end{enumerate}
\begin{proof}[Beweis]
\begin{enumerate}[(a)]
  \item ``$+$'' wohl-definiert folgt aus Satz 44. ($I \subseteq (R, 0, +)$ Ideal!)

        ``$\cdot$''  wohl-definiert: Gelte $a+I = a'+I$ und $b+I = b'+I$.
        $$\implies a'b' + I = ab + aj + bi + ij + I = ab + I$$
  \item (Übung)
  \item Wie in 45 (d)
\end{enumerate}
\end{proof}
\end{satz}
\subsection*{Die Isomorphiesätze}
\begin{satz}[\textbf{Erster Isomorphiesatz}]
  Sei $G$ eine Gruppe, $N \trianglelefteq G$ ein Normalteiler und $H \le G$ eine Untergruppe, dann gelten:
  \begin{enumerate}[(a)]
    \item $HN = \{hn \mid h \in H, n \in N\} \subseteq G$ ist ein Untergruppe.
    \item $H \cap N \subseteq H$ ist ein Normalteiler (und (Übung) $N \trianglelefteq HN$)
    \item Die folgende Abbildung ist wohl-definiert und ein Gruppenisomorphismus $$\faktor H{H \cap N} \longrightarrow \faktor {HN}N, h(H\cap N) \longmapsto hN$$
  \end{enumerate}
\begin{proof}[Beweis]
\begin{enumerate}[(a)]
  \item Seien $hn, h'n' \in HN$, dann:
        $$(h'n')(hn)^{-1}=h'\underbrace{n'n^{-1}h^{-1}}_{\in Nh^{-1} \underset{N \trianglelefteq G} = h^{-1}N} = h'h^{-1}\tilde n \underset{H \text{ U.G.}}= (h'h^{-1})\tilde n \in HN$$
        und $e = ee = HN$
  \item Zu zeigen: für $h \in H$ gilt $h(H \cap N)h^{-1} \subseteq H \cap N$

        Dazu:
        $$\begin{matrix}
            h (H \cap N)h^{-1} \subseteq hHh^{-1} = H \\
            h (H \cap N)h^{-1} \subseteq hNh^{-1} \underset{N \trianglelefteq G}= N
          \end{matrix} \implies h(H \cap N)h^{-1} \subseteq H \cap N.$$

  \item Betrachte die Verkettung von Gruppenhomomorphismen
        $$\varphi: H \xrightarrow[h \longmapsto h]{\text{Inklusion}} HN \xrightarrow[x \longmapsto xN]{} \faktor{HN}N$$
        dann ist $\varphi$ ein Gruppenautomorphismus.

        $\varphi$ ist surjektiv: Jede Klasse in $\faktor{HN}N$ ist von der Form $$hnN = \underbrace{hN}_{=\varphi(h)}$$
        für ein $h \in H$. Nach Homomorphiesatz: nur noch zu zeigen $\mathrm{Kern}(\varphi) = H \cap N$:
        für $h \in H$: $$h \in \mathrm{Kern}(\varphi) \iff \varphi(h) = eN \iff hN = eN \underset{44(c)}\implies h \in N \underset{h \in H}\implies h \in N \cap H$$
        Umgekehrt: $h \in N \cap H \implies h \in N \implies hN = eN = N$.
\end{enumerate}
\end{proof}
\end{satz}

\begin{satz}[\textbf{Zweiter Isomorphiesatz}]
  Sei $G$ eine Gruppe und $N \trianglelefteq G$ ein Normailteiler, und sei $\pi : G \longrightarrow \faktor GN, g \longmapsto \overline g = gN $ die Faktorabbildung.
  \begin{enumerate}[(a)]
    \item Sei $X:= \{H \le G \mid N \subseteq H\}$, und sei $\overline X := \left\{\overline H \le \faktor GN\right\}$, dann ist die Abbildung $$\psi : X \longrightarrow \overline X, H \longmapsto \pi(H) = \faktor HN =: \overline H$$
          eine Bijektion mit inverser Abbildung $$\nu: \overline X \longrightarrow X, \overline H \longmapsto \pi^{-1}(\overline H).$$
          Dabei gilt:
          $$X \ni H \trianglelefteq G \iff \overline X \ni \pi(H) \trianglelefteq \faktor GN$$
    \item Ist $H \in X$ ein Normalteiler von $G$, so ist
          $$\faktor GH \longrightarrow \faktor{\left(\faktor GN\right)}{\left(\faktor HN \right)}, g \longmapsto \underbrace{\overline g}_{gN} \underbrace{\overline H}_{\pi(H)}$$
          wohl-definiert und ein Gruppenisomorphismus.
  \end{enumerate}
\begin{proof}[Beweis]
  \begin{enumerate}[(a)]
    \item Nach Proposition 33 sind $\psi$ und $\nu$ wohl-definiert.
\begin{itemize}

          \item $\nu \circ \psi = \mathrm{id}_{X}$: Sei $H \le G$ mit $N \subseteq H$, zu zeigen ist $\pi^{-1}(\pi(H)) = H$. Es gilt:
          $$g \in \pi^{-1}(\pi(H)) \iff \pi(g) \in \pi(H) \iff gN \in \bigcup_{h \in H}hN$$
          $$\iff \exists h \in H : gN = hN \underset{44(c)}\implies h^{-1}g \in N \subseteq H \implies g \in hH = H.$$
          (``$\impliedby$'' klar: $g \in H \implies g \in \pi^{-1}(\pi(H))$).

          \item $\psi \circ \nu = \mathrm{id}_{\overline X}$: Für $\overline H \in \overline X$ (d.h. $\overline H \le \faktor GN$) ist zu zeigen $\pi(\pi^{-1}(\overline H)) = \overline H$. Dies gilt, denn $\pi$ ist surjektiv.

  \item Schließlich: Sei $H \in X$, zu zeigen ist $H \trianglelefteq G \iff \pi(H) \trianglelefteq \faktor GN$
        $$H \trianglelefteq G \iff \forall g \in G : gHg^{-1} \subseteq H$$
        $$\underset{\pi: G \to \overline G \text{ surj.}}\implies \forall \overline g \in \faktor GN: \overline g \pi(H)\overline g \subseteq \pi(H) \implies \pi(H) \trianglelefteq \overline G$$
        Umgekehrt: Falls $\pi(H) \trianglelefteq \trianglelefteq \overline G$ und $g \in G$:
        $$\pi(gHg^{-1}) = \overline g\pi(H)\overline g^{-1} \le \pi(H)$$
        $$\implies gHg^{-1} \subseteq \pi^{-1}(\pi(gHg^{-1})) \subseteq \pi^{-1}(\pi(H)) \underset{\nu \circ \psi = \mathrm{id}_{X}} = H$$
\end{itemize}
    \item Sei $H \trianglelefteq G$ ein Normalteiler mit $N \subseteq H$, so dass nach (a)
          $$\overline H = \underbrace{\faktor HN}_{\pi(H)} \trianglelefteq \underbrace{\faktor GN}_{\pi(G)}$$
          ein Normalteiler ist. Betrachte den verketteten Gruppenautomorphismus
          $$\varphi : G \xrightarrow[g \longmapsto gN]{\pi}\faktor GN \xrightarrow[\overline g \longmapsto \overline g \overline H]{\pi'} \faktor{\left(\faktor GN\right)}{\left(\faktor HN\right)}$$
          $\pi, \pi'$ sind surjektive Gruppenhomomorphismen nach Satz 44(d) $\implies$ die Verkettung $\varphi$ ist ein surjektiver Gruppenhomomorphismus.

          Nach Homomorphiesatz für Gruppen bleibt zu zeigen: $\mathrm{Kern}(\varphi) = H$:
          $$g \in \mathrm{Kern}(\varphi) \underset{\pi'(\pi(g))=e}\iff\pi(g) \in \mathrm{Kern}(\pi') \iff gN \in \faktor HN$$
          $$\iff gN \subseteq H \underset{N \le H}\iff g \in H.$$
  \end{enumerate}
\end{proof}
\end{satz}

\subsection*{(Semi-)direkte Produkte}%
\label{sec:Direkte Produkte}
\begin{lemm}[Übung]
  Seien $(G_{1}, e_{1}, \circ_{1})$ und $(G_{2}, e_{2}, \circ_{2})$ Gruppen, dann ist $G = (G_{1} \times G_{2}, (e_{1}, e_{2}), \circ)$ eine Gruppe mit
  $$(g_{1}, g_{2}) \circ (h_{1}, h_{2}) = (g_{1} \circ h_{1}, g_{2} \circ h_{2})$$
  Analog für $k \ge 2$ Faktoren. Dabei sind $G_{1} \times \{e_{2}\} \trianglelefteq G$ und $\{e_{1}\} \times G_{2} \trianglelefteq G$ Nullteiler von $G$.
\end{lemm}
\begin{defi}[\textbf{Direktes Produkt}] Die Gruppe $G$ aus Lemma 52 heißt das direkte Produkt von $G_{1}$ und $G_{2}$, Notation $G_{1} \times G_{2}$.
\end{defi}
\begin{bsp*}
$$(\Bbb R^{n}, \underline 0, +) = (\Bbb R, 0, +) \times \cdots \times (\Bbb R, 0, +) = \bigtimes_{i = 1}^{n}(\Bbb R, 0, +)$$
\end{bsp*}

\begin{prop}
  Sei $G$ eine Gruppe, seien $N_{1}, N_{2} \trianglelefteq G$ Nullteiler mit $N_{1} \cap N_{2} = \{e\}$, dann gelten:
  \begin{enumerate}[(a)]
    \item $\forall n_{1} \in N_{1}, n_{2} \in N_{2}: n_{1}n_{2} = n_{2}n_{1}$
    \item $N_{1}N_{2} \trianglelefteq G$ ist ein Normalteiler in G
    \item $\psi : N_{1} \times N_{2} \to N_{1}N_{2}, (n_{1},n_{2}) \mapsto n_{1}n_{2}$ ist ein Gruppenisomorphismus. (Insbesondere gilt $\#N_{1}N_{2} = \#N_{1} \#N_{2}$)
  \end{enumerate}
  Zusatz: Gilt $G = N_{1}N_{2}$, so folgt $G \cong N_{1} \times N_{2}$ via $\psi$.
  \begin{proof}[Beweis]
\begin{enumerate}[(a)]
  \item  Seien $n_{1} \in N_{1}, n_{2} \in N_{2}$, setze $x = n_{1}n_{2}n_{1}^{-1}n_{2}^{-1}$.
        Nun: $$x = (n_{1}n_{2}n_{1}^{-1})n_{2}^{-1} \in (n_{1}N_{2}n_{1}^{-1})N_{2} \subseteq N_{2}N_{2} = N_{2}$$
        analog $$x = n_{1}(n_{2}n_{1}^{-1}n_{2}^{-1}) \in N_{1}(n_{2}N_{1}n_{2}^{-1}) \overset{N_{2} \trianglelefteq G}\subseteq N_{1}N_{1} = N_{1}$$
        damit ist $x \in N_{1} \cap N_{2} = \{e\} \implies x = e \implies n_{1}n_{2} = n_{2}n_{1}$.

  \item Für $g \in G$:
        $$gN_{1}N_{2}g^{-1}= gN_{1}g^{-1}gN_{2}g^{-1} \subseteq  N_{1}N_{2}$$

  \item $\psi$ ist wohl-definiert: klar. $\psi$ ein Gruppenhomomorphismus folgt aus (a)
        $$\psi((n_{1}, n_{2}) \circ (n_{1}', n_{2}')) = \psi((n_{1} \circ n_{1}', n_{2} \circ n_{2}')) = n_{1}n_{1}' n_{2}n_{2}'$$
        $$\underset{(a)}= n_{1}n_{2}n_{1}'n_{2}' = \psi(n_{1}, n_{2}) \circ \psi(n_{1}', n_{2}')$$
        $\{(e, e)\} = \mathrm{Kern}(\psi)$:
        $$\psi(n_{1}, n_{2}) = e \iff n_{1}n_{2} = e \iff n_{1} = n_{2}^{-1} \in N_{1}\cap N_{2} = \{e\}$$
        $$\iff n_{1} = n_{2} = e$$
$\mathrm{Bild}(\psi) = N_{1}N_{2}$.
\end{enumerate}
  \end{proof}
\end{prop}

\begin{kor}[Übung]
  Sei $G$ eine endliche Gruppe. Seien $N_{1}, ..., N_{k} \trianglelefteq G$ Normalteiler von $G$ und gelte:
  \begin{enumerate}[(i)]
    \item $\forall i \ne j : \mathrm{ggT}(\#N_{i}, \#N_{j}) = 1$
    \item $\prod_{j=1}^k \# N_{j} = \#G$
  \end{enumerate}
  Dann ist $$\psi : \bigtimes_{j=1}^{k}N_{j} \longrightarrow G, (n_{1}, ..., n_{k}) \longmapsto n_{1}\cdot ... \cdot n_{k} = \prod_{j=1}^{k}n_{j}$$
  ein Gruppenisomorphismus.
\item
\end{kor}
\begin{ubng*}
Spezialfall: $n = \prod_{i=1}^{k}p_{i}^{f_{i}}$ für $p_{1}, ..., p_{k}$ paarweise verschiedene Primzahlen, dann gilt:
$$\bigtimes_{i}^{k}\faktor{\Bbb Z}{(p_{i}^{f_{i}})} \cong \faktor{\Bbb Z}{(n)}$$
ist Folge von Korollar 55.
\end{ubng*}

\begin{lemm}
  Seien $H=(H, e_{H}, \circ_{H}), N=(N, e_{N}, \circ_{N})$ Gruppen und sei $\varphi : H \to \mathrm{Aut}(N)$ ein Gruppenhomomorphismus. Definiere
  $$G:= N \rtimes H:= N \rtimes_{\varphi} H = (N \times H, \underbrace{(e_{n},e_{H})}_{=:e}, \circ)$$
  mit $\circ$ der Verknüpfung auf $G$ definiert durch
  $$(n_{1},h_{1})\circ (n_{2},h_{2}) = (n_{1} \circ_{N} \varphi(h_{1})(n_{2}), h_{1} \circ_{H} h_{2})$$
  Dann ist $G$ eine Gruppe und es gelten:
  \begin{itemize}
\item $N':= \{(n, e_{H}) \mid n \in N\} \cong N$ ist ein Normalteiler in $G$,
    \item $H':= \{(e_{N}, h) \mid h \in H\} \cong H$ ist eine Untergruppe von $G$,
    \item $N'H' = G$ und $N' \cap H' = \{e\}$,
    \item $G \to H, (n,h) \mapsto h$ ist ein Gruppenepimorphismus (surj.) mit Kern $N'$.
  \end{itemize}
\end{lemm}
%Tausendste Zeile WeeweeweeW!
\begin{defi}[\textbf{Semi-direktes Produkt}]
  Die Gruppe $G = N \rtimes H$ heißt das semi-direkte Produkt von $N$ mit $H$ (bezüglich $\varphi$).
\end{defi}
\begin{satz}
  Sei $G$ eine Gruppe, $N \trianglelefteq G$ ein Normalteiler, $H \le G$ eine Untergruppe, dann gelten:
  \begin{enumerate}[(a)]
    \item $\varphi : H \to \mathrm{Aut}(N), h \mapsto (\underbrace{c_{h}|_{N}: N \to N, n \mapsto hnh^{-1}}_{\text{Konjugation mit }h})$ ist wohl-definiert und ein Gruppenhomomorphismus.
    \item Gelten zusätzlich (i) $NH = G$, (ii) $N \cap H = \{e\}$, so ist
          $$\psi : N \rtimes_{\varphi} H \to G, (n,h) \mapsto n \circ_{G} h$$
          ein Gruppenisomorphismus.
  \end{enumerate}
  \begin{proof}[Beweis]
    Siehe Jantzen, Schwermer - Algebra. %TODO-REFERENCE
  \end{proof}
\end{satz}

\begin{bspe*}\item
  \begin{enumerate}[1.]
    \item Seien $A_{n} = \mathrm{Kern}(\mathrm{sign}: S_{n} \to \{\pm1\})$ die Untergruppe der geraden Permutationen und $\tau$ eine beliebige Transposition, dann gilt:
          $$S_{n} \cong A_{n} \rtimes \{\mathrm{id}, \tau\}$$
    \item $V$ Sei ein endlich dimensionaler euklidischer Vektorraum und $\sigma \in \mathrm{O}(V)$ eine Spiegelung, dann gilt
          $$\mathrm O(V) \cong \mathrm{SO}(V) \rtimes \{\mathrm{id}, \sigma\}$$
    \item Sei $K$ ein Körper, dann gilt
          $$\GL_{n}(K) \cong \SL_{n}(K) \rtimes H \cong \mathrm{SL}_{n}(K) \rtimes K^{\times}$$
          wobei
          \[H = \left\{ \begin{pmatrix}
                          1 &0&\cdots& 0 \\
                          0 & 1&\cdots & 0\\
                          \vdots &\vdots & \ddots& \vdots\\
                          0 & 0& \cdots& 1
                        \end{pmatrix} \ \ \middle\vert\ \  a \in K^{\times}\right\} \cong K^{\times}
          \]
    \item Sei $\sigma \in A_{4}$ ein 3-Zykel, z.B. $\sigma = \begin{pmatrix} 1 & 2 & 3 & 4 \\ 2 & 3 & 1 & 4 \end{pmatrix}$, und $V$ ist die kleinsche Vierergruppe
          $$V = \{\id, (1\ 2)(3\ 4), (1\ 3)(2\ 4), (1\ 4)(2\ 3)\} \trianglelefteq A_{4},$$
          dann gilt $$A_{4} \cong V \rtimes \{\id, \sigma, \sigma^{2}\}$$
  \end{enumerate}
  \begin{proof}[Beweis] (Übung) eventuell noch 12 Tage warten.
  \end{proof}
\end{bspe*}
\end{document}
